\documentstyle[12pt]{article}
\pagestyle{empty}
\setlength{\oddsidemargin}{.15in}
\setlength{\evensidemargin}{.15in}
\setlength{\textwidth}{6in}
\setlength{\textheight}{9.25in}
\setlength{\topmargin}{-.1in}
\setlength{\headheight}{0in}
\setlength{\headsep}{0in}
\setlength{\parskip}{15pt}
\setlength{\labelsep}{10pt}
\setlength{\parindent}{0pt}
\setlength{\medskipamount}{3ex} 
\setlength{\smallskipamount}{1ex}
\def\binom#1#2{{#1\choose#2}}
\def\dg{\raisebox{.15pt}{$^{\circ}$}}
\newenvironment{List}{% 
    \begin{list}{}{\setlength{\labelwidth}{.15in}
    \setlength{\leftmargin}{.55in}
    \setlength{\rightmargin}{.25in}
    \setlength{\topsep}{0pt}
    \setlength{\partopsep}{0pt} 
    }}{\end{list}}
\newtheorem{fact}{Proposition}
\begin{document}

\setlength{\baselineskip}{.24in}
\begin{center}
${\bf 24}^{\mbox{\bf th}}$ {\bf United States of America
Mathematical Olympiad} \\[.1in]
{\bf April 27, 1995}\\
{\bf Time Limit: 3}${{\bf 1}\over{\bf 2}}$ {\bf
hours}
\end{center}
\begin{enumerate}
\item %AND3 
Let $\, p \,$ be an odd prime.  The sequence $(a_n)_{n \geq 0}$ is
defined as follows: $\, a_0 = 0, $ $a_1 = 1, \, \ldots,
\, a_{p-2} = p-2 \,$ and, for all $\, n \geq p-1, \,$ $\, a_n \,$
is the least positive integer that does not form an arithmetic
sequence of length $\, p \,$ with any of the preceding terms.
Prove that, for all $\, n, \,$ $\, a_n \,$ is the number obtained
by writing $\, n \,$ in base $\, p-1 \,$ and reading the result 
in base $\, p$.

\item %PNN1
A calculator is broken so that the only keys that still work are
the $\, \sin, \; \cos, $  $\tan, \; \sin^{-1}, \; \cos^{-1}, \,$
and $\, \tan^{-1} \,$ buttons.  The display initially shows 0.
Given any positive rational number $\, q, \,$ show that pressing
some finite sequence of buttons will yield $\, q$.   Assume that
the calculator does real number calculations with infinite
precision.  All functions are in terms of radians.


\item %HUD2
Given a nonisosceles, nonright triangle $\, ABC, \,$ let
$\, O \,$ denote the center of its circumscribed circle, and let
$\, A_1, \, B_1, \,$ and $\, C_1 \,$ be the midpoints of
sides $\, BC, \, CA, \,$ and $\, AB, \,$ respectively.
Point $\, A_2 \,$ is located on the ray
$\, OA_1 \,$ so that $\, \Delta OAA_1 \,$ is similar
to $\, \Delta OA_2A$.  Points $\, B_2 \,$
and $\, C_2 \,$ on rays $\, OB_1 \,$ and $\, OC_1, \,$
respectively, are defined similarly.  
Prove that lines $\, AA_2, \, BB_2, \,$
and $\, CC_2 \,$ are concurrent, i.e. these three lines intersect
at a point.

\item %PNN3
Suppose $\, q_0, \, q_1, \,  q_2, \ldots \; \,$ is an infinite
sequence of integers satisfying the following two conditions:
\begin{List}
\item[(i)]  $\, m-n \,$ divides $\, q_m - q_n \,$ for
$\, m > n \geq 0,$
\item[(ii)] there is a polynomial $\, P \,$ such that
$\, |q_n| < P(n) \,$ for all $\, n$.
\end{List}
Prove that there is a polynomial $\, Q \,$ such that
$\, q_n = Q(n) \,$ for all $\, n$.

\item %ROU2
Suppose that in a certain society, each pair of persons can be
classified as either {\em amicable} or {\em hostile}. We shall say
that each member of an amicable pair is a {\em friend} of the
other, and each member of a hostile pair is a {\em foe} of the
other.  Suppose that the society has $\, n \,$ persons and $\, q
\,$ amicable pairs, and that for every set of three persons, at
least one pair is hostile.  Prove that there is at least one
member of the society whose foes include $\, q(1 - 4q/n^2) \,$ or
fewer amicable pairs.  
\end{enumerate}
\vspace*{\fill}
\begin{center}
{\footnotesize Copyright \copyright \hspace{.05in} Committee on
the American
Mathematics Competitions, \\ Mathematical Association of America}
\end{center}
\end{document}

