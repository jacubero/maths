\documentclass[12pt]{article}
\usepackage{amsfonts}

\pagestyle{empty}
\setlength{\oddsidemargin}{.15in}
\setlength{\evensidemargin}{.15in}
\setlength{\textwidth}{6in}
\setlength{\textheight}{9.25in}
\setlength{\topmargin}{.2in}
\setlength{\headheight}{0in}
\setlength{\headsep}{0in}
\setlength{\parskip}{20pt}
\setlength{\labelsep}{10pt}
\setlength{\parindent}{0pt}
\setlength{\medskipamount}{3ex} 
\setlength{\smallskipamount}{1ex}
\def\binom#1#2{{#1\choose#2}}
\def\dg{\raisebox{.15pt}{$^{\circ}$}}
\def\RR{{\Bbb R}}
\newenvironment{List}{% 
    \begin{list}{}{\setlength{\labelwidth}{.15in}
    \setlength{\leftmargin}{.55in}
    \setlength{\rightmargin}{.25in}
    \setlength{\topsep}{0pt}
    \setlength{\partopsep}{0pt} 
    }}{\end{list}}
\newtheorem{fact}{Proposition}
\begin{document}

\setlength{\baselineskip}{.24in}
\begin{center}
${\bf 25}^{\mbox{\bf th}}$ {\bf United States of America
Mathematical Olympiad} \\[.1in]
{\bf Part I \hspace{.25in} 9 a.m. - 12 noon}\\[.05in]
{\bf May 2, 1996}
\end{center}
\begin{enumerate}

\vspace*{.3in}

\item %AND2
Prove that the average of the numbers $n \sin n^{\circ} \; 
(n = 2,4,6,\ldots,180)$ is $\cot 1^{\circ}$.

\item %AND4
For any nonempty set $S$ of real numbers, let $\sigma(S)$ denote
the sum of the elements of $S$. Given a set $A$ of $n$ positive
integers, consider the collection of all distinct sums $\sigma(S)$
as $S$ ranges over the nonempty subsets of $A$.  Prove that this
collection of sums can be partitioned into $n$ classes so that in
each
class, the ratio of the largest sum to the smallest sum does not
exceed 2.

\item %PNN4
Let $ABC$ be a triangle.  Prove that there is a line $\ell$ (in
the plane of triangle $ABC$) such that the intersection of the
interior of triangle $ABC$ and the interior of its reflection
$A'B'C'$ in $\ell$ has area more than $2/3$ the area of triangle
$ABC$.

\end{enumerate}
\vspace*{\fill}
\begin{center}
{\footnotesize Copyright \copyright \hspace{.05in} Committee on
the American Mathematics Competitions, \\ 
Mathematical Association of America}
\end{center}

\newpage
\begin{center}
${\bf 25}^{\mbox{\bf th}}$ {\bf United States of America
Mathematical Olympiad} \\[.1in]
{\bf Part II \hspace{.25in} 1 p.m. - 4 p.m.}\\[.05in]
{\bf May 2, 1996}
\end{center}

\vspace*{.3in}

\begin{enumerate}
\setcounter{enumi}{3}

\item %KED2 
An $n$-term sequence $(x_1, x_2, \ldots, x_n)$ in which each
term is either 0 or 1 is called a {\em binary sequence of length}
$n$.  Let $a_n$ be the number of binary sequences of length $n$
containing no three consecutive terms equal to 0, 1, 0 in that
order.  Let $b_n$ be the number of binary sequences
of length $n$ that contain no four consecutive terms equal to 
0, 0, 1, 1 or 1, 1, 0, 0 in that order. Prove that 
$b_{n+1} = 2a_n$ for all positive integers $n$.

\item %AND5
Triangle $ABC$ has the following property: there is an
interior point $P$ such that $\angle PAB = 10^{\circ}$,
$\angle PBA = 20^{\circ}$, $\angle PCA = 30^{\circ}$, 
and $\angle PAC = 40^{\circ}$.  Prove that triangle
$ABC$ is isosceles.

\item %STG2
Determine (with proof) whether there is a subset $X$ of the
integers with the following property: for any integer $n$
there is exactly one solution of $a + 2b = n$ with
$a,b \in X$.

\end{enumerate}
\vspace*{\fill}
\begin{center}
{\footnotesize Copyright \copyright \hspace{.05in} Committee on
the American Mathematics Competitions, \\ 
Mathematical Association of America}
\end{center}
\end{document}