\documentclass[12pt]{article}
\usepackage{amsfonts}

\setlength{\textheight}{9.25in}
\setlength{\textwidth}{6.5in}
\setlength{\topmargin}{0.0in}
\setlength{\headheight}{0.0in}
\setlength{\headsep}{0.0in}
\setlength{\leftmargin}{0.0in}
\setlength{\oddsidemargin}{0.0in}
\setlength{\parindent}{1pc}

\def\binom#1#2{{#1 \choose #2}}
\def\mymod#1{\,(\mbox{mod}\,\, #1)}
\def\pmatrix#1{\left( \begin{array}{ccc} #1 \end{array} \right)}
%Structural macros
\def\head#1{{\bf \noindent \newline #1:  } \nopagebreak}
\def\map#1#2#3{#1\!: #2\to #3}
\def\benum{\begin{enumerate}}
\def\eenum{\end{enumerate}}
\def\soln{\head{Solution}}
\def\first{\head{First Solution}}
\def\second{{\bf \noindent \newline Second Solution:  } \nopagebreak}
\def\third{{\bf \noindent \newline Third Solution:  } \nopagebreak}
\def\note{{\bf \noindent \newline Note:  } \nopagebreak}
\def\be{\begin{enumerate}}
\def\ee{\end{enumerate}}
\def\beq{\begin{equation}}
\def\eeq{\end{equation}}
\def\beqa{\begin{eqnarray*}}
\def\eeqa{\end{eqnarray*}}
\def\bea{\begin{array}}
\def\eea{\end{array}}
\def\ii{\item}
\def\ds{\dsp}
\def\seqa{a_{1}, \dots, a_{n}}
\def\seqb{b_{1}, \dots, b_{n}}
\def\seqx{x_{1}, \dots, x_{n}}
\def\lf{\lfloor}
\def\rf{\rfloor}
\def\Lp{\left(}
\def\Rp{\right)}
\def\L[{\left[}
\def\R]{\right]}
\def\alf{\alpha}
\def\lcm{\mbox{lcm}}
\def\reseteq{\setcounter{equation}{0}}
\def\Avec{{(A_{1}, \ldots, A_{n})}}
\def\Acup{{A_{1} \cup A_{2} \cup \cdots \cup A_{n}}}
\def\crn{{\sqrt[3]{n}}}
\def\ang{\angle}
\def\ep{\epsilon}
\def\dsp{\displaystyle}
\def\oar{\overrightarrow}
\newtheorem{lemma}{Lemma}

%\newcounter{felist}
%\def\bfe{\begin{list}{\alph{felist}}{\usecounter{fe}
\def\bi{\begin{itemize}}
\def\ei{\end{itemize}}
\def\dd{\dots}
\def\half{\frac{1}{2}}
\def\sang{\sin \angle}


\def\binom#1#2{{#1 \choose #2}}
\def\bZ{\mathbb{Z}}
\def\th{^{\scriptstyle\mbox{th}}}

\def\head#1{\begin{center} {\bf #1} \end{center}}

\pagestyle{empty}
\begin{document}
%\input epsf

\begin{center}
${\bf 30\th}$ {\bf United States of America Mathematical Olympiad}
\\[.1in]
{\bf Part I \hspace{.25in}  9 a.m. - 12 p.m.}\\[.05in]
{\bf May 1, 2001}
\end{center}

\bigskip 

\be
\ii %KED5
Each of eight boxes contains six balls. Each ball has been colored with
one of $n$ colors, such that no two balls in the same box are the same
color, and no two colors occur together in more than one box. Determine,
with justification, the smallest integer $n$ for which this is possible.


\ii %FEN2
Let $ABC$ be a triangle and let $\omega$ be its incircle. Denote by
$D_1$ and $E_1$ the points where $\omega$ is tangent to sides $BC$ and
$AC$, respectively. Denote by $D_2$ and $E_2$ the points on sides $BC$
and $AC$, respectively, such that $CD_2=BD_1$ and $CE_2=AE_1$, and
denote by $P$ the point of intersection of segments $AD_2$ and $BE_2$.
Circle $\omega$ intersects segment $AD_2$ at two points, the closer of
which to the vertex $A$ is denoted by $Q$. Prove that $AQ=D_2P$.
 
\ii %AND2
Let $a, b$, and $c$ be nonnegative real numbers such that
\[
a^2+b^2+c^2 + abc = 4.
\]
Prove that
\[
0\le ab+bc+ca-abc \le 2.
\]

\ee

\vspace{80 mm}

{\small
\begin{center}
Copyright \copyright \ \ Committee on the American  Mathematics  Competitions,\\
Mathematical Association of America
\end{center}
}
\newpage

\begin{center}
${\bf 30\th}$ {\bf United States of America Mathematical Olympiad}
\\[.1in]
{\bf Part II \hspace{.25in}  1 p.m. - 4 p.m.}\\[.05in]
{\bf May 1, 2001}
\end{center}

\bigskip

\be
\ii[4.] %AND3
Let $P$ be a point in the plane of triangle $ABC$ such that the segments
$PA$, $PB$, and $PC$ are the sides of an obtuse triangle. Assume that
in this triangle the obtuse angle opposes the side congruent to $PA$.
Prove that $\angle BAC$ is acute. 

\ii[5.] %PNN2
Let $S$ be a set of integers (not necessarily positive) such that
\begin{itemize}
\ii
[(a)]
there exist $a,b \in S$ with $\gcd(a,b)=\gcd(a-2,b-2)=1$;
\ii
[(b)]
if $x$ and $y$ are elements of $S$ (possibly equal), then $x^2-y$
also belongs to $S$.
\end{itemize}
Prove that $S$ is the set of all integers.

\ii[6.] %PNN1
Each point in the plane is assigned a real number such that, for any
triangle, the number at the center of its inscribed circle is equal to
the arithmetic mean of the three numbers at its vertices. Prove that
all points in the plane are assigned the same number.

\ee

\vspace{120mm}
{\small
\begin{center}
Copyright \copyright \ \  Committee on the American  Mathematics  Competitions,\\
Mathematical Association of America
\end{center}
}
\end{document}



