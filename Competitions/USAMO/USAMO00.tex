\documentclass[12pt]{article}
\usepackage{amsfonts}

\setlength{\textheight}{9.25in}
\setlength{\textwidth}{6.5in}
\setlength{\topmargin}{0.0in}
\setlength{\headheight}{0.0in}
\setlength{\headsep}{0.0in}
\setlength{\leftmargin}{0.0in}
\setlength{\oddsidemargin}{0.0in}
\setlength{\parindent}{1pc}

\def\calF{\mathcal{F}}
\def\calG{\mathcal{G}}
\def\calO{\mathcal{O}}
\def\calS{\mathcal{S}}
\def\calT{\mathcal{T}}
\def\ida{\mathfrak{a}}
\def\idb{\mathfrak{b}}
\def\idm{\mathfrak{m}}
\def\idn{\mathfrak{n}}
\def\idp{\mathfrak{p}}
\def\CC{\mathbb{C}}
\def\FF{\mathbb{F}}
\def\N{\mathbb{N}}
\def\NN{\mathbb{N}}
\def\PP{\mathbb{P}}
\def\QQ{\mathbb{Q}}
\def\RR{\mathbb{R}}
\def\R{\mathbb{R}}
\def\Z{\mathbb{Z}}
\def\ZZ{\mathbb{Z}}

\def\binom#1#2{{#1 \choose #2}}
\def\mymod#1{\,(\mbox{mod}\,\, #1)}
\def\pmatrix#1{\left( \begin{array}{ccc} #1 \end{array} \right)}
%Structural macros
\def\head#1{{\bf \noindent \newline #1:  } \nopagebreak}
\def\map#1#2#3{#1\!: #2\to #3}
\def\benum{\begin{enumerate}}
\def\eenum{\end{enumerate}}
\def\soln{\head{Solution}}
\def\first{\head{First Solution}}
\def\second{{\bf \noindent \newline Second Solution:  } \nopagebreak}
\def\third{{\bf \noindent \newline Third Solution:  } \nopagebreak}
\def\note{{\bf \noindent \newline Note:  } \nopagebreak}
\def\be{\begin{enumerate}}
\def\ee{\end{enumerate}}
\def\beq{\begin{equation}}
\def\eeq{\end{equation}}
\def\beqa{\begin{eqnarray*}}
\def\eeqa{\end{eqnarray*}}
\def\bea{\begin{array}}
\def\eea{\end{array}}
\def\ii{\item}
\def\ds{\dsp}
\def\seqa{a_{1}, \dots, a_{n}}
\def\seqb{b_{1}, \dots, b_{n}}
\def\seqx{x_{1}, \dots, x_{n}}
\def\lf{\lfloor}
\def\rf{\rfloor}
\def\Lp{\left(}
\def\Rp{\right)}
\def\L[{\left[}
\def\R]{\right]}
\def\alf{\alpha}
\def\lcm{\mbox{lcm}}
\def\reseteq{\setcounter{equation}{0}}
\def\Avec{{(A_{1}, \ldots, A_{n})}}
\def\Acup{{A_{1} \cup A_{2} \cup \cdots \cup A_{n}}}
\def\crn{{\sqrt[3]{n}}}
\def\ang{\angle}
\def\ep{\epsilon}
\def\dsp{\displaystyle}
\def\oar{\overrightarrow}
\newtheorem{lemma}{Lemma}

%\newcounter{felist}
%\def\bfe{\begin{list}{\alph{felist}}{\usecounter{fe}
\def\bi{\begin{itemize}}
\def\ei{\end{itemize}}
\def\dd{\dots}
\def\half{\frac{1}{2}}
\def\sang{\sin \angle}


\def\binom#1#2{{#1 \choose #2}}
\def\bZ{\mathbb{Z}}

\def\head#1{\begin{center} {\bf #1} \end{center}}

\pagestyle{empty}
\begin{document}
%\input epsf

\begin{center}
${\bf 29^{th}}$ {\bf United States of America Mathematical Olympiad}
\end{center}



\begin{center}
{\bf  Part  I \hspace{ 6mm} 9 a.m. -12 noon}
\end{center}


\begin{center}
{\bf May 2, 2000}
\end{center}

\bigskip 


\be
\ii %PNN1
Call a real-valued function $f$ {\em very convex} if
\[
\frac{f(x) + f(y)}{2} \ge f\Lp \frac{x+y}{2} \Rp + |x-y|
\]
holds for all real numbers $x$ and $y$. Prove that no very convex function exists.
\vspace{5mm}

\ii %AND1
Let $S$ be the set of all triangles $ABC$ for which
\[
5\Lp \frac{1}{AP} + \frac{1}{BQ} + \frac{1}{CR} \Rp - \frac{3}{\min\{ AP, BQ, CR \}} =
\frac{6}{r},
\]
where $r$ is the inradius and $P, Q, R$ are the points of tangency of the incircle with sides
$AB, BC, CA,$ respectively. Prove that all triangles in $S$ are isosceles and similar to one
another.
\vspace{5mm}

 
\ii %Stong3
A game of solitaire is played with $R$ red cards, $W$ white cards, and $B$ blue cards. A player
plays all the cards one at a time. With each play he accumulates a penalty. If he plays a blue
card, then he is charged a penalty which is the number of white cards still in his hand. If he
plays a white card, then he is charged a penalty which is twice the number of red cards still in
his hand. If he plays a red card, then he is charged a penalty which is three times the number of
blue cards still in his hand. Find, as a function of $R, W,$ and $B,$ the minimal total penalty
a player can amass and all the ways in which this minimum can be achieved.

\ee

\vspace{80 mm}

{\small
\begin{center}
Copyright \copyright \ \ Committee on the American  Mathematics  Competitions,\\
Mathematical Association of America
\end{center}
}
\newpage

\begin{center}
${\bf 29^{th}}$ {\bf United States of America Mathematical Olympiad}
\end{center}



\begin{center}
{\bf  Part  II \hspace{ 6mm} 1 p.m. - 4 p.m.}
\end{center}


\begin{center}
{\bf May 2, 2000}
\end{center}

\bigskip

\be
\ii[4.] %SOI4
Find the smallest positive integer $n$ such that if $n$ squares of a $1000 \times 1000$ chessboard are
colored, then there will exist three colored squares whose centers form a right triangle with sides
parallel to the edges of the board.
\vspace{5mm}

\ii[5.] %KED2
Let $A_1A_2A_3$ be a triangle and let $\omega_1$ be a circle in its plane passing through $A_1$ and
$A_2.$ Suppose there exist circles $\omega_2, \omega_3, \dots, \omega_7$ such that for $k = 2, 3, \dots,
7,$ $\omega_k$ is externally tangent to $\omega_{k-1}$ and passes through $A_k$ and $A_{k+1},$  where
$A_{n+3} = A_{n}$ for all $n \ge 1$. Prove that $\omega_7 = \omega_1.$
\vspace{5mm}

\ii[6.] %AND3
Let $a_1, b_1, a_2, b_2, \dots , a_n, b_n$ be nonnegative real numbers. Prove that
\[
\sum_{i, j = 1}^{n} \min\{a_ia_j, b_ib_j\} \le \sum_{i, j = 1}^{n} \min\{a_ib_j, a_jb_i\}.
\] 

\ee

\vspace{120mm}
{\small
\begin{center}
Copyright \copyright \ \  Committee on the American  Mathematics  Competitions,\\
Mathematical Association of America
\end{center}
}
\end{document}



