\documentclass[12pt]{article}
\usepackage{amsfonts}

\setlength{\textheight}{9.25in}
\setlength{\textwidth}{6.5in}
\setlength{\topmargin}{0.0in}
\setlength{\headheight}{0.0in}
\setlength{\headsep}{0.0in}
\setlength{\leftmargin}{0.0in}
\setlength{\oddsidemargin}{0.0in}
\setlength{\parindent}{1pc}

\def\be{\begin{enumerate}}
\def\ee{\end{enumerate}}
\def\ii{\item}

\def\binom#1#2{{#1 \choose #2}}
\def\bZ{\mathbb{Z}}
\def\nd{^{\scriptstyle\mbox{nd}}}
\def\calP{\mathcal{P}}

\pagestyle{empty}
\begin{document}

\begin{center}
${\bf 32\nd}$ {\bf United States of America Mathematical Olympiad} \\[.1in]
%{\bf Lincoln, Nebraska} \\ [.05in]
{\bf Day I \hspace{.25in} 12:30 PM - 5:00 PM}\\[.05in]
{\bf April 29, 2003}
\end{center}
\vspace*{.3in}

\be
\ii [1.] %EEE
%Titu
Prove that for every positive integer $n$ there exists an
$n$-digit number divisible by $5^n$ all of whose digits are odd.


\ii [2.]
% Gal2, MMH
A convex polygon $\calP$ in the plane is dissected into smaller
convex polygons by drawing all of its diagonals. The lengths of
all sides and all diagonals of the polygon $\calP$ are rational
numbers. Prove that the lengths of all sides of all polygons in
the dissection are also rational numbers.

\ii [3.]
%{\bf SUN2} %MMH
Let $n \neq 0$. For every sequence of integers
\[
A = a_0,a_1,a_2,\dots, a_n
\]
satisfying $0 \le a_i \le i$, for $i=0,\dots,n$, define another
sequence
\[
t(A)= t(a_0), t(a_1), t(a_2), \dots, t(a_n)
\]
by setting $t(a_i)$ to be the number of terms in the sequence $A$
that precede the term $a_i$ and are different from $a_i$. Show
that, starting from any sequence $A$ as above, fewer than $n$
applications of the transformation $t$ lead to a sequence $B$ such
that $t(B) = B$.
\ee

\vspace{80mm}

{\small
\begin{center}
Copyright \copyright \ \ Committee on the American  Mathematics  Competitions,\\
Mathematical Association of America
\end{center}
}
\newpage

\begin{center}
${\bf 32\nd}$ {\bf United States of America Mathematical Olympiad} \\[.1in]
%{\bf Lincoln, Nebraska} \\ [.05in]
{\bf Day II \hspace{.25in} 12:30 PM - 5:00 PM}\\[.05in]
{\bf April 30, 2003}
\end{center}

\vfill

\be
\ii [4.]
%T2 and ZF, EMM
Let $ABC$ be a triangle. A circle passing through $A$ and $B$
intersects segments $AC$ and $BC$ at $D$ and $E$, respectively.
Lines $AB$ and $DE$ intersect at $F$ while lines $BD$ and $CF$
intersect at $M$. Prove that $MF = MC$ if and only if $MB\cdot MD
= MC^2$.


\ii [5.]
%T2 and ZF, MMH
Let $a, b, c$ be positive real numbers. Prove that
\[
\frac{(2a+b+c)^2}{2a^2+(b+c)^2} + \frac{(2b+c+a)^2}{2b^2+(c+a)^2} +
\frac{(2c+a+b)^2}{2c^2+(a+b)^2} \le 8.
\]

\ii [6.]
% BAR3, MHH
At the vertices of a regular hexagon are written six nonnegative
integers whose sum is 2003. Bert is allowed to make moves of the
following form: he may pick a vertex and replace the number
written there by the absolute value of the difference between the
numbers written at the two neighboring vertices. Prove that Bert
can make a sequence of moves, after which the number 0 appears at
all six vertices.

\vfill

\ee
{\small
\begin{center}
Copyright \copyright \ \ Committee on the American  Mathematics  Competitions,\\
Mathematical Association of America
\end{center}
}

\end{document}
