\documentclass[12pt]{article}
\usepackage{amssymb, latexsym, amsmath, amsthm}
\pagestyle{empty}
\setlength{\oddsidemargin}{.15in}
\setlength{\evensidemargin}{.15in}
\setlength{\textwidth}{6in}
\setlength{\textheight}{9.25in}
\setlength{\topmargin}{.2in}
\setlength{\headheight}{0in}
\setlength{\headsep}{0in}
\setlength{\parskip}{20pt}
\setlength{\labelsep}{10pt}
\setlength{\parindent}{0pt}
\setlength{\medskipamount}{3ex} 
\setlength{\smallskipamount}{1ex}
\def\binom#1#2{{#1\choose#2}}
\def\dg{\raisebox{.15pt}{$^{\circ}$}}
\def\RR{{\Bbb R}}
\newenvironment{List}{% 
    \begin{list}{}{\setlength{\labelwidth}{.15in}
    \setlength{\leftmargin}{.55in}
    \setlength{\rightmargin}{.25in}
    \setlength{\topsep}{0pt}
    \setlength{\partopsep}{0pt} 
    }}{\end{list}}
\newtheorem{fact}{Proposition}
\begin{document}

\setlength{\baselineskip}{.24in}
\begin{center}
${\bf 26}^{\mbox{\bf th}}$ {\bf United States of America
Mathematical Olympiad} \\[.1in]
{\bf Part I \hspace{.25in} 9 a.m. - 12 noon}\\[.05in]
{\bf May 1, 1997}
\end{center}
\begin{enumerate}

\vspace*{.2in}

\item %JOH2
Let $p_1, p_2, p_3, \ldots$ be the prime numbers listed in
increasing order, and let $x_0$ be a real number between 0
and 1.  For positive integer $k$, define
\[
x_k = 
\begin{cases}
0 & \mbox{if} \; x_{k-1} = 0, \\[.1in]
{\displaystyle \left\{ \frac{p_k}{x_{k-1}} \right\}} & 
\mbox{if} \; x_{k-1} \neq 0,
\end{cases}
\]
where $\{x\}$ denotes the fractional part of $x$.  (The fractional
part of $x$ is given by $x - \lfloor x \rfloor$ where $\lfloor
x \rfloor$ is the greatest integer less than or equal to $x$.)
Find, with proof, all $x_0$ satisfying $0 < x_0 < 1$ 
for which the sequence $x_0, x_1, x_2, \ldots $ eventually becomes
0.

\item %KED4
Let $ABC$ be a triangle, and draw isosceles triangles
$BCD, CAE, ABF$ externally to $ABC$, with $BC, CA, AB$ as their
respective bases.  Prove that the lines through $A,B,C$
perpendicular to the lines $\stackrel{\longleftrightarrow}{EF}, 
\stackrel{\longleftrightarrow}{FD},
\stackrel{\longleftrightarrow}{DE}$, 
respectively, are concurrent.

\item %KED1
Prove that for any integer $n$, there exists a unique polynomial
$Q$ with coefficients in $\{0,1,\ldots,9\}$ such that
$Q(-2) = Q(-5) = n$.

\end{enumerate}
\vspace*{\fill}
\begin{center}
{\footnotesize Copyright \copyright \hspace{.05in} Committee on
the American Mathematics Competitions, \\ 
Mathematical Association of America}
\end{center}

\newpage
\begin{center}
${\bf 26}^{\mbox{\bf th}}$ {\bf United States of America
Mathematical Olympiad} \\[.1in]
{\bf Part II \hspace{.25in} 1 p.m. - 4 p.m.}\\[.05in]
{\bf May 1, 1997}
\end{center}

\vspace*{.2in}

\begin{enumerate}
\setcounter{enumi}{3}

\item %PNN1 
To {\em clip} a convex $n$-gon means to choose a pair of
consecutive sides $AB, BC$ and to replace them by the three
segments $AM, MN$, and $NC$, where $M$ is the midpoint of $AB$
and $N$ is the midpoint of $BC$.  In other words, one cuts off the
triangle $MBN$ to obtain a convex $(n+1)$-gon.  A regular hexagon
${\cal P}_6$ of area 1 is clipped to obtain a heptagon ${\cal
P}_7$.  Then ${\cal P}_7$ is clipped (in one of the seven possible
ways) to obtain an octagon ${\cal P}_8$, and so on.  Prove that no
matter how the clippings are done, the area of ${\cal P}_n$ is
greater than $1/3$, for all $n \geq 6$.

\item %AND3
Prove that, for all positive real numbers $a,b,c$,
\[
(a^3 + b^3 + abc)^{-1} + (b^3 + c^3 + abc)^{-1} 
+ (c^3 + a^3 + abc)^{-1} \leq (abc)^{-1}.
\]

\item %PRO1
Suppose the sequence of nonnegative integers 
$a_1, a_2, \ldots, a_{1997}$ satisfies 
\[
a_i + a_j \leq a_{i+j} \leq a_i + a_j + 1
\]
for all $i,j \geq 1$ with $i + j \leq 1997$.  Show that there
exists a real number $x$ such that $a_n = \lfloor nx \rfloor$ 
(the greatest integer $\leq nx$) for all $1 \leq n \leq 1997$.


\end{enumerate}
\vspace*{\fill}
\begin{center}
{\footnotesize Copyright \copyright \hspace{.05in} Committee on
the American Mathematics Competitions, \\ 
Mathematical Association of America}
\end{center}
\end{document}

