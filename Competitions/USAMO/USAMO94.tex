\documentstyle[12pt]{article}
\pagestyle{empty}
\setlength{\oddsidemargin}{.15in}
\setlength{\evensidemargin}{.15in}
\setlength{\textwidth}{6.05in}
\setlength{\textheight}{9in}
\setlength{\parskip}{15pt}
\setlength{\labelsep}{15pt}
\setlength{\parindent}{0pt}
\setlength{\medskipamount}{3ex} 
\setlength{\smallskipamount}{1ex}
\def\binom#1#2{{#1\choose#2}}
\begin{document}
\setlength{\baselineskip}{.25in}
\begin{center}
${\bf 23}^{\mbox{\bf rd}}$ {\bf United States of America
Mathematical
Olympiad}
\\[.1in]
{\bf April 28, 1994}\\
{\bf Time Limit: 3}${{\bf 1}\over{\bf 2}}$ {\bf
hours}
\end{center}
\begin{enumerate}
\item %AND2
Let $\, k_1 < k_2 < k_3 < \cdots \,$ be positive integers, no two
consecutive, and let $\, s_m = k_1 + k_2 + \cdots + k_m \,$ for 
$\, m = 1,2,3, \ldots  \; \;$.  Prove that, for each positive
integer $\, n, \,$ the interval $\, [s_n, s_{n+1}) \,$ contains 
at least one perfect square.
\item %PRO3
The sides of a 99-gon are initially colored so that consecutive
sides are red, blue, red, blue, $\,\ldots, \,$ red, blue, yellow. 
We make a sequence of modifications in the coloring, changing the
color of one side at a time to one of the three given colors (red,
blue, yellow), under the constraint that no two adjacent sides may 
be the same color.  By making a sequence of such modifications, is
it possible to arrive at the coloring in which consecutive sides
are red, blue, red, blue, red, blue, $\, \ldots, \,$ red, yellow,
blue?
\item %HUD2
A convex hexagon $\, ABCDEF \,$ is inscribed in a circle such 
that $\, AB = CD = EF \,$ and diagonals $\, AD, \; BE, \,$ and 
$\, CF \,$ are concurrent. Let $\, P \,$ be the intersection of
$\, AD \,$ and $\, CE$.  Prove that $\, CP/PE = ( AC/CE )^2$.
\item %PEM2
Let $\, a_1, a_2, a_3, \ldots \,$ be a sequence of positive real
numbers satisfying $\, \sum_{j=1}^n a_j \geq \sqrt{n} \,$ for all
$\, n \geq 1$. Prove that, for all $\, n \geq 1, \,$
\[
\sum_{j=1}^n a_j^2 > \frac{1}{4} \left( 1 + \frac{1}{2} +
\cdots + \frac{1}{n} \right).
\]
\item %KED4
Let $\, |U|, \, \sigma(U) \,$ and $\, \pi(U) \,$ denote the
number of elements, the sum, and the product, respectively, of a
finite set $\, U \,$ of positive integers.  (If $\, U \,$ is the
empty set, $\, |U| = 0, \, \sigma(U) = 0, \, \pi(U) = 1$.)
Let $\, S \,$ be a finite set of positive integers. As usual, let
$\, \binom{n}{k} \,$ denote $\, n! \over k! \, (n-k)!$. Prove that
\[
\sum_{U \subseteq S} (-1)^{|U|} \binom{m - \sigma(U)}{|S|} =
\pi(S)
\]
for all integers $\, m \geq \sigma(S)$.  
\end{enumerate}
\vspace*{\fill}
\begin{center}
{\footnotesize Copyright \copyright \hspace{.05in} Committee on
the American
Mathematics Competitions, \\ Mathematical Association of America}
\end{center}
\end{document}

