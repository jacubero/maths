\documentclass[12pt]{article}
\usepackage{amsfonts}

\setlength{\textheight}{9.25in}
\setlength{\textwidth}{6.5in}
\setlength{\topmargin}{0.0in}
\setlength{\headheight}{0.0in}
\setlength{\headsep}{0.0in}
\setlength{\leftmargin}{0.0in}
\setlength{\oddsidemargin}{0.0in}
\setlength{\parindent}{1pc}

\def\binom#1#2{{#1 \choose #2}}
\def\bZ{\mathbb{Z}}

\def\head#1{\begin{center} {\bf #1} \end{center}}

\pagestyle{empty}
\begin{document}
%\input epsf

\begin{center}
${\bf 27^{th}}$ {\bf United States of America Mathematical Olympiad}
\end{center}



\begin{center}
{\bf  Part  I \hspace{ 6mm} 9 a.m. -12 noon}
\end{center}


\begin{center}
{\bf April 28, 1998}
\end{center}

\bigskip 


\begin{itemize}

\item[1.] Suppose that the set $\{1,2,\cdots, 1998\}$ has been partitioned
into disjoint pairs $\{a_i,b_i\}$ ($1\leq i\leq 999$) so that
for all $i$, $|a_i-b_i|$ equals $1$ or $6$. Prove that the sum
$$|a_1-b_1|+|a_2-b_2|+\cdots +|a_{999}-b_{999}|$$ 
ends in 
the digit $9$.  


\item[2.] Let ${\cal C}_1$ and ${\cal C}_2$ be  concentric circles, with
${\cal C}_2$ in the interior of  ${\cal C}_1$. From a point $A$
on ${\cal C}_1$ one draws the tangent $AB$ to ${\cal C}_2$ ($B\in {\cal C}_2$).
Let $C$ be the second point of intersection 
of $AB$ and ${\cal C}_1$, and 
let   $D$ be the midpoint of 
$AB$. A line passing through $A$ intersects ${\cal C}_2$
at $E$ and $F$ in such a way that the perpendicular  bisectors of 
$DE$ and $CF$ intersect at a point $M$ on $AB$.
Find, with proof,  the ratio $AM/MC$.


 
\item[3.] Let $a_0,a_1,\cdots ,a_n$ be numbers from the interval
$(0,\pi/2)$ such that 
\begin{eqnarray*}
\tan (a_0-\frac{\pi}{4})+ \tan (a_1-\frac{\pi}{4})+\cdots
+\tan (a_n-\frac{\pi}{4})\geq n-1.
\end{eqnarray*}
Prove that
\begin{eqnarray*}
\tan a_0\tan a_1 \cdots \tan a_n\geq n^{n+1}.
\end{eqnarray*}

\end{itemize}

\vspace{110 mm}

{\small
\begin{center}
Copyright \copyright \ \ Committee on the American  Mathematics  Competitions,\\
Mathematical Association of America
\end{center}
}
\newpage

\begin{center}
${\bf 27^{th}}$ {\bf United States of America Mathematical Olympiad}
\end{center}



\begin{center}
{\bf  Part  II \hspace{ 6mm} 1 p.m. - 4 p.m.}
\end{center}


\begin{center}
{\bf April 28, 1998}
\end{center}

\bigskip

\begin{itemize}

% \item[4.] A computer screen shows a $1998 \times 1998$ chessboard
% colored in the usual way.  One can select any rectangular subregion
% (with edges parallel to the board) with a mouse and when the mouse
% button is clicked, the colors in the subregion become inverted (black
% becomes white, white becomes black; see picture).  Find, with proof,
% the minimum number of mouse clicks needed to make the chessboard
% all one color.
 


\item[4.]A computer screen shows a $98 \times 98$ chessboard,
 colored in the usual
 way.
One can select with a mouse any rectangle with sides on the lines of the
chessboard
and click the mouse button: as a result,  the colors in the selected
rectangle switch
(black becomes white, white becomes black). Find, with proof, the minimum
number of mouse clicks needed to make the chessboard all one color.



\item[5.] Prove that for each $n\geq 2$, there is a set $S$ of
$n$ integers such that $(a-b)^2$ divides $ab$ for every distinct
$a,b\in S$. 

\item[6.] Let $n \geq 5$ be an integer. Find the largest integer $k$ (as a
function of $n$) such that there exists a convex $n$-gon
$A_{1}A_{2}\dots A_{n}$ for which exactly $k$ of the
quadrilaterals $A_{i}A_{i+1}A_{i+2}A_{i+3}$ have an inscribed circle.
(Here $A_{n+j} = A_{j}$.)

\end{itemize}

\vspace{130mm}
{\small
\begin{center}
Copyright \copyright \ \  Committee on the American  Mathematics  Competitions,\\
Mathematical Association of America
\end{center}
}



\end{document}



