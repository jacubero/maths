\documentstyle[12pt]{article}
\def\dg{^\circ}
\begin{document}
\begin{center}
${\bf 19}^{\mbox{\bf th}}$ {\bf USA 
Mathematical 
Olympiad} 
\\[.1in] 
{\bf April 24, 1990}\\ 
{\bf Time Limit: 3}${{\bf 1}\over{\bf 2}}$ {\bf 
hours} 
\end{center} 
\begin{enumerate} 
\item
A certain state issues license plates consisting of six digits (from 0 
through 9). The state requires that any two plates differ in at least two 
places. (Thus the plates \fbox{027592} and \fbox{020592} cannot both be 
used.) Determine, with proof, the maximum number of distinct license 
plates that the state can use.
\item
A sequence of functions $\, \{f_n(x) \} \,$ is defined recursively as follows:
\begin{eqnarray*}
f_1(x) &=& \sqrt{x^2 + 48}, \quad \mbox{and} \\
f_{n+1}(x) &=& \sqrt{x^2 + 6f_n(x)} \quad \mbox{for $n \geq 1$.}
\end{eqnarray*}
(Recall that $\sqrt{\makebox[5mm]{}}$
is understood to represent the positive square root.)
For each positive integer $n$, find all real solutions of the equation 
$\, f_n(x) = 2x \,$.
\item 
Suppose that necklace $\, A \,$ has 14 beads and necklace $\, B \,$ has 
19. Prove that for any odd integer $n \geq 1$, there is a way to number 
each of the 33 beads with an integer from the sequence
\[
\{ n, n+1, n+2, \dots, n+32 \}
\]
so that each integer is used once, and adjacent beads correspond to 
relatively prime integers. (Here a ``necklace'' is viewed as a circle in 
which each bead is adjacent to two other beads.)
\item 
Find, with proof, the number of positive integers whose base-$n$ 
representation consists of distinct digits with the property that, except 
for the leftmost digit, every digit differs by $\pm 1$ from some digit 
further to the left.
(Your answer should be an explicit function of $n$ in simplest form.)
\item 
An acute-angled triangle $ABC$ is given in the plane. The circle with 
diameter $\, AB \,$ intersects altitude $\, CC' \,$ and its extension at 
points $\, M \,$ and $\, N \,$, and the circle with diameter $\, AC \,$ 
intersects altitude $\, BB' \,$ and its extensions at $\, P \,$ and $\, Q 
\,$. Prove that the points $\, M, N, P, Q \,$ lie on a common circle.
\end{enumerate} 
\end{document}