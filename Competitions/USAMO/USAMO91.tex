%% Note: this file uses PicTeX.
%% When I processed it, LaTeX complained several times that the
%% command \fiverm was unknown, but to no ill consequence.
\documentstyle[12pt]{article}
\include{pictex}
\setlength{\baselineskip}{.25in}
\begin{document}
\begin{center}
{\bf 20th USA Mathematical Olympiad} \\[.1in]
{\bf April 23, 1991}\\
{\bf Time Limit: 3}${{\bf 1}\over{\bf 2}}$ {\bf
hours}
\end{center}
\begin{enumerate}
\item
In triangle $\, ABC, \,$ angle $\,A\,$ is twice angle $\,B,\,$
angle $\,C\,$ is obtuse, and the three side lengths $\,a,b,c\,$ are
integers.  Determine, with proof, the minimum possible perimeter.
\item
For any nonempty set $\,S\,$ of numbers, let $\,\sigma(S)\,$ and
$\,\pi(S)\,$ denote the sum and product, respectively, of the
elements of $\,S\,$.  Prove that
\[
\sum \frac{\sigma(S)}{\pi(S)} = (n^2 + 2n) - \left(
1 + \frac{1}{2} + \frac{1}{3} + \cdots + \frac{1}{n} \right)
(n+1),
\]
where ``$\Sigma$'' denotes a sum involving all nonempty subsets
$S$ of \newline
$\{1,2,3, \ldots,n\}$.
\item
Show that, for any fixed integer $\,n \geq 1,\,$ the sequence
\[
2, \; 2^2, \; 2^{2^2}, \; 2^{2^{2^2}}, \ldots  (\mbox{mod} \; n)
\]
is eventually constant. 

[The tower of exponents is defined by $a_1 = 2, \; a_{i+1} =
2^{a_i}$.
Also $a_i \; (\mbox{mod} \; n)$ means the remainder which results
from dividing $\,a_i\,$ by $\,n$.]
\vspace*{\fill}
\begin{center}
{\footnotesize Copyright \copyright \hspace{.05in} Committee on
the American
Mathematics Competitions, Mathematical Association of America}
\end{center}
\pagebreak
\item
Let $\, a =(m^{m+1} + n^{n+1})/(m^m + n^n), \,$ where $\,m\,$ and
$\,n\,$
are positive integers.  Prove that $\,a^m + a^n \geq m^m +
n^n$. 

[You may wish to analyze the ratio $\,(a^N - N^N)/(a-N),$ for 
real $\, a \geq 0 \,$ and integer $\, N \geq 1$.]
\item
Let $\, D \,$ be an arbitrary point on side $\, AB \,$ of a given
triangle
$\, ABC, \,$ and let $\, E \,$ be the interior point where 
$\, CD \,$ intersects
the external common tangent to the incircles of triangles $\,
ACD \,$
and $\, BCD$.  As $\, D \,$ assumes all positions between $\, A
\,$ and $\, B \,$,
prove that the point $\, E \, $ traces the arc of a circle.
\end{enumerate}
$$\vbox{\hbox{\beginpicture
\setcoordinatesystem units <1pt,1pt>
\put {$\scriptstyle\bullet$} at  0.00000 0.00000
\put {$A    $} [rt] <-2.12pt,-2.12pt> at  -0.88000 -0.88000
\put {$\scriptstyle\bullet$} at  220.000 0.00000
\put {$B    $} [lt] <2.12pt,-2.12pt> at  220.880 -0.88000
\put {$\scriptstyle\bullet$} at  18.7723 118.523
\put {$C    $} [r] <-3pt,0pt> at  17.8923 118.523
\put {$\scriptstyle\bullet$} at  72.6000 0.00000
\put {$D    $} [t] <0pt,-3pt> at  72.6000 0.00000
\put {$\scriptstyle\bullet$} at  46.3820 57.7292
\put {$E    $} [lb] <2.12pt,2.12pt> at  44.8820 59.7292
\put {$\scriptstyle\bullet$} at  86.6196 67.4129
\put {$\scriptstyle\bullet$} at  24.9758 52.5775
\linethickness=0.6pt \putrule from  220.000 0.00000 to  0.00000
0.00000
\setlinear \setplotsymbol ({\fiverm.}) \plotsymbolspacing=.4pt
\plot
 0.00000 0.00000 18.7723 118.523
 220.000 0.00000 /
\setlinear \setplotsymbol ({\fiverm.}) \plotsymbolspacing=.4pt
\plot
 18.7723 118.523 72.6000 0.00000 /
\setlinear \setplotsymbol ({\fiverm.}) \plotsymbolspacing=.4pt
\plot
 24.9758 52.5775 86.6196 67.4129 /
\circulararc 360 degrees from
 128.798 34.1809
center at  94.6173 34.1809
\circulararc 360 degrees from
 57.8723 26.6588
center at  31.2135 26.6588
\endpicture}}$$
\end{document}