\documentclass[12pt]{article}
\usepackage{amsfonts}

\setlength{\textheight}{9.25in}
\setlength{\textwidth}{6.5in}
\setlength{\topmargin}{0.0in}
\setlength{\headheight}{0.0in}
\setlength{\headsep}{0.0in}
\setlength{\leftmargin}{0.0in}
\setlength{\oddsidemargin}{0.0in}
\setlength{\parindent}{1pc}

\def\binom#1#2{{#1 \choose #2}}
\def\mymod#1{\,(\mbox{mod}\,\, #1)}
\def\pmatrix#1{\left( \begin{array}{ccc} #1 \end{array} \right)}
%Structural macros
\def\head#1{{\bf \noindent \newline #1:  } \nopagebreak}
\def\map#1#2#3{#1\!: #2\to #3}
\def\benum{\begin{enumerate}}
\def\eenum{\end{enumerate}}
\def\soln{\head{Solution}}
\def\first{\head{First Solution}}
\def\second{{\bf \noindent \newline Second Solution:  } \nopagebreak}
\def\third{{\bf \noindent \newline Third Solution:  } \nopagebreak}
\def\note{{\bf \noindent \newline Note:  } \nopagebreak}
\def\be{\begin{enumerate}}
\def\ee{\end{enumerate}}
\def\beq{\begin{equation}}
\def\eeq{\end{equation}}
\def\beqa{\begin{eqnarray*}}
\def\eeqa{\end{eqnarray*}}
\def\bea{\begin{array}}
\def\eea{\end{array}}
\def\ii{\item}
\def\ds{\dsp}
\def\seqa{a_{1}, \dots, a_{n}}
\def\seqb{b_{1}, \dots, b_{n}}
\def\seqx{x_{1}, \dots, x_{n}}
\def\lf{\lfloor}
\def\rf{\rfloor}
\def\Lp{\left(}
\def\Rp{\right)}
\def\L[{\left[}
\def\R]{\right]}
\def\alf{\alpha}
\def\lcm{\mbox{lcm}}
\def\reseteq{\setcounter{equation}{0}}
\def\Avec{{(A_{1}, \ldots, A_{n})}}
\def\Acup{{A_{1} \cup A_{2} \cup \cdots \cup A_{n}}}
\def\crn{{\sqrt[3]{n}}}
\def\ang{\angle}
\def\ep{\epsilon}
\def\dsp{\displaystyle}
\def\oar{\overrightarrow}
\newtheorem{lemma}{Lemma}

%\newcounter{felist}
%\def\bfe{\begin{list}{\alph{felist}}{\usecounter{fe}
\def\bi{\begin{itemize}}
\def\ei{\end{itemize}}
\def\dd{\dots}
\def\half{\frac{1}{2}}
\def\sang{\sin \angle}


\def\binom#1#2{{#1 \choose #2}}
\def\bZ{\mathbb{Z}}
\def\st{^{\scriptstyle\mbox{st}}}

\def\head#1{\begin{center} {\bf #1} \end{center}}

\pagestyle{empty}
\begin{document}
%\input epsf

\begin{center}
${\bf 31\st}$ {\bf United States of America Mathematical Olympiad}
\\[.1in]
{\bf Cambridge, Massachusetts}\\[.05in]
{\bf Part I \hspace{.25in}  1 p.m. - 5:30 p.m.}\\[.05in]
{\bf May 3, 2002}
\end{center}

\bigskip 

\be
\ii %CAR1
Let $S$ be a set with 2002 elements, and let $N$ be an integer with
$0 \leq N \leq 2^{2002}$. Prove that it is possible to color every
subset of $S$ either black or white so that the following conditions
hold:
\begin{enumerate}
\item[(a)] the union of any two white subsets is white;
\item[(b)] the union of any two black subsets is black;
\item[(c)] there are exactly $N$ white subsets.
\end{enumerate}

\ii %AND1
Let $ABC$ be a triangle such that
\[
\left( \cot \frac{A}{2} \right)^2 +
\left( 2\cot \frac{B}{2} \right)^2 +
\left( 3\cot \frac{C}{2} \right)^2 =
 \left( \frac{6s}{7r} \right)^2,
\]
where $s$ and $r$ denote its semiperimeter and its inradius, respectively.
Prove that triangle $ABC$ is similar to a triangle $T$ whose side lengths
are all positive integers with no common divisors and
determine these integers.

\ii %STG1
Prove that any monic polynomial (a polynomial with leading coefficient 1)
of degree $n$ with real coefficients is the average of two monic polynomials
of degree $n$ with $n$ real roots.

\end{enumerate}

\vspace{80 mm}

{\small
\begin{center}
Copyright \copyright \ \ Committee on the American  Mathematics  Competitions,\\
Mathematical Association of America
\end{center}
}
\newpage

\begin{center}
${\bf 31\st}$ {\bf United States of America Mathematical Olympiad}
\\[.1in]
{\bf Cambridge, Massachusetts}\\[.05in]
{\bf Part II \hspace{.25in}  1 p.m. - 5:30 p.m.}\\[.05in]
{\bf May 4, 2002}
\end{center}

\bigskip

\be
\ii[4.] %FEN3
Let $\mathbb{R}$ be the set of real numbers. Determine all functions
$f: \mathbb{R} \to \mathbb{R}$ such that
\[
f(x^2 - y^2) = x f(x) - y f(y)
\]
for all pairs of real numbers $x$ and $y$.

\ii[5.] %CAR2
Let $a,b$ be integers greater than 2. Prove that there exists a positive
integer $k$ and a finite sequence $n_1, n_2, \dots, n_k$ of positive
integers such that $n_1 = a$, $n_k = b$, and $n_i n_{i+1}$
is divisible by $n_i + n_{i+1}$ for each $i$ ($1 \leq i < k$).

\ii[6.] %KED3
I have an $n \times n$ sheet of stamps, from which I've been asked to tear 
out blocks of three adjacent stamps in a single row or column. (I can only
tear along the perforations separating adjacent stamps, and each block
must come out of the sheet in one piece.) Let $b(n)$ be the smallest number
of blocks I can tear out and make it impossible to tear out any more
blocks. Prove that there are real constants $c$ and $d$ such that
\[
\frac{1}{7} n^2 - cn \leq b(n) \leq \frac{1}{5} n^2 + dn
\]
for all $n > 0$.

\end{enumerate}

\vspace{80mm}
{\small
\begin{center}
Copyright \copyright \ \  Committee on the American  Mathematics  Competitions,\\
Mathematical Association of America
\end{center}
}
\end{document}



