%% This document created by Scientific Notebook (R) Version 3.0


\documentclass[12pt,thmsa]{article}
%%%%%%%%%%%%%%%%%%%%%%%%%%%%%%%%%%%%%%%%%%%%%%%%%%%%%%%%%%%%%%%%%%%%%%%%%%%%%%%%%%%%%%%%%%%%%%%%%%%%%%%%%%%%%%%%%%%%%%%%%%%%
\usepackage{sw20jart}

%TCIDATA{TCIstyle=article/art4.lat,jart,sw20jart}

%TCIDATA{<META NAME="GraphicsSave" CONTENT="32">}
%TCIDATA{Created=Mon Aug 19 14:52:24 1996}
%TCIDATA{LastRevised=Mon Feb 10 19:12:12 1997}
%TCIDATA{CSTFile=Lab Report.cst}
%TCIDATA{PageSetup=72,72,72,72,0}
%TCIDATA{AllPages=
%F=36,\PARA{038<p type="texpara" tag="Body Text" >\hfill \thepage}
%}


\input{tcilatex}
\begin{document}


\section{Eighteenth International Olympiad, 1976}

\subsection{1976/1.}

In a plane convex quadrilateral of area $32,$ the sum of the lengths of two
opposite sides and one diagonal is $16.$ Determine all possible lengths of
the other diagonal.

\subsection{1976/2. }

Let $P_{1}(x)=x^{2}-2$ and $P_{j}(x)=P_{1}(P_{j-1}(x))$ for $j=2,3,\cdots $.
Show that, for any positive integer $n,$ the roots of the equation $%
P_{n}(x)=x$ are real and distinct.

\subsection{1976/3. }

A rectangular box can be filled completely with unit cubes. If one places as
many cubes as possible, each with volume $2,$ in the box, so that their
edges are parallel to the edges of the box, one can fill exactly $40\%$ of
the box. Determine the possible dimensions of all such boxes.

\subsection{1976/4. }

Determine, with proof, the largest number which is the product of positive
integers whose sum is $1976.$

\subsection{1976/5. }

Consider the system of $p$ equations in $q=2p$ unknowns $x_{1},x_{2},\cdots
,x_{q}:$%
\begin{eqnarray*}
a_{11}x_{1}+a_{12}x_{2}+\cdots +a_{1q}x_{q} &=&0 \\
a_{21}x_{1}+a_{22}x_{2}+\cdots +a_{2q}x_{q} &=&0 \\
&&\cdots  \\
a_{p1}x_{1}+a_{p2}x_{2}+\cdots +a_{pq}x_{q} &=&0
\end{eqnarray*}

with every coefficient $a_{ij}$ member of the set $\{-1,0,1\}.$ Prove that
the system has a solution $(x_{1},x_{2},\cdots ,x_{q})$ such that

(a) all $x_{j}\;(j=1,2,...,q)$ are integers, 

(b) there is at least one value of $j$ for which $x_{j}\neq 0,$

(c) $\left| x_{j}\right| \leq q(j=1,2,...,q).$

\subsection{1976/6. }

A sequence $\{u_{n}\}$ is defined by
\[
u_{0}=2,u_{1}=5/2,u_{n+1}=u_{n}(u_{n-1}^{2}-2)-u_{1}\text{for }n=1,2,\cdots 
\]
Prove that for positive integers $n,$%
\[
\left[ u_{n}\right] =2^{\left[ 2^{n}-(-1)^{n}\right] /3}
\]

where $[x]$ denotes the greatest integer $\leq x.$

\end{document}
