\documentclass[12pt]{article}
\usepackage{amsfonts}

\pagestyle{empty}
\setlength{\oddsidemargin}{.15in}
\setlength{\evensidemargin}{.15in}
\setlength{\textwidth}{6in}
\setlength{\textheight}{9.25in}
\setlength{\topmargin}{.2in}
\setlength{\headheight}{0in}
\setlength{\headsep}{0in}
\setlength{\parskip}{20pt}
\setlength{\labelsep}{10pt}
\setlength{\parindent}{0pt}
\setlength{\medskipamount}{3ex}
\setlength{\smallskipamount}{1ex}

\begin{document}
\begin{center}
${\bf 28}^{\mbox{\bf th}}$ {\bf International
Mathematical Olympiad} \\[.1in]
{\bf Havana, Cuba} \\ [.05in]
{\bf Day I}\\[.05in]
{\bf July 10, 1987}
\end{center}

\vspace*{.3in}

\begin{enumerate}
\item
Let $p_n(k)$ be the number of permutations of the set $\{1, \ldots, n\}$, $n
\geq 1$, which have exactly $k$ fixed points.  Prove that
$$\sum_{k=0}^n k \cdot p_n(k) = n!.$$

(Remark: A permutation $f$ of a set $S$ is a one-to-one mapping of $S$ onto
itself.  An element $i$ in $S$ is called a fixed point of the permutation $f$
if $f(i) = i$.)

\item
In an acute-angled triangle $ABC$ the interior bisector of the angle $A$
intersects $BC$ at $L$ and intersects the circumcircle of $ABC$ again at $N$.
From point $L$ perpendiculars are drawn to $AB$ and $AC$, the feet of these
perpendiculars being $K$ and $M$ respectively.  Prove that the quadrilateral
$AKNM$ and the triangle $ABC$ have equal areas.

\item
Let $x_1$, $x_2$, \ldots, $x_n$ be real numbers satisfying $x_1^2 + x_2^2 +
\cdots + x_n^2 = 1$.  Prove that for every integer $k \geq 2$ there are
integers $a_1$, $a_2$, \ldots, $a_n$, not all 0, such that $|a_i| \leq k - 1$
for all $i$ and
$$|a_1 x_1 + a_1 x_2 + \cdots + a_n x_n| \leq \frac{(k - 1) \sqrt{n}}{k^n -
1}.$$
\end{enumerate}

\pagebreak %% DAY 2
\begin{center}
${\bf 28}^{\mbox{\bf th}}$ {\bf International
Mathematical Olympiad} \\[.1in]
{\bf Havana, Cuba} \\ [.05in]
{\bf Day II}\\[.05in]
{\bf July 11, 1987}
\end{center}

\vspace*{.3in}

\begin{enumerate}
\setcounter{enumi}{3}
\item
Prove that there is no function $f$ from the set of non-negative integers into
itself such that $f(f(n)) = n + 1987$ for every $n$.

\item
Let $n$ be an integer greater than or equal to 3.  Prove that there is a set of
$n$ points in the plane such that the distance between any two points is
irrational and each set of three points determines a non-degenerate triangle
with rational area.

\item
Let $n$ be an integer greater than or equal to 2.  Prove that if $k^2 + k + n$
is prime for all integers $k$ such that $0 \leq k \leq \sqrt{n/3}$, then $k^2 +
k + n$ is prime for all integers $k$ such that $0 \leq k \leq n - 2$.
\end{enumerate}
\end{document}
