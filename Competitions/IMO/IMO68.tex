%% This document created by Scientific Notebook (R) Version 3.0


\documentclass[12pt,thmsa]{article}
%%%%%%%%%%%%%%%%%%%%%%%%%%%%%%%%%%%%%%%%%%%%%%%%%%%%%%%%%%%%%%%%%%%%%%%%%%%%%%%%%%%%%%%%%%%%%%%%%%%%%%%%%%%%%%%%%%%%%%%%%%%%
\usepackage{sw20jart}

%TCIDATA{TCIstyle=article/art4.lat,jart,sw20jart}

%TCIDATA{<META NAME="GraphicsSave" CONTENT="32">}
%TCIDATA{Created=Mon Aug 19 14:52:24 1996}
%TCIDATA{LastRevised=Mon Feb 10 16:23:47 1997}
%TCIDATA{CSTFile=Lab Report.cst}
%TCIDATA{PageSetup=72,72,72,72,0}
%TCIDATA{AllPages=
%F=36,\PARA{038<p type="texpara" tag="Body Text" >\hfill \thepage}
%}


\input{tcilatex}
\begin{document}


\section{Tenth International Olympiad, 1968}

\subsection{1968/1.}

Prove that there is one and only one triangle whose side lengths are
consecutive integers, and one of whose angles is twice as large as another.

\subsection{1968/2.}

Find all natural numbers $x$ such that the product of their digits (in
decimal notation) is equal to $x^{2}-10x-22.$

\subsection{1968/3.}

Consider the system of equations

\begin{eqnarray*}
ax_{1}^{2}+bx_{1}+c &=&x_{2} \\
ax_{2}^{2}+bx_{2}+c &=&x_{3} \\
&&\cdots  \\
ax_{n-1}^{2}+bx_{n-1}+c &=&x_{n} \\
ax_{n}^{2}+bx_{n}+c &=&x_{1},
\end{eqnarray*}

with unknowns $x_{1},x_{2},\cdots ,x_{n}$, where $a,b,c$ are real and $a\neq
0.$ Let $\Delta =(b-1)^{2}-4ac.$ Prove that for this system

(a) if $\Delta <0,$ there is no solution,

(b) if $\Delta =0,$ there is exactly one solution,

(c) if $\Delta >0,$ there is more than one solution.

\subsection{1968/4.}

Prove that in every tetrahedron there is a vertex such that the three edges
meeting there have lengths which are the sides of a triangle.

\subsection{1968/5.}

Let $f$ be a real-valued function defined for all real numbers $x$ such
that, for some positive constant $a,$ the equation
\[
f(x+a)=\frac{1}{2}+\sqrt{f(x)-[f(x)]^{2}}
\]

holds for all $x.$

(a) Prove that the function $f$ is periodic (i.e., there exists a positive
number $b$ such that $f(x+b)=f(x)$ for all $x$).

(b) For $a=1,$ give an example of a non-constant function with the required
properties.

\subsection{1968/6.}

For every natural number $n,$ evaluate the sum
\[
\sum_{k=0}^{\infty }\left[ \frac{n+2^{k}}{2^{k+1}}\right] =\left[ \frac{n+1}{%
2}\right] +\left[ \frac{n+2}{4}\right] +\cdots +\left[ \frac{n+2^{k}}{2^{k+1}%
}\right] +\cdots 
\]

(The symbol $[x]$ denotes the greatest integer not exceeding $x.$)

\end{document}
