%% This document created by Scientific Notebook (R) Version 3.0


\documentclass[12pt,thmsa]{article}
%%%%%%%%%%%%%%%%%%%%%%%%%%%%%%%%%%%%%%%%%%%%%%%%%%%%%%%%%%%%%%%%%%%%%%%%%%%%%%%%%%%%%%%%%%%%%%%%%%%%%%%%%%%%%%%%%%%%%%%%%%%%
\usepackage{sw20jart}

%TCIDATA{TCIstyle=article/art4.lat,jart,sw20jart}

%TCIDATA{<META NAME="GraphicsSave" CONTENT="32">}
%TCIDATA{Created=Mon Aug 19 14:52:24 1996}
%TCIDATA{LastRevised=Mon Feb 10 16:49:41 1997}
%TCIDATA{CSTFile=Lab Report.cst}
%TCIDATA{PageSetup=72,72,72,72,0}
%TCIDATA{AllPages=
%F=36,\PARA{038<p type="texpara" tag="Body Text" >\hfill \thepage}
%}


\input{tcilatex}
\begin{document}


\section{Eleventh International Olympiad, 1969}

\subsection{1969/1.}

Prove that there are infinitely many natural numbers $a$ with the following
property: the number $z=n^{4}+a$ is not prime for any natura1 number $n.$

\subsection{1969/2.}

Let $a_{1},a_{2},\cdots ,a_{n}$ be real constants, $x$ a real variable, and

\begin{eqnarray*}
f(x) &=&\cos (a_{1}+x)+\frac{1}{2}\cos (a_{2}+x)+\frac{1}{4}\cos (a_{3}+x) \\
&&+\cdots +\frac{1}{2^{n-1}}\cos (a_{n}+x).
\end{eqnarray*}

Given that $f(x_{1})=f(x_{2})=0,$ prove that $x_{2}-x_{1}=m\pi $ for some
integer $m.$

\subsection{1969/3.}

For each value of $k=1,2,3,4,5,$ find necessary and sufficient conditions on
the number $a>0$ so that there exists a tetrahedron with $k$ edges of length 
$a,$ and the remaining $6-k$ edges of length $1$.

\subsection{1969/4.}

A semicircular arc $\gamma $ is drawn on $AB$ as diameter. $C$ is a point on 
$\gamma $ other than $A$ and $B,$ and $D$ is the foot of the perpendicular
from $C$ to $AB.$ We consider three circles, $\gamma _{1},\gamma _{2},\gamma
_{3}$, all tangent to the line $AB.$ Of these, $\gamma _{1}$ is inscribed in 
$\Delta ABC,$ while $\gamma _{2}$ and $\gamma _{3}$ are both tangent to $CD$
and to $\gamma $, one on each side of $CD.$ Prove that $\gamma _{1},\gamma
_{2}$ and $\gamma _{3}$ have a second tangent in common.

\subsection{1969/5.}

\vspace{1pt}Given $n>4$ points in the plane such that no three are
collinear. Prove that there are at least $\binom{n-3}{2}$ convex
quadrilaterals whose vertices are four of the given points.

\subsection{1969/6.}

Prove that for all real numbers $x_{1},x_{2},y_{1},y_{2},z_{1},z_{2},$ with $%
x_{1}>0,$ $x_{2}>0,x_{1}y_{1}-z_{1}^{2}>0,x_{2}y_{2}-z_{2}^{2}>0,$ the
inequality
\[
\frac{8}{\left( x_{1}+x_{2}\right) \left( y_{1}+y_{2}\right) -\left(
z_{1}+z_{2}\right) ^{2}}\leq \frac{1}{x_{1}y_{1}-z_{1}^{2}}+\frac{1}{%
x_{2}y_{2}-z_{2}^{2}}
\]

is satisfied. Give necessary and sufficient conditions for equality.

\end{document}
