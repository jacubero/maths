%% This document created by Scientific Notebook (R) Version 3.0


\documentclass[12pt,thmsa]{article}
%%%%%%%%%%%%%%%%%%%%%%%%%%%%%%%%%%%%%%%%%%%%%%%%%%%%%%%%%%%%%%%%%%%%%%%%%%%%%%%%%%%%%%%%%%%%%%%%%%%%%%%%%%%%%%%%%%%%%%%%%%%%
\usepackage{sw20jart}

%TCIDATA{TCIstyle=article/art4.lat,jart,sw20jart}

%TCIDATA{<META NAME="GraphicsSave" CONTENT="32">}
%TCIDATA{Created=Mon Aug 19 14:52:24 1996}
%TCIDATA{LastRevised=Mon Feb 10 11:15:57 1997}
%TCIDATA{CSTFile=Lab Report.cst}
%TCIDATA{PageSetup=72,72,72,72,0}
%TCIDATA{AllPages=
%F=36,\PARA{038<p type="texpara" tag="Body Text" >\hfill \thepage}
%}


\input{tcilatex}
\begin{document}


\section{Twenty-fourth International Olympiad, 1983}

1983/1. Find all functions $f$ defined on the set of positive real numbers
which take positive real values and satisfy the conditions:

(i) $f(xf(y))=yf(x)$ for all positive $x,y;$

(ii) $f(x)\rightarrow 0$ as $x\rightarrow \infty .$

1983/2. Let $A$ be one of the two distinct points of intersection of two
unequal coplanar circles $C_{1}$ and $C_{2}$ with centers $O_{1}$ and $O_{2}$%
, respectively. One of the common tangents to the circles touches $C_{1}$ at 
$P_{1}$ and $C_{2}$ at $P_{2}$, while the other touches $C_{1}$ at $Q_{1}$
and $C_{2}$ at $Q_{2}$. Let $M_{1}$ be the midpoint of $P_{1}Q_{1},$and $%
M_{2}$ be the midpoint of $P_{2}Q_{2}$. Prove that $\angle
O_{1}AO_{2}=\angle M_{1}AM_{2}$.

1983/3. Let $a,b$ and $c$ be positive integers, no two of which have a
common divisor greater than $1.$ Show that $2abc-ab-bc-ca$ is the largest
integer which cannot be expressed in the form $xbc+yca+zab,$where $x,y$ and $%
z$ are non-negative integers.

1983/4. Let $ABC$ be an equilateral triangle and $\mathcal{E}$ the set of
all points contained in the three segments $AB,BC$ and $CA$ (including $A,B$
and $C$). Determine whether, for every partition of $\mathcal{E}$ into two
disjoint subsets, at least one of the two subsets contains the vertices of a
right-angled triangle. Justify your answer.

1983/5. Is it possible to choose $1983$ distinct positive integers, all less
than or equal to $10^{5}$, no three of which are consecutive terms of an
arithmetic progression? Justify your answer.

1983/6. Let $a,b$ and $c$ be the lengths of the sides of a triangle. Prove
that 
\[
a^{2}b(a-b)+b^{2}c(b-c)+c^{2}a(c-a)\geq 0.
\]
Determine when equality occurs.

\end{document}
