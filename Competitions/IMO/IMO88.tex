\documentclass[12pt]{article}
\usepackage{amsfonts}

\pagestyle{empty}
\setlength{\oddsidemargin}{.15in}
\setlength{\evensidemargin}{.15in}
\setlength{\textwidth}{6in}
\setlength{\textheight}{9.25in}
\setlength{\topmargin}{.2in}
\setlength{\headheight}{0in}
\setlength{\headsep}{0in}
\setlength{\parskip}{20pt}
\setlength{\labelsep}{10pt}
\setlength{\parindent}{0pt}
\setlength{\medskipamount}{3ex}
\setlength{\smallskipamount}{1ex}

\begin{document}
\begin{center}
${\bf 29}^{\mbox{\bf th}}$ {\bf International
Mathematical Olympiad} \\[.1in]
{\bf Canberra, Australia} \\ [.05in]
{\bf Day I}\\[.05in]
\end{center}

\vspace*{.3in}

\begin{enumerate}
\item
Consider two coplanar circles of radii $R$ and $r$ ($R > r$) with the same
center.  Let $P$ be a fixed point on the smaller circle and $B$ a variable
point on the larger circle.  The line $BP$ meets the larger circle again at
$C$.  The perpendicular $l$ to $BP$ at $P$ meets the smaller circle again at
$A$.  (If $l$ is tangent to the circle at $P$ then $A = P$.)
  \begin{enumerate}
  \item[(i)] Find the set of values of $BC^2 + CA^2 + AB^2$.
  \item[(ii)] Find the locus of the midpoint of $BC$.
  \end{enumerate}

\item
Let $n$ be a positive integer and let $A_1$, $A_2$, \ldots, $A_{2n+1}$ be
subsets of a set $B$.  Suppose that
  \begin{enumerate}
  \item[(a)] Each $A_i$ has exactly $2n$ elements,
  \item[(b)] Each $A_i \cap A_j$ ($1 \leq i < j \leq 2n + 1$) contains exactly
  one element, and
  \item[(c)] Every element of $B$ belongs to at least two of the $A_i$.
  \end{enumerate}
For which values of $n$ can one assign to every element of $B$ one of the
numbers 0 and 1 in such a way that $A_i$ has 0 assigned to exactly $n$ of its
elements?


\item
A function $f$ is defined on the positive integers by
\begin{eqnarray*}
f(1) &=& 1, \ \ f(3) = 3, \\
f(2n) &=& f(n), \\
f(4n + 1) &=& 2f(2n + 1) - f(n), \\
f(4n + 3) &=& 3f(2n + 1) - 2f(n),
\end{eqnarray*}
for all positive integers $n$.

Determine the number of positive integers $n$, less than or equal to 1988, for
which $f(n) = n$.
\end{enumerate}

\pagebreak %% DAY 2
\begin{center}
${\bf 29}^{\mbox{\bf th}}$ {\bf International
Mathematical Olympiad} \\[.1in]
{\bf Canberra, Australia} \\ [.05in]
{\bf Day II}\\[.05in]
\end{center}

\vspace*{.3in}

\begin{enumerate}
\setcounter{enumi}{3}
\item
Show that set of real numbers $x$ which satisfy the inequality
$$\sum_{k=1}^{70} \frac{k}{x - k} \geq \frac{5}{4}$$
is a union of disjoint intervals, the sum of whose lengths is 1988.

\item
$ABC$ is a triangle right-angled at $A$, and $D$ is the foot of the altitude
from $A$.  The straight line joining the incenters of the triangles $ABD$,
$ACD$ intersects the sides $AB$, $AC$ at the points $K$, $L$ respectively.
$S$ and $T$ denote the areas of the triangles $ABC$ and $AKL$ respectively.
Show that $S \geq 2T$.

\item
Let $a$ and $b$ be positive integers such that $ab + 1$ divides $a^2 + b^2$.
Show that
$$\frac{a^2 + b^2}{ab + 1}$$
is the square of an integer.
\end{enumerate}
\end{document}
