\documentclass[12pt]{article}
\usepackage{amsfonts}

\pagestyle{empty}
\setlength{\oddsidemargin}{.15in}
\setlength{\evensidemargin}{.15in}
\setlength{\textwidth}{6in}
\setlength{\textheight}{9.25in}
\setlength{\topmargin}{.2in}
\setlength{\headheight}{0in}
\setlength{\headsep}{0in}
\setlength{\parskip}{20pt}
\setlength{\labelsep}{10pt}
\setlength{\parindent}{0pt}
\setlength{\medskipamount}{3ex}
\setlength{\smallskipamount}{1ex}
\def\rd{^{\scriptstyle\mbox{rd}}}
\def\ang{\angle}
\def\dg{^\circ}
\def\be{\begin{enumerate}}
\def\ee{\end{enumerate}}
\def\ii{\item}

\begin{document}

%% DAY 1
\begin{center}
${\bf 43\rd}$ {\bf International Mathematical Olympiad} \\[.1in]
{\bf Glasgow, Scotland, United Kingdom} \\ [.05in]
{\bf Day I%\hspace{.25in} 9 a.m. - 1:30 p.m.
}\\[.05in]
{\bf July 24, 2002}
\end{center}

\vspace*{.3in}

\begin{enumerate}
\item %% IMO1
Let $n$ be a positive integer. Let $T$ be the set of points $(x, y)$ in the
plane where $x$ and $y$ are non-negative integers and $x + y < n$.
Each point of $T$ is colored red or blue. If a point $(x, y)$ is red, then
so are all points $(x', y')$ of $T$ with both $x' \le x$ and $y' \le y$.
Define an $X$-set to be a set of $n$ blue points having distinct 
$x$-coordinates, and a $Y$-set to be a set of $n$ blue points having 
distinct $y$-coordinates. Prove that the number of $X$-sets is equal 
to the number of $Y$-sets.

\item %% IMO2
Let $BC$ be a diameter of circle $\omega$ with center $O$. Let $A$ be a point
of circle $\omega$ such that $0\dg < \ang AOB < 120\dg$. Let $D$ be the midpoint
 of arc $AB$ not containing $C$. Line $\ell$ passes through $O$ and is 
parallel to line $AD$. Line $\ell$ intersects line $AC$ at $J$. 
The perpendicular bisector of segment $OA$ intersects circle $\omega$ at $E$
 and $F$. Prove that $J$ is the incenter of triangle $CEF$.

\item %% IMO3
Find all pairs of integers $m, n \ge 3$ such that there exist infinitely
many positive integers $a$ for which
\[
\frac{a^m + a - 1}{a^n + a^2 - 1}
\]
is an integer.
\end{enumerate}

\pagebreak %% DAY 2
\begin{center}
${\bf 43\rd}$ {\bf International Mathematical Olympiad} \\[.1in]
{\bf Glasgow, Scotland, United Kingdom} \\ [.05in]
{\bf Day II%\hspace{.25in} 9 a.m. - 1:30 p.m.
}\\[.05in]
{\bf July 25, 2002}
\end{center}

\vspace*{.3in}

\begin{enumerate}
\setcounter{enumi}{3}
\item %% IMO4
Let $n$ be an integer greater than 1. The positive divisors of $n$ are
$d_1, d_2, \dots, d_k$ where $1 = d_1 < d_2 < \cdots <d_k = n$.
Define $D = d_1d_2 + d_2d_3 + \cdots + d_{k-1}d_k$.
\begin{enumerate}
\item[(a)] Prove that $D<n^2$.
\item[(b)] Determine all $n$ for which $D$ is a divisor of $n^2$.
\end{enumerate}

\item %% IMO5
Find all functions $f$ from the set $\mathbb{R}$ of real numbers to
itself such that
\[
(f(x) + f(z))(f(y) + f(t)) = f(xy-zt) + f(xt+yz)
\]
for all $x,y,z,t$ in $\mathbb{R}$.

\item %% IMO6
Let $\Gamma_1, \Gamma_2, \dots, \Gamma_n$ be circles of radius 1 in the
plane, where $n \geq 3$. Denote their centers by $O_1, O_2, \dots, O_n$
respectively. Suppose that no line meets more than two of the circles.
Prove that
\[
\sum_{1 \leq i < j \leq n} \frac{1}{O_iO_j} \leq \frac{(n-1)\pi}{4}.
\]

\end{enumerate}

\end{document}

 
                   

