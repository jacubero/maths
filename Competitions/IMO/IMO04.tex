\documentclass[12pt]{article}
\usepackage{amsfonts}

\pagestyle{empty}
\setlength{\oddsidemargin}{.15in}
\setlength{\evensidemargin}{.15in}
\setlength{\textwidth}{6in}
\setlength{\textheight}{9.25in}
\setlength{\topmargin}{.2in}
\setlength{\headheight}{0in}
\setlength{\headsep}{0in}
\setlength{\parskip}{20pt}
\setlength{\labelsep}{10pt}
\setlength{\parindent}{0pt}
\setlength{\medskipamount}{3ex}
\setlength{\smallskipamount}{1ex}
\def\th{^{\scriptstyle\mbox{th}}}
\def\ang{\angle}
\def\dg{^\circ}
\def\be{\begin{enumerate}}
\def\ee{\end{enumerate}}
\def\ii{\item}

\begin{document}

\begin{center}
${\bf 45\th}$ {\bf International Mathematical Olympiad} \\[.1in]
{\bf Athens, Greece} \\ [.05in]
{\bf Day I}\\[.05in]
{\bf July 12, 2004}
\end{center}

\vspace*{.3in}

\begin{enumerate}
\item
Let $ABC$ be an acute-angled triangle with $AB\ne AC$. The circle with 
diameter $BC$ intersects the sides $AB$ and $AC$ at $M$ and $N$, 
respectively. Denote by $O$ the midpoint of the side $BC$. The bisectors 
of the angles $BAC$ and $MON$ intersect at $R$. Prove that the 
circumcircles of the triangles $BMR$ and $CNR$ have a common point lying 
on the side $BC$.

\item
Find all polynomials $P(x)$ with real coefficients which satisfy the 
equality
\[P(a-b)+P(b-c)+P(c-a)= 2 P(a+b+c)\]
for all real numbers $a$, $b$, $c$ such that $ab + bc + ca = 0$.

\item
Define a \emph{hook} to be a figure made up of six unit squares as shown 
in the diagram
\begin{center}
\begin{picture}(30,30)
\put(0,30){\line(1,0){30}}
\put(0,20){\line(1,0){30}}
\put(0,10){\line(1,0){10}}
\put(20,10){\line(1,0){10}}
\put(0,0){\line(1,0){10}}
\put(0,0){\line(0,1){30}}
\put(10,0){\line(0,1){30}}
\put(20,10){\line(0,1){20}}
\put(30,10){\line(0,1){20}}
\end{picture}
\end{center}

or any of the figures obtained by applying rotations and reflections to 
this figure.\\
Determine all $m\times n$ rectangles that can be covered with hooks so 
that
\begin{itemize}
\item the rectangle is covered without gaps and without overlaps
\item no part of a hook covers area outside the rectangle.
\end{itemize}
\end{enumerate}

\pagebreak %% DAY 2
\begin{center}
${\bf 45\th}$ {\bf International Mathematical Olympiad} \\[.1in]
{\bf Athens, Greece} \\ [.05in]
{\bf Day II}\\[.05in]
{\bf July 13, 2004}
\end{center}

\vspace*{.3in}

\begin{enumerate}
\setcounter{enumi}{3}
\item
Let $n\ge 3$ be an integer. Let $t_1$, $t_2$, \dots, $t_n$ be positive 
real numbers such that
\[
n^2+1 > (t_1 + t_2 + \cdots + t_n)
\left(\frac{1}{t_1} + \frac{1}{t_2} + \cdots + \frac{1}{t_n}\right).
\]
Show that $t_i$, $t_j$, $t_k$ are side lengths of a triangle for all $i$, 
$j$, $k$ with $1\le i < j <k \le n$.

\item
In a convex quadrilateral $ABCD$ the diagonal $BD$
bisects neither the angle $ABC$ nor the angle $CDA$.
A point $P$ lies inside $ABCD$ and satisfies
\[
\angle PBC = \angle DBA\quad\textrm{and}\quad \angle PDC = \angle BDA.
\]
Prove that $ABCD$ is a cyclic quadrilateral if and only if $AP=CP$.

\item
We call a positive integer \emph{alternating} if every two consecutive 
digits in its decimal representation are of different parity.\\
Find all positive integers $n$ such that $n$ has a multiple which is 
alternating. 
\end{enumerate}

\end{document}
