%% This document created by Scientific Notebook (R) Version 3.0


\documentclass[12pt,thmsa]{article}
%%%%%%%%%%%%%%%%%%%%%%%%%%%%%%%%%%%%%%%%%%%%%%%%%%%%%%%%%%%%%%%%%%%%%%%%%%%%%%%%%%%%%%%%%%%%%%%%%%%%%%%%%%%%%%%%%%%%%%%%%%%%
\usepackage{sw20jart}

%TCIDATA{TCIstyle=article/art4.lat,jart,sw20jart}

%TCIDATA{<META NAME="GraphicsSave" CONTENT="32">}
%TCIDATA{Created=Mon Aug 19 14:52:24 1996}
%TCIDATA{LastRevised=Mon Feb 10 12:20:19 1997}
%TCIDATA{CSTFile=Lab Report.cst}
%TCIDATA{PageSetup=72,72,72,72,0}
%TCIDATA{AllPages=
%F=36,\PARA{038<p type="texpara" tag="Body Text" >\hfill \thepage}
%}


\input{tcilatex}
\begin{document}


\section{Third International Olympiad, 1961}

\subsection{1961/1.}

Solve the system of equations:
\begin{eqnarray*}
x+y+z &=&a \\
x^{2}+y^{2}+z^{2} &=&b^{2} \\
xy &=&z^{2}
\end{eqnarray*}

where $a$ and $b$ are constants. Give the conditions that $a$ and $b$ must
satisfy so that $x,y,z$ (the solutions of the system) are distinct positive
numbers.

\subsection{1961/2. }

Let $a,b,c$ be the sides of a triangle, and $T$ its area. Prove: $%
a^{2}+b^{2}+c^{2}\geq 4\sqrt{3}T.$ In what case does equality hold?

\subsection{1961/3.}

Solve the equation $\cos ^{n}x-\sin ^{n}x=1,$ where $n$ is a natural number.

\subsection{1961/4.}

Consider triangle $P_{1}P_{2}P_{3}$ and a point $P$ within the triangle.
Lines $P_{1}P,P_{2}P,P_{3}P$ intersect the opposite sides in points $%
Q_{1},Q_{2},Q_{3}$ respectively. Prove that, of the numbers
\[
\frac{P_{1}P}{PQ_{1}},\frac{P_{2}P}{PQ_{2}},\frac{P_{3}P}{PQ_{3}}
\]

at least one is $\leq 2$ and at least one is $\geq 2$.

\subsection{1961/5.}

Construct triangle $ABC$ if $AC=b,AB=c$ and $\angle AMB=\omega $, where $M$
is the midpoint of segment $BC$ and $\omega <90^{\circ }$. Prove that a

solution exists if and only if
\[
b\tan \frac{\omega }{2}\leq c<b.
\]

In what case does the equality hold?

\subsection{1961/6.}

Consider a plane $\varepsilon $ and three non-collinear points $A,B,C$ on
the same side of $\varepsilon $; suppose the plane determined by these three
points is not parallel to $\varepsilon $. In plane a take three arbitrary
points $A^{\prime },B^{\prime },C^{\prime }.$ Let $L,M,N$ be the midpoints
of segments $AA^{\prime },BB^{\prime },CC^{\prime };$ let $G$ be the
centroid of triangle $LMN.$ (We will not consider positions of the points $%
A^{\prime },B^{\prime },C^{\prime }$ such that the points $L,M,N$ do not
form a triangle.) What is the locus of point $G$ as $A^{\prime },B^{\prime
},C^{\prime }$ range independently over the plane $\varepsilon $?

\end{document}
