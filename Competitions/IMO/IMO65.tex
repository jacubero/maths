%% This document created by Scientific Notebook (R) Version 3.0


\documentclass[12pt,thmsa]{article}
%%%%%%%%%%%%%%%%%%%%%%%%%%%%%%%%%%%%%%%%%%%%%%%%%%%%%%%%%%%%%%%%%%%%%%%%%%%%%%%%%%%%%%%%%%%%%%%%%%%%%%%%%%%%%%%%%%%%%%%%%%%%
\usepackage{sw20jart}

%TCIDATA{TCIstyle=article/art4.lat,jart,sw20jart}

%TCIDATA{<META NAME="GraphicsSave" CONTENT="32">}
%TCIDATA{Created=Mon Aug 19 14:52:24 1996}
%TCIDATA{LastRevised=Mon Feb 10 15:33:01 1997}
%TCIDATA{CSTFile=Lab Report.cst}
%TCIDATA{PageSetup=72,72,72,72,0}
%TCIDATA{AllPages=
%F=36,\PARA{038<p type="texpara" tag="Body Text" >\hfill \thepage}
%}


\input{tcilatex}
\begin{document}


\section{Seventh Internatioaal Olympiad, 1965}

\subsection{1965/1.}

Determine all values $x$ in the interval $0\leq x\leq 2\pi $ which satisfy
the inequality
\[
2\cos x\leq \left| \sqrt{1+\sin 2x}-\sqrt{1-\sin 2x}\right| \leq \sqrt{2}.
\]

\subsection{1965/2.}

Consider the system of equations
\begin{eqnarray*}
a_{11}x_{1}+a_{12}x_{2}+a_{13}x_{3} &=&0 \\
a_{21}x_{1}+a_{22}x_{2}+a_{23}x_{3} &=&0 \\
a_{31}x_{1}+a_{32}x_{2}+a_{33}x_{3} &=&0
\end{eqnarray*}

with unknowns $x_{1},x_{2},x_{3}$. The coefficients satisfy the conditions:

(a) $a_{11},a_{22},a_{33}$ are positive numbers;

(b) the remaining coefficients are negative numbers;

(c) in each equation, the sum of the coefficients is positive.

Prove that the given system has only the solution $x_{1}=x_{2}=x_{3}=0$.

\subsection{1965/3.}

Given the tetrahedron $ABCD$ whose edges $AB$ and $CD$ have lengths $a$ and $%
b$ respectively. The distance between the skew lines $AB$ and $CD$ is $d,$
and the angle between them is $\omega $. Tetrahedron $ABCD$ is divided into
two solids by plane $\varepsilon $, parallel to lines $AB$ and $CD.$ The
ratio of the distances of $\varepsilon $ from $AB$ and $CD$ is equal to $k.$
Compute the ratio of the volumes of the two solids obtained.

\subsection{1965/4.}

Find all sets of four real numbers  $x_{1},x_{2},x_{3},x_{4}$ such that the
sum of any one and the product of the other three is equal to $2.$

\subsection{1965/5.}

Consider $\Delta OAB$ with acute angle $AOB.$ Through a point $M\neq O$
perpendiculars are drawn to $OA$ and $OB,$ the feet of which are $P$ and $Q$
respectively. The point of intersection of the altitudes of $\Delta OPQ$ is $%
H.$ What is the locus of $H$ if $M$ is permitted to range over (a) the side $%
AB,$ (b) the interior of $\Delta OAB$?

\subsection{1965/6.}

In a plane a set of $n$ points ($n\geq 3$) is given. Each pair of points is
connected by a segment. Let $d$ be the length of the longest of these
segments. We define a diameter of the set to be any connecting segment of
length $d.$ Prove that the number of diameters of the given set is at most $%
n.$

\end{document}
