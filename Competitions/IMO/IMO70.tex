%% This document created by Scientific Notebook (R) Version 3.0


\documentclass[12pt,thmsa]{article}
%%%%%%%%%%%%%%%%%%%%%%%%%%%%%%%%%%%%%%%%%%%%%%%%%%%%%%%%%%%%%%%%%%%%%%%%%%%%%%%%%%%%%%%%%%%%%%%%%%%%%%%%%%%%%%%%%%%%%%%%%%%%
\usepackage{sw20jart}

%TCIDATA{TCIstyle=article/art4.lat,jart,sw20jart}

%TCIDATA{<META NAME="GraphicsSave" CONTENT="32">}
%TCIDATA{Created=Mon Aug 19 14:52:24 1996}
%TCIDATA{LastRevised=Mon Feb 10 17:19:58 1997}
%TCIDATA{Language=American English}
%TCIDATA{CSTFile=Lab Report.cst}
%TCIDATA{PageSetup=72,72,72,72,0}
%TCIDATA{AllPages=
%F=36,\PARA{038<p type="texpara" tag="Body Text" >\hfill \thepage}
%}


\input{tcilatex}
\begin{document}


\section{Twelfth International Olympiad, 1970}

\subsection{1970/1.}

Let $M$ be a point on the side $AB$ of $\Delta ABC.$ Let $r_{1},r_{2}$ and $r
$ be the radii of the inscribed circles of triangles $AMC,BMC$ and $ABC.$
Let $q_{1},q_{2}$ and $q$ be the radii of the escribed circles of the same
triangles that lie in the angle $ACB.$ Prove that
\[
\frac{r_{1}}{q_{1}}\cdot \frac{r_{2}}{q_{2}}=\frac{r}{q}.
\]

\subsection{1970/2.}

Let $a,b$ and $n$ be integers greater than $1,$ and let $a$ and $b$ be the
bases of two number systems. $A_{n-1}$ and $A_{n}$ are numbers in the system
with base $a$, and $B_{n-1}$ and $B_{n}$are numbers in the system with base $%
b;$ these are related as follows:
\begin{eqnarray*}
A_{n} &=&x_{n}x_{n-1}\cdots x_{0},A_{n-1}=x_{n-1}x_{n-2}\cdots x_{0}, \\
B_{n} &=&x_{n}x_{n-1}\cdots x_{0},B_{n-1}=x_{n-1}x_{n-2}\cdots x_{0}, \\
x_{n} &\neq &0,x_{n-1}\neq 0.
\end{eqnarray*}

Prove:
\[
\frac{A_{n-1}}{A_{n}}<\frac{B_{n-1}}{B_{n}}\text{ if and only if }a>b.
\]

\subsection{1970/3.}

The real numbers $a_{0},a_{1},...,a_{n},...$ satisfy the condition:
\[
1=a_{0}\leq a_{1}\leq a_{2}\leq \cdots \leq a_{n}\leq \cdots .
\]

The numbers $b_{1},b_{2},...,b_{n},...$ are defined by
\[
b_{n}=\sum_{k=1}^{n}\left( 1-\frac{a_{k-1}}{a_{k}}\right) \frac{1}{\sqrt{%
a_{k}}}.
\]

(a) Prove that $0\leq b_{n}<2$ for all $n.$

(b) Given $c$ with $0\leq c<2,$ prove that there exist numbers $%
a_{0},a_{1},...$ with the above properties such that $b_{n}>c$ for large
enough $n.$

\subsection{1970/4.}

Find the set of all positive integers $n$ with the property that the set $%
\{n,n+1,n+2,n+3,n+4,n+5\}$ can be partitioned into two sets such that the
product of the numbers in one set equals the product of the numbers in the
other set.

\subsection{1970/5.}

In the tetrahedron $ABCD,$ angle $BDC$ is a right angle. Suppose that the
foot $H$ of the perpendicular from $D$ to the plane $ABC$ is the
intersection of the altitudes of $\Delta ABC.$ Prove that
\[
(AB+BC+CA)^{2}\leq 6(AD^{2}+BD^{2}+CD^{2}).
\]

For what tetrahedra does equality hold?

\subsection{1970/6.}

In a plane there are $100$ points, no three of which are collinear. Consider
all possible triangles having these points as vertices. Prove that no more
than $70\%$ of these triangles are acute-angled.

\end{document}
