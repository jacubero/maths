%% This document created by Scientific Notebook (R) Version 3.0


\documentclass[12pt,thmsa]{article}
%%%%%%%%%%%%%%%%%%%%%%%%%%%%%%%%%%%%%%%%%%%%%%%%%%%%%%%%%%%%%%%%%%%%%%%%%%%%%%%%%%%%%%%%%%%%%%%%%%%%%%%%%%%%%%%%%%%%%%%%%%%%
\usepackage{sw20jart}

%TCIDATA{TCIstyle=article/art4.lat,jart,sw20jart}

%TCIDATA{<META NAME="GraphicsSave" CONTENT="32">}
%TCIDATA{Created=Mon Aug 19 14:52:24 1996}
%TCIDATA{LastRevised=Mon Feb 10 11:22:41 1997}
%TCIDATA{Language=American English}
%TCIDATA{CSTFile=Lab Report.cst}
%TCIDATA{PageSetup=72,72,72,72,0}
%TCIDATA{AllPages=
%F=36,\PARA{038<p type="texpara" tag="Body Text" >\hfill \thepage}
%}


\input{tcilatex}
\begin{document}


\section{Twenty-sixth International Olympiad, 1985}

1985/1. A circle has center on the side $AB$ of the cyclic quadrilateral $%
ABCD.$ The other three sides are tangent to the circle. Prove that $%
AD+BC=AB. $

1985/2. Let $n$ and $k$ be given relatively prime natural numbers, $k<n.$
Each number in the set $M=\{1,2,...,n-1\}$ is colored either blue or white.
It is given that

(i) for each $i\in M,$ both $i$ and $n-i$ have the same color;

(ii) for each $i\in M,i\ne k,$ both $i$ and $\left| i-k\right| $ have the
same color. Prove that all numbers in $M$ must have the same color.

1985/3. For any polynomial $P(x)=a_{0}+a_{1}x+\cdots +a_{k}x^{k}$ with
integer coefficients, the number of coefficients which are odd is denoted by 
$w(P).$ For $i=0,1,...,$ let $Q_{i}(x)=(1+x)^{i}.$ Prove that if $%
i_{1}i_{2},...,i_{n}$ are integers such that $0\le i_{1}<i_{2}<\cdots
<i_{n}, $ then 
\[
w(Q_{i_{1}}+Q_{i_{2}},++Q_{i_{n}})\ge w(Q_{i_{1}}). 
\]

1985/4. Given a set $M$ of $1985$ distinct positive integers, none of which
has a prime divisor greater than $26$. Prove that $M$ contains at least one
subset of four distinct elements whose product is the fourth power of an
integer.

1985/5. A circle with center $O$ passes through the vertices $A$ and $C$ of
triangle $ABC$ and intersects the segments $AB$ and $BC$ again at distinct
points $K$ and $N,$ respectively. The circumscribed circles of the triangles 
$ABC$ and $EBN$ intersect at exactly two distinct points $B$ and $M.$ Prove
that angle $OMB$ is a right angle.

1985/6. For every real number $x_{1},$ construct the sequence $%
x_{1},x_{2},...$ by setting 
\[
x_{n+1}=x_{n}\left( x_{n}+\frac{1}{n}\right) \text{for each }n\geq 1.
\]
Prove that there exists exactly one value of $x_{1}$ for which 
\[
0<x_{n}<x_{n+1}<1
\]
for every $n.$

\end{document}
