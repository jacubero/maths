\documentclass[12pt]{article}
\usepackage{amsfonts}

\pagestyle{empty}
\setlength{\oddsidemargin}{.15in}
\setlength{\evensidemargin}{.15in}
\setlength{\textwidth}{6in}
\setlength{\textheight}{9.25in}
\setlength{\topmargin}{.2in}
\setlength{\headheight}{0in}
\setlength{\headsep}{0in}
\setlength{\parskip}{20pt}
\setlength{\labelsep}{10pt}
\setlength{\parindent}{0pt}
\setlength{\medskipamount}{3ex}
\setlength{\smallskipamount}{1ex}

\begin{document}
\begin{center}
${\bf 27}^{\mbox{\bf th}}$ {\bf International
Mathematical Olympiad} \\[.1in]
{\bf Warsaw, Poland} \\ [.05in]
{\bf Day I}\\[.05in]
{\bf July 9, 1986}
\end{center}

\vspace*{.3in}

\begin{enumerate}
\item
Let $d$ be any positive integer not equal to 2, 5, or 13.  Show that one can
find distinct $a$, $b$ in the set $\{2,5,13,d\}$ such that $ab - 1$ is not a
perfect square.

\item
A triangle $A_1 A_2 A_3$ and a point $P_0$ are given in the plane.  We define
$A_s = A_{s-3}$ for all $s \geq 4$.  We construct a set of points $P_1$, $P_2$,
$P_3$, \ldots, such that $P_{k+1}$ is the image of $P_k$ under a rotation with
center $A_{k+1}$ through angle $120^\circ$ clockwise (for $k = 0$, 1, 2,
\ldots).  Prove that if $P_{1986} = P_0$, then the triangle $A_1 A_2 A_3$ is
equilateral.

\item
To each vertex of a regular pentagon an integer is assigned in such a way that
the sum of all five numbers is positive.  If three consecutive vertices are
assigned the numbers $x$, $y$, $z$ respectively and $y < 0$ then the following
operation is allowed: the numbers $x$, $y$, $z$ are replaced by $x + y$, $-y$,
$z + y$ respectively.  Such an operation is performed repeatedly as long as at
least one of the five numbers is negative.  Determine whether this procedure
necessarily comes to and end after a finite number of steps.
\end{enumerate}

\pagebreak %% DAY 2
\begin{center}
${\bf 27}^{\mbox{\bf th}}$ {\bf International
Mathematical Olympiad} \\[.1in]
{\bf Warsaw, Poland} \\ [.05in]
{\bf Day II}\\[.05in]
{\bf July 10, 1986}
\end{center}

\vspace*{.3in}

\begin{enumerate}
\setcounter{enumi}{3}
\item
Let $A$, $B$ be adjacent vertices of a regular $n$-gon ($n \geq 5$) in the
plane having center at $O$.  A triangle $XYZ$, which is congruent to and
initially conincides with $OAB$, moves in the plane in such a way that $Y$ and
$Z$ each trace out the whole boundary of the polygon, $X$ remaining inside the
polygon.  Find the locus of $X$.

\item
Find all functions $f$, defined on the non-negative real numbers and taking
non-negative real values, such that:
  \begin{enumerate}
  \item[(i)] $f(xf(y)) f(y) = f(x + y)$ for all $x$, $y \geq 0$,
  \item[(ii)] $f(2) = 0$,
  \item[(iii)] $f(x) \neq 0$ for $0 \leq x < 2$.
  \end{enumerate}

\item
One is given a finite set of points in the plane, each point having integer
coordinates.  Is it always possible to color some of the points in the set red
and the remaining points white in such a way that for any straight line $L$
parallel to either one of the coordinate axes the difference (in absolute
value) between the numbers of white point and red points on $L$ is not greater
than 1?
\end{enumerate}
\end{document}
