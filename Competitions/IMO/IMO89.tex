\documentclass[12pt]{article}
\usepackage{amsfonts}

\pagestyle{empty}
\setlength{\oddsidemargin}{.15in}
\setlength{\evensidemargin}{.15in}
\setlength{\textwidth}{6in}
\setlength{\textheight}{9.25in}
\setlength{\topmargin}{.2in}
\setlength{\headheight}{0in}
\setlength{\headsep}{0in}
\setlength{\parskip}{20pt}
\setlength{\labelsep}{10pt}
\setlength{\parindent}{0pt}
\setlength{\medskipamount}{3ex}
\setlength{\smallskipamount}{1ex}

\begin{document}
\begin{center}
${\bf 30}^{\mbox{\bf th}}$ {\bf International
Mathematical Olympiad} \\[.1in]
{\bf Braunschweig, Germany} \\ [.05in]
{\bf Day I}\\[.05in]
\end{center}

\vspace*{.3in}

\begin{enumerate}
\item
Prove that the set $\{1, 2, \ldots, 1989\}$ can be expressed as the disjoint
union of subsets $A_i$ ($i = 1$, 2, \ldots, 117) such that:
  \begin{enumerate}
  \item[(i)] Each $A_i$ contains 17 elements;
  \item[(ii)] The sum of all the elements in each $A_i$ is the same.
  \end{enumerate}

\item
In an acute-angled triangle $ABC$ the internal bisector of angle $A$ meets the
circumcircle of the triangle again at $A_1$.  Points $B_1$ and $C_1$ are
defined similarly.  Let $A_0$ be the point of intersection of the line $AA_1$
with the external bisectors of angles $B$ and $C$.  Points $B_0$ and $C_0$ are
defined similarly.  Prove that:
  \begin{enumerate}
  \item[(i)] The area of the triangle $A_0 B_0 C_0$ is twice the area of the
  hexagon $AC_1 BA_1 CB_1$.
  \item[(ii)] The area of the triangle $A_0 B_0 C_0$ is at least four times the
  area of the triangle $ABC$.
  \end{enumerate}

\item
Let $n$ and $k$ be positive integers and let $S$ be a set of $n$ points in the
plane such that
  \begin{enumerate}
  \item[(i)] No three points of $S$ are collinear, and
  \item[(ii)] For any point $P$ of $S$ there are at least $k$ points of $S$
  equidistant from $P$.
  \end{enumerate}
Prove that:
$$k < \frac{1}{2} + \sqrt{2n}.$$
\end{enumerate}

\pagebreak %% DAY 2
\begin{center}
${\bf 30}^{\mbox{\bf th}}$ {\bf International
Mathematical Olympiad} \\[.1in]
{\bf Braunschweig, Germany} \\ [.05in]
{\bf Day II}\\[.05in]
\end{center}

\vspace*{.3in}

\begin{enumerate}
\setcounter{enumi}{3}
\item
Let $ABCD$ be a convex quadrilateral such that the sides $AB$, $AD$, $BC$
satisfy $AB = AD + BC$.  There exists a point $P$ inside the quadrilateral at a
distance $h$ from the line $CD$ such that $AP = h + AD$ and $BP = h + BC$.
Show that:
$$\frac{1}{\sqrt{h}} \geq \frac{1}{\sqrt{AD}} + \frac{1}{\sqrt{BC}}.$$

\item
Prove that for each positive integer $n$ there exist $n$ consecutive positive
integers none of which is an integral power of a prime number.

\item
A permutation $(x_1, x_2, \ldots, x_m)$ of the set $\{1, 2, \ldots, 2n\}$,
where $n$ is a positive integer, is said to have property $P$ if $|x_i -
x_{i+1}| = n$ for at least one $i$ in $\{1, 2, \ldots, 2n - 1\}$.  Show that,
for each $n$, there are more permutations with property $P$ than without.
\end{enumerate}
\end{document}
