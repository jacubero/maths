%% This document created by Scientific Notebook (R) Version 3.0


\documentclass[12pt,thmsa]{article}
%%%%%%%%%%%%%%%%%%%%%%%%%%%%%%%%%%%%%%%%%%%%%%%%%%%%%%%%%%%%%%%%%%%%%%%%%%%%%%%%%%%%%%%%%%%%%%%%%%%%%%%%%%%%%%%%%%%%%%%%%%%%
\usepackage{sw20jart}

%TCIDATA{TCIstyle=article/art4.lat,jart,sw20jart}

%TCIDATA{<META NAME="GraphicsSave" CONTENT="32">}
%TCIDATA{Created=Mon Aug 19 14:52:24 1996}
%TCIDATA{LastRevised=Mon Feb 10 15:52:30 1997}
%TCIDATA{Language=American English}
%TCIDATA{CSTFile=Lab Report.cst}
%TCIDATA{PageSetup=72,72,72,72,0}
%TCIDATA{AllPages=
%F=36,\PARA{038<p type="texpara" tag="Body Text" >\hfill \thepage}
%}


\input{tcilatex}
\begin{document}


\section{Eighth International Olympiad, 1966}

\subsection{1966/1.}

In a mathematical contest, three problems, $A,B,C$ were posed. Among the
participants there were $25$ students who solved at least one problem each.
Of all the contestants who did not solve problem $A,$ the number who solved $%
B$ was twice the number who solved $C.$ The number of students who solved
only problem $A$ was one more than the number of students who solved $A$ and
at least one other problem. Of all students who solved just one problem,
half did not solve problem $A.$ How many students solved only problem $B$?

\subsection{1966/2.}

Let $a,b,c$ be the lengths of the sides of a triangle, and $\alpha ,\beta
,\gamma $, respectively, the angles opposite these sides. Prove that if
\[
a+b=\tan \frac{\gamma }{2}(a\tan \alpha +b\tan \beta ),
\]

the triangle is isosceles.

\subsection{1966/3.}

Prove: The sum of the distances of the vertices of a regular tetrahedron
from the center of its circumscribed sphere is less than the sum of the
distances of these vertices from any other point in space.

\subsection{1966/4.}

Prove that for every natural number $n,$ and for every real number $x\neq
k\pi /2^{t}(t=0,1,...,n;k$ any integer)
\[
\frac{1}{\sin 2x}+\frac{1}{\sin 4x}+\cdots +\frac{1}{\sin 2^{n}x}=\cot
x-\cot 2^{n}x.
\]

\subsection{1966/5.}

Solve the system of equations
\[
\begin{array}{lllll}
& \left| a_{1}-a_{2}\right| x_{2} & +\left| a_{1}-a_{3}\right| x_{3} & 
+\left| a_{1}-a_{4}\right| x_{4} & =1 \\ 
\left| a_{2}-a_{1}\right| x_{1} &  & +\left| a_{2}-a_{3}\right| x_{3} & 
+\left| a_{2}-a_{3}\right| x_{3} & =1 \\ 
\left| a_{3}-a_{1}\right| x_{1} & +\left| a_{3}-a_{2}\right| x_{2} &  &  & =1
\\ 
\left| a_{4}-a_{1}\right| x_{1} & +\left| a_{4}-a_{2}\right| x_{2} & +\left|
a_{4}-a_{3}\right| x_{3} &  & =1
\end{array}
\]

where $a_{1},a_{2},a_{3},a_{4}$ are four different real numbers.

\subsection{1966/6.}

In the interior of sides $BC,CA,AB$ of triangle $ABC,$ any points $K,L,M,$
respectively, are selected. Prove that the area of at least one of the
triangles $AML,BKM,CLK$ is less than or equal to one quarter of the area of
triangle $ABC.$

\end{document}
