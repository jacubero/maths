%% This document created by Scientific Notebook (R) Version 3.0


\documentclass[12pt,thmsa]{article}
%%%%%%%%%%%%%%%%%%%%%%%%%%%%%%%%%%%%%%%%%%%%%%%%%%%%%%%%%%%%%%%%%%%%%%%%%%%%%%%%%%%%%%%%%%%%%%%%%%%%%%%%%%%%%%%%%%%%%%%%%%%%
\usepackage{sw20jart}

%TCIDATA{TCIstyle=article/art4.lat,jart,sw20jart}

%TCIDATA{<META NAME="GraphicsSave" CONTENT="32">}
%TCIDATA{Created=Mon Aug 19 14:52:24 1996}
%TCIDATA{LastRevised=Mon Feb 10 11:09:48 1997}
%TCIDATA{Language=American English}
%TCIDATA{CSTFile=Lab Report.cst}
%TCIDATA{PageSetup=72,72,72,72,0}
%TCIDATA{AllPages=
%F=36,\PARA{038<p type="texpara" tag="Body Text" >\hfill \thepage}
%}


\input{tcilatex}
\begin{document}


\section{Twenty-third International Olympiad, 1982}

1982/1. The function $f(n)$ is defined for all positive integers $n$ and
takes on non-negative integer values. Also, for all $m,n$%
\[
f(m+n)-f(m)-f(n)=0\text{ or }1 
\]
\[
f(2)=0,f(3)>0,\text{ and }f(9999)=3333. 
\]
Determine $f(1982).$

1982/2. A non-isosceles triangle $A_{1}A_{2}A_{3}$ is given with sides $%
a_{1},a_{2},a_{3}$ ($a_{i}$ is the side opposite $A_{i}$). For all $%
i=1,2,3,M_{i}$ is the midpoint of side $a_{i}$, and $T_{i}$. is the point
where the incircle touches side $a_{i}$. Denote by $S_{i}$ the reflection of 
$T_{i}$ in the interior bisector of angle $A_{i}$. Prove that the lines $%
M_{1},S_{1},M_{2}S_{2},$ and $M_{3}S_{3}$ are concurrent.

1982/3. Consider the infinite sequences $\{x_{n}\}$ of positive real numbers
with the following properties:

\[
x_{0}=1,\text{and for all }i\ge 0,x_{i+1}\le x_{i}. 
\]

(a) Prove that for every such sequence, there is an $n\ge 1$ such that 
\[
\frac{x_{0}^{2}}{x_{1}}+\frac{x_{1}^{2}}{x_{2}}+\cdots +\frac{x_{n-1}^{2}}{%
x_{n}}\ge 3.999. 
\]

(b) Find such a sequence for which 
\[
\frac{x_{0}^{2}}{x_{1}}+\frac{x_{1}^{2}}{x_{2}}+\cdots +\frac{x_{n-1}^{2}}{%
x_{n}}<4. 
\]

1982/4. Prove that if $n$ is a positive integer such that the equation 
\[
x^{3}-3xy^{2}+y^{3}=n 
\]
has a solution in integers $(x,y),$ then it has at least three such
solutions.

Show that the equation has no solutions in integers when $n=2891.$

1982/5. The diagonals $AC$ and $CE$ of the regular hexagon $ABCDEF$ are
divided by the inner points $M$ and $N$, respectively, so that 
\[
\frac{AM}{AC}=\frac{CN}{CE}=r. 
\]
Determine $r$ if $B,M,$ and $N$ are collinear.

1982/6. Let $S$ be a square with sides of length $100,$ and let $L$ be a
path within $S$ which does not meet itself and which is composed of line
segments $A_{0}A_{1},A_{1}A_{2},\cdots ,A_{n-1}A_{n}$ with $A_{0}\neq A_{n}$%
. Suppose that for every point $P$ of the boundary of $S$ there is a point
of $L$ at a distance from $P$ not greater than $1/2.$ Prove that there are
two points $X$ and $Y$ in $L$ such that the distance between $X$ and $Y$ is
not greater than $1,$ and the length of that part of $L$ which lies between $%
X$ and $Y$ is not smaller than $198.$

\end{document}
