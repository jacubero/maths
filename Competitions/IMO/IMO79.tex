%% This document created by Scientific Notebook (R) Version 3.0


\documentclass[12pt,thmsa]{article}
%%%%%%%%%%%%%%%%%%%%%%%%%%%%%%%%%%%%%%%%%%%%%%%%%%%%%%%%%%%%%%%%%%%%%%%%%%%%%%%%%%%%%%%%%%%%%%%%%%%%%%%%%%%%%%%%%%%%%%%%%%%%
\usepackage{sw20jart}

%TCIDATA{TCIstyle=article/art4.lat,jart,sw20jart}

%TCIDATA{<META NAME="GraphicsSave" CONTENT="32">}
%TCIDATA{Created=Mon Aug 19 14:52:24 1996}
%TCIDATA{LastRevised=Mon Feb 10 11:04:41 1997}
%TCIDATA{CSTFile=Lab Report.cst}
%TCIDATA{PageSetup=72,72,72,72,0}
%TCIDATA{AllPages=
%F=36,\PARA{038<p type="texpara" tag="Body Text" >\hfill \thepage}
%}


\input{tcilatex}
\begin{document}


\section{Twenty-first International Olympiad, 1979}

1979/1. Let $p$ and $q$ be natural numbers such that 
\[
\frac{p}{q}=1-\frac{1}{2}+\frac{1}{3}-\frac{1}{4}+\cdots -\frac{1}{1318}+%
\frac{1}{1319}. 
\]
Prove that $p$ is divisible by $1979.$

1979/2. A prism with pentagons $A_{1}A_{2}A_{3}A_{4}A_{5}$ and $%
B_{1}B_{2}B_{3}B_{4}B_{5}$ as top and bottom faces is given. Each side of
the two pentagons and each of the line-segments $A_{i}B_{j}$ for all $%
i,j=1,...,5,$ is colored either red or green. Every triangle whose vertices
are vertices of the prism and whose sides have all been colored has two
sides of a different color. Show that all $10$ sides of the top and bottom
faces are the same color.

1979/3. Two circles in a plane intersect. Let $A$ be one of the points of
intersection. Starting simultaneously from $A$ two points move with constant
speeds, each point travelling along its own circle in the same sense. The
two points return to A simultaneously after one revolution. Prove that there
is a fixed point $P$ in the plane such that, at any time, the distances from 
$P$ to the moving points are equal.

1979/4. Given a plane $\pi ,$ a point $P$ in this plane and a point $Q$ not
in $\pi ,$ find all points $R$ in $\pi $ such that the ratio $(QP+PA)/QR$ is
a maximum.

1979/5. Find all real numbers a for which there exist non-negative real
numbers $x_{1},x_{2},x_{3},x_{4},x_{5}$ satisfying the relations 
\[
\sum_{k=1}^{5}kx_{k}=a,\sum_{k=1}^{5}k^{3}x_{k}=a^{2},%
\sum_{k=1}^{5}k^{5}x_{k}=a^{3}. 
\]

1979/6. Let $A$ and $E$ be opposite vertices of a regular octagon. A frog
starts jumping at vertex $A$. From any vertex of the octagon except $E,$ it
may jump to either of the two adjacent vertices. When it reaches vertex $E,$
the frog stops and stays there.. Let $a_{n}$ be the number of distinct paths
of exactly $n$ jumps ending at $E.$ Prove that $a_{2n-1}=0,$%
\[
a_{2n}=\frac{1}{\sqrt{2}}(x^{n-1}-y^{n-1}),n=1,2,3,\cdots , 
\]
where $x=2+\sqrt{2}$ and $y=2-\sqrt{2}.$

Note. A path of $n$ jumps is a sequence of vertices $(P_{0},...,P_{n})$ such
that

(i) $P_{0}=A,P_{n}=E;$

(ii) for every $i,0\le i\le n-1,P_{i}$ is distinct from $E;$

(iii) for every $i,0\leq i\leq n-1,P_{i}$ and $P_{i+1}$ are adjacent.

\end{document}
