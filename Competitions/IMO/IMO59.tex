%% This document created by Scientific Notebook (R) Version 3.0


\documentclass[12pt,thmsa]{article}
%%%%%%%%%%%%%%%%%%%%%%%%%%%%%%%%%%%%%%%%%%%%%%%%%%%%%%%%%%%%%%%%%%%%%%%%%%%%%%%%%%%%%%%%%%%%%%%%%%%%%%%%%%%%%%%%%%%%%%%%%%%%
\usepackage{sw20jart}

%TCIDATA{TCIstyle=article/art4.lat,jart,sw20jart}

%TCIDATA{<META NAME="GraphicsSave" CONTENT="32">}
%TCIDATA{Created=Mon Aug 19 14:52:24 1996}
%TCIDATA{LastRevised=Mon Feb 10 11:48:19 1997}
%TCIDATA{CSTFile=Lab Report.cst}
%TCIDATA{PageSetup=72,72,72,72,0}
%TCIDATA{AllPages=
%F=36,\PARA{038<p type="texpara" tag="Body Text" >\hfill \thepage}
%}


\input{tcilatex}
\begin{document}


\section{First International Olympiad, 1959}

\subsection{1959/1.}

Prove that the fraction $\frac{21n+4}{14n+3}$ is irreducible for every
natural number $n.$

\subsection{1959/2.}

For what real values of $x$ is

\[
\sqrt{(x+\sqrt{2x-1})}+\sqrt{(x-\sqrt{2x-1})}=A,
\]

given (a) $A=\sqrt{2},$ (b) $A=1$, (c) $A=2,$ where only non-negative real
numbers are admitted for square roots?

\subsection{1959/3.}

Let $a,b,c$ be real numbers. Consider the quadratic equation in $\cos x:$

\[
a\cos ^{2}x+b\cos x+c=0.
\]

Using the numbers $a,b,c,$ form a quadratic equation in $\cos 2x$, whose
roots are the same as those of the original equation. Compare the equations
in $\cos x$ and $\cos 2x$ for $a=4,b=2,c=-1.$

\subsection{1959/4.}

Construct a right triangle with given hypotenuse $c$ such that the median
drawn to the hypotenuse is the geometric mean of the two legs of the
triangle.

\subsection{1959/5.}

An arbitrary point $M$ is selected in the interior of the segment $AB.$ The
squares $AMCD$ and $MBEF$ are constructed on the same side of $AB,$ with the
segments $AM$ and $MB$ as their respective bases. The circles circumscribed
about these squares, with centers $P$ and $Q,$ intersect at $M$ and also at
another point $N.$ Let $N^{\prime }$ denote the point of intersection of the
straight lines $AF$ and $BC.$

(a) Prove that the points $N$ and $N^{\prime }$ coincide.

(b) Prove that the straight lines $MN$ pass through a fixed point $S$
independent of the choice of $M.$

(c) Find the locus of the midpoints of the segments $PQ$ as $M$ varies
between $A$ and $B.$

\subsection{1959/6.}

Two planes, $P$ and $Q,$ intersect along the line $p.$ The point $A$ is
given in the plane $P,$ and the point $C$ in the plane $Q;$ neither of these
points lies on the straight line $p.$ Construct an isosceles trapezoid $ABCD$
(with $AB$ parallel to $CD$) in which a circle can be inscribed, and with
vertices $B$ and $D$ lying in the planes $P$ and $Q$ respectively.

\end{document}
