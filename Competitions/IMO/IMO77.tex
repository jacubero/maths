%% This document created by Scientific Notebook (R) Version 3.0


\documentclass[12pt,thmsa]{article}
%%%%%%%%%%%%%%%%%%%%%%%%%%%%%%%%%%%%%%%%%%%%%%%%%%%%%%%%%%%%%%%%%%%%%%%%%%%%%%%%%%%%%%%%%%%%%%%%%%%%%%%%%%%%%%%%%%%%%%%%%%%%
\usepackage{sw20jart}

%TCIDATA{TCIstyle=article/art4.lat,jart,sw20jart}

%TCIDATA{<META NAME="GraphicsSave" CONTENT="32">}
%TCIDATA{Created=Mon Aug 19 14:52:24 1996}
%TCIDATA{LastRevised=Mon Feb 10 19:23:44 1997}
%TCIDATA{CSTFile=Lab Report.cst}
%TCIDATA{PageSetup=72,72,72,72,0}
%TCIDATA{AllPages=
%F=36,\PARA{038<p type="texpara" tag="Body Text" >\hfill \thepage}
%}


\input{tcilatex}
\begin{document}


\section{Nineteenth International Mathematical Olympiad, 1977}

\subsection{1977/1. }

Equilateral triangles $ABK,BCL,CDM,DAN$ are constructed inside the square $%
ABCD.$ Prove that the midpoints of the four segments $KL,LM,MN,NK$ and the
midpoints of the eight segments $AKBK,BL,CL,CM,DM,DN,AN$ are the twelve
vertices of a regular dodecagon.

\subsection{1977/2. }

In a finite sequence of real numbers the sum of any seven successive terms
is negative, and the sum of any eleven successive terms is positive.
Determine the maximum number of terms in the sequence.

\subsection{1977/3. }

Let $n$ be a given integer $>2,$ and let $V_{n}$ be the set of integers $%
1+kn,$ where $k=1,2,...$. A number $m\in V_{n}$  is called \emph{%
indecomposable} in $V_{n}$ if there do not exist numbers $p,q\in V_{n}$ such
that $pq=m.$ Prove that there exists a number $r\in V_{n}$ that can be
expressed as the product of elements indecomposable in $V_{n}$ in more than
one way. (Products which differ only in the order of their factors will be
considered the same.)

\subsection{1977/4. }

Four real constants $a,b,A,B$ are given, and

\[
f(\theta )=1-a\cos \theta -b\sin \theta -A\cos 2\theta -B\sin 2\theta .
\]
Prove that if $f(\theta )\geq 0$ for all real $\theta $, then

\begin{center}
$a^{2}+b^{2}\leq 2$ and $A^{2}+B^{2}\leq 1.$
\end{center}

\subsection{1977/5. }

Let $a$ and $b$ be positive integers. When $a^{2}+b^{2}$ is divided by $a+b,$
the quotient is $q$ and the remainder is $r.$ Find all pairs $(a,b)$ such
that $q^{2}+r=1977.$

\subsection{1977/6. }

Let $f(n)$ be a function defined on the set of all positive integers and
having all its values in the same set. Prove that if
\[
f(n+1)>f(f(n))
\]
for each positive integer $n,$ then

\begin{center}
$f(n)=n$ for each $n.$
\end{center}

\end{document}
