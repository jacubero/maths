%% This document created by Scientific Notebook (R) Version 3.0


\documentclass[12pt,thmsa]{article}
%%%%%%%%%%%%%%%%%%%%%%%%%%%%%%%%%%%%%%%%%%%%%%%%%%%%%%%%%%%%%%%%%%%%%%%%%%%%%%%%%%%%%%%%%%%%%%%%%%%%%%%%%%%%%%%%%%%%%%%%%%%%
\usepackage{sw20jart}

%TCIDATA{TCIstyle=article/art4.lat,jart,sw20jart}

%TCIDATA{<META NAME="GraphicsSave" CONTENT="32">}
%TCIDATA{Created=Mon Aug 19 14:52:24 1996}
%TCIDATA{LastRevised=Mon Feb 10 15:02:36 1997}
%TCIDATA{Language=American English}
%TCIDATA{CSTFile=Lab Report.cst}
%TCIDATA{PageSetup=72,72,72,72,0}
%TCIDATA{AllPages=
%F=36,\PARA{038<p type="texpara" tag="Body Text" >\hfill \thepage}
%}


\input{tcilatex}
\begin{document}


\section{Fifth International Olympiad, 1963}

\subsection{1963/1.}

Find all real roots of the equation
\[
\sqrt{x^{2}-p}+2\sqrt{x^{2}-1}=x,
\]

where $p$ is a real parameter.

\subsection{1963/2.}

Point $A$ and segment $BC$ are given. Determine the locus of points in space
which are vertices of right angles with one side passing through $A,$ and
the other side intersecting the segment $BC.$

\subsection{1963/3.}

In an $n$-gon all of whose interior angles are equal, the lengths of
consecutive sides satisfy the relation
\[
a_{1}\geq a_{2}\geq \cdots \geq a_{n}.
\]

Prove that $a_{1}=a_{2}=\cdots =a_{n}.$

\subsection{1963/4.}

Find all solutions $x_{1},x_{2},x_{3},x_{4},x_{5}$ of the system
\begin{eqnarray*}
x_{5}+x_{2} &=&yx_{1} \\
x_{1}+x_{3} &=&yx_{2} \\
x_{2}+x_{4} &=&yx_{3} \\
x_{3}+x_{5} &=&yx_{4} \\
x_{4}+x_{1} &=&yx_{5},
\end{eqnarray*}

where $y$ is a parameter.

\subsection{1963/5.}

Prove that $\cos \frac{\pi }{7}-\cos \frac{2\pi }{7}+\cos \frac{3\pi }{7}=%
\frac{1}{2}.$

\subsection{1963/6.}

Five students, $A,B,C,D,E,$ took part in a contest. One prediction was that
the contestants would finish in the order $ABCDE.$ This prediction was very
poor. In fact no contestant finished in the position predicted, and no two
contestants predicted to finish consecutively actually did so. A second
prediction had the contestants finishing in the order $DAECB.$ This
prediction was better. Exactly two of the contestants finished in the places
predicted, and two disjoint pairs of students predicted to finish
consecutively actually did so. Determine the order in which the contestants
finished.

\end{document}
