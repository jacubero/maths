%% This document created by Scientific Notebook (R) Version 3.0


\documentclass[12pt,thmsa]{article}
%%%%%%%%%%%%%%%%%%%%%%%%%%%%%%%%%%%%%%%%%%%%%%%%%%%%%%%%%%%%%%%%%%%%%%%%%%%%%%%%%%%%%%%%%%%%%%%%%%%%%%%%%%%%%%%%%%%%%%%%%%%%
\usepackage{sw20jart}

%TCIDATA{TCIstyle=article/art4.lat,jart,sw20jart}

%TCIDATA{<META NAME="GraphicsSave" CONTENT="32">}
%TCIDATA{Created=Mon Aug 19 14:52:24 1996}
%TCIDATA{LastRevised=Mon Feb 10 14:49:37 1997}
%TCIDATA{CSTFile=Lab Report.cst}
%TCIDATA{PageSetup=72,72,72,72,0}
%TCIDATA{AllPages=
%F=36,\PARA{038<p type="texpara" tag="Body Text" >\hfill \thepage}
%}


\input{tcilatex}
\begin{document}


\section{Fourth International Olympiad, 1962}

\subsection{1962/1.}

Find the smallest natural number $n$ which has the following properties:

(a) Its decimal representation has $6$ as the last digit. 

(b) If the last digit $6$ is erased and placed in front of the remaining
digits, the resulting number is four times as large as the original number $%
n.$

\subsection{1962/2.}

Determine all real numbers $x$ which satisfy the inequality:
\[
\sqrt{3-x}-\sqrt{x+1}>\frac{1}{2}.
\]

\subsection{1962/3.}

Consider the cube $ABCDA^{\prime }B^{\prime }C^{\prime }D^{\prime }$ ($ABCD$
and $A^{\prime }B^{\prime }C^{\prime }D^{\prime }$ are the upper and lower
bases, respectively, and edges $AA^{\prime },BB^{\prime },CC^{\prime
},DD^{\prime }$ are parallel). The point $X$ moves at constant speed along
the perimeter of the square $ABCD$ in the direction $ABCDA,$ and the point $Y
$ moves at the same rate along the perimeter of the square $B^{\prime
}C^{\prime }CB$ in the direction $B^{\prime }C^{\prime }CBB^{\prime }$.
Points $X$ and $Y$ begin their motion at the same instant from the starting
positions $A$ and $B^{\prime }$, respectively. Determine and draw the locus
of the midpoints of the segments $XY.$

\subsection{1962/4.}

Solve the equation $\cos ^{2}x+\cos ^{2}2x+\cos ^{2}3x=1.$

\subsection{1962/5.}

On the circle $K$ there are given three distinct points $A,B,C.$ Construct
(using only straightedge and compasses) a fourth point $D$ on $K$ such that
a circle can be inscribed in the quadrilateral thus obtained.

\subsection{1962/6.}

Consider an isosceles triangle. Let $r$ be the radius of its circumscribed
circle and $\rho $ the radius of its inscribed circle. Prove that the
distance d between the centers of these two circles is
\[
d=\sqrt{r(r-2\rho )}.
\]

\subsection{1962/7.}

The tetrahedron $SABC$ has the following property: there exist five spheres,
each tangent to the edges $SA,SB,SC,BCCA,AB,$ or to their extensions.

(a) Prove that the tetrahedron $SABC$ is regular.

(b) Prove conversely that for every regular tetrahedron five such spheres
exist.

\end{document}
