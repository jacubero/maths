%% This document created by Scientific Notebook (R) Version 3.0


\documentclass[12pt,thmsa]{article}
%%%%%%%%%%%%%%%%%%%%%%%%%%%%%%%%%%%%%%%%%%%%%%%%%%%%%%%%%%%%%%%%%%%%%%%%%%%%%%%%%%%%%%%%%%%%%%%%%%%%%%%%%%%%%%%%%%%%%%%%%%%%
\usepackage{sw20jart}

%TCIDATA{TCIstyle=article/art4.lat,jart,sw20jart}

%TCIDATA{<META NAME="GraphicsSave" CONTENT="32">}
%TCIDATA{Created=Mon Aug 19 14:52:24 1996}
%TCIDATA{LastRevised=Mon Feb 10 11:21:36 1997}
%TCIDATA{Language=American English}
%TCIDATA{CSTFile=Lab Report.cst}
%TCIDATA{PageSetup=72,72,72,72,0}
%TCIDATA{AllPages=
%F=36,\PARA{038<p type="texpara" tag="Body Text" >\hfill \thepage}
%}


\input{tcilatex}
\begin{document}


\section{Twenty-fifth International Olympiad, 1984}

1984/1. Prove that $0\le yz+zx+xy-2xyz\le 7/27,$ where $x,y$ and $z$ are
non-negative real numbers for which $x+y+z=1.$

1984/2. Find one pair of positive integers $a$ and $b$ such that:

(i) $ab(a+b)$ is not divisible by $7;$

(ii) $(a+b)^{7}-a^{7}-b^{7}$ is divisible by $7^{7}$ .

Justify your answer.

1984/3. In the plane two different points $O$ and $A$ are given. For each
point $X$ of the plane, other than $O$, denote by $a(X)$ the measure of the
angle between $OA$ and $OX$ in radians, counterclockwise from $OA(0\le
a(X)<2\pi ).$ Let $C(X)$ be the circle with center $O$ and radius of length $%
OX+a(X)/OX.$ Each point of the plane is colored by one of a finite number of
colors. Prove that there exists a point $Y$ for which $a(Y)>0$ such that its
color appears on the circumference of the circle $C(Y).$

1984/4. Let $ABCD$ be a convex quadrilateral such that the line $CD$ is a
tangent to the circle on $AB$ as diameter. Prove that the line $AB$ is a
tangent to the circle on $CD$ as diameter if and only if the lines $BC$ and $%
AD$ are parallel.

1984/5. Let $d$ be the sum of the lengths of all the diagonals of a plane
convex polygon with $n$ vertices $(n>3)$, and let $p$ be its perimeter.
Prove that 
\[
n-3<\frac{2d}{p}<\left[ \frac{n}{2}\right] \left[ \frac{n+1}{2}\right] -2, 
\]
where $[x]$ denotes the greatest integer not exceeding $x.$

1984/6. Let $a,b,c$ and $d$ be odd integers such that $0<a<b<c<d$ and $ad=bc.
$ Prove that if $a+d=2^{k}$ and $b+c=2^{m}$ for some integers $k$ and $m,$
then $a=1.$

\end{document}
