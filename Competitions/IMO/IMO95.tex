\documentstyle[12pt]{article}
\setlength{\baselineskip}{0.25in}
\setlength{\leftmargin}{0.0in}
\setlength{\textwidth}{6.5in}
\setlength{\topmargin}{0.0in}
\setlength{\textheight}{9in}
\setlength{\headheight}{0.0in}
\setlength{\headsep}{0.0in}
\setlength{\oddsidemargin}{0.0in}

\begin{document}

\begin{center}
${\bf 36}^{\mbox{\bf th}}$
{\bf International Mathematical Olympiad} \\[.1in]
{\bf First Day - Toronto - July 19, 1995} \\
{\bf Time Limit: 4}${{\bf 1}\over{\bf 2}}$ {\bf 
hours} 
\end{center}
\begin{enumerate}
\item
Let $A,B,C, D$ be four distinct points on a line, in that order. The 
circles with diameters $AC$ and $BD$ intersect at $X$ and $Y$. 
The line $XY$ meets $BC$ at $Z$. Let $P$ be a point on the line 
$XY$ other than $Z$. The line $CP$ intersects the circle with diameter 
$AC$ at $C$ and $M$, and the line $BP$ intersects the circle 
with diameter $BD$ at $B$ and $N$. Prove that the lines $AM, 
DN, XY$ are concurrent.
\item
Let $a, b, c$ be positive real numbers such that $abc = 1$. Prove that
\[
\frac{1}{a^3(b+c)} + \frac{1}{b^3(c+a)} + \frac{1}{c^3(a+b)} \geq \frac{3}{2}.
\]
\item
Determine all integers $n > 3$ for which there exist $n$ points $A_1, 
\dots, A_n$ in the plane, no three collinear, and real numbers $r_1, \dots, 
r_n$ such that for $1 \leq i <j< k \leq n$, the area of $\triangle 
A_iA_jA_k$ is $r_i + r_j + r_k$.
\end{enumerate}

\begin{center}
${\bf 36}^{\mbox{\bf th}}$
{\bf International Mathematical Olympiad} \\[.1in]
{\bf Second Day - Toronto - July 20, 1995} \\
{\bf Time Limit: 4}${{\bf 1}\over{\bf 2}}$ {\bf 
hours} 
\end{center}
\begin{enumerate}
\setcounter{enumi}{3}
\item
Find the maximum value of $x_0$ for which there exists a sequence $x_0, x_1
\dots, x_{1995}$ of positive reals with $x_0 = x_{1995}$, such that
for $i = 1, \dots, 1995$,
\[
x_{i-1} + \frac{2}{x_{i-1}} = 2x_i + \frac{1}{x_i}.
\]
\item
Let $ABCDEF$ be a convex hexagon with $AB=BC=CD$ and $DE=EF=FA$, such 
that $\angle BCD = \angle EFA = \pi/3$. Suppose $G$ and $H$ are points in 
the interior of the hexagon such that $\angle AGB = \angle DHE = 2\pi/3$. 
Prove that $AG + GB + GH + DH + HE \geq CF$.
\item
Let $p$ be an odd prime number. How many $p$-element subsets 
$A$ of $\{1, 2, \dots\ 2p\}$ are there, the sum of whose elements is 
divisible by $p$?
\end{enumerate}
\end{document}


