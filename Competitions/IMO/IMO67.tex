%% This document created by Scientific Notebook (R) Version 3.0


\documentclass[12pt,thmsa]{article}
%%%%%%%%%%%%%%%%%%%%%%%%%%%%%%%%%%%%%%%%%%%%%%%%%%%%%%%%%%%%%%%%%%%%%%%%%%%%%%%%%%%%%%%%%%%%%%%%%%%%%%%%%%%%%%%%%%%%%%%%%%%%
\usepackage{sw20jart}

%TCIDATA{TCIstyle=article/art4.lat,jart,sw20jart}

%TCIDATA{<META NAME="GraphicsSave" CONTENT="32">}
%TCIDATA{Created=Mon Aug 19 14:52:24 1996}
%TCIDATA{LastRevised=Mon Feb 10 16:10:04 1997}
%TCIDATA{Language=American English}
%TCIDATA{CSTFile=Lab Report.cst}
%TCIDATA{PageSetup=72,72,72,72,0}
%TCIDATA{AllPages=
%F=36,\PARA{038<p type="texpara" tag="Body Text" >\hfill \thepage}
%}


\input{tcilatex}
\begin{document}


\section{Ninth International Olympiad, 1967}

\subsection{1967/1.}

Let $ABCD$ be a parallelogram with side lengths $AB=a,AD=1,$ and with $%
\angle BAD=\alpha $. If $\Delta ABD$ is acute, prove that the four circles
of radius $1$ with centers $A,B,C,D$ cover the parallelogram if and only if
\[
a\leq \cos \alpha +\sqrt{3}\sin \alpha .
\]

\subsection{1967/2.}

Prove that if one and only one edge of a tetrahedron is greater than $1,$
then its volume is $\leq 1/8.$

\subsection{1967/3.}

Let $k,m,n$ be natural numbers such that $m+k+1$ is a prime greater than $%
n+1.$ Let $c_{s}=s(s+1).$ Prove that the product
\[
(c_{m+1}-c_{k})(c_{m+2}-c_{k})\cdots (c_{m+n}-c_{k})
\]

is divisible by the product $c_{1}c_{2}\cdots c_{n}$.

\subsection{1967/4.}

Let $A_{0}B_{0}C_{0}$ and $A_{1}B_{1}C_{1}$ be any two acute-angled
triangles. Consider all triangles $ABC$ that are similar to $\Delta
A_{1}B_{1}C_{1}$ (so that vertices $A_{1},B_{1},C_{1}$ correspond to
vertices $A,B,C,$ respectively) and circumscribed about triangle $%
A_{0}B_{0}C_{0}$ (where $A_{0}$ lies on $BC,B_{0}$ on $CA,$ and $AC_{0}$ on $%
AB$). Of all such possible triangles, determine the one with maximum area,
and construct it.

\subsection{1967/5.}

Consider the sequence $\{c_{n}\}$, where

\begin{eqnarray*}
c_{1} &=&a_{1}+a_{2}+\cdots +a_{8} \\
c_{2} &=&a_{1}^{2}+a_{2}^{2}+\cdots +a_{8}^{2} \\
&&\cdots  \\
c_{n} &=&a_{1}^{n}+a_{2}^{n}+\cdots +a_{8}^{n} \\
&&\cdots 
\end{eqnarray*}

in which $a_{1},a_{2},\cdots ,a_{8}$ are real numbers not all equal to zero.
Suppose that an infinite number of terms of the sequence $\{c_{n}\}$ are
equal to zero. Find all natural numbers $n$ for which $c_{n}=0.$

\subsection{1967/6.}

In a sports contest, there were $m$ medals awarded on $n$ successive days ($%
n>1$). On the first day, one medal and $1/7$ of the remaining $m-1$ medals
were awarded. On the second day, two medals and $1/7$ of the now remaining
medals were awarded; and so on. On the $n$-th and last day, the remaining $n$
medals were awarded. How many days did the contest last, and how many medals
were awarded altogether?

\end{document}
