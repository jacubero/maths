\documentclass[12pt]{article}
\usepackage{amsfonts}

\pagestyle{empty}
\setlength{\oddsidemargin}{.15in}
\setlength{\evensidemargin}{.15in}
\setlength{\textwidth}{6in}
\setlength{\textheight}{9.25in}
\setlength{\topmargin}{.2in}
\setlength{\headheight}{0in}
\setlength{\headsep}{0in}
\setlength{\parskip}{20pt}
\setlength{\labelsep}{10pt}
\setlength{\parindent}{0pt}
\setlength{\medskipamount}{3ex}
\setlength{\smallskipamount}{1ex}

\begin{document}
\begin{center}
${\bf 38}^{\mbox{\bf th}}$ {\bf International
Mathematical Olympiad} \\[.1in]
{\bf Mar del Plata, Argentina} \\ [.05in]
{\bf Day I}\\[.05in]
{\bf July 24, 1997}
\end{center}

\vspace*{.3in}

\begin{enumerate}
\item %% IMO 1
In the plane the points with integer coordinates are the vertices of
unit squares. The
squares are colored alternately black and white (as on a chessboard). 

For any pair of positive integers $m$ and $n$, consider a right-angled triangle
whose
vertices have integer coordinates and whose legs, of lengths $m$ and $n$,
lie along edges of the squares. 

Let $S_1$ be the total area of the black part of the triangle and
$S_2$ be the total area of the
white part. Let
\[
f(m,n) = |S_1 - S_2|.
\]
\begin{enumerate}
\item
Calculate $f(m,n)$ for all positive integers $m$ and $n$ which are either
both even or both odd. 

\item
Prove that $f(m,n) \leq \frac{1}{2} \max\{m,n\}$ for all $m$ and $n$. 

\item
Show that there is no constant $C$ such that $f(m,n) < C$ for all $m$ and $n$. 
\end{enumerate}

\item % IMO 2

The angle at $A$ is the smallest angle of triangle $ABC$.
The points $B$ and $C$ divide the circumcircle of the triangle into
two arcs. Let $U$ be an interior point of the arc between $B$ and $C$
which does not contain $A$. The perpendicular bisectors of $AB$ and
$AC$ meet the line $AU$ at $V$ and $W$, respectively. The lines $BV$
and $CW$ meet at $T$. Show that
\[
AU = TB + TC.
\]

\item %% IMO3
Let $x_1, x_2, \dots, x_n$ be real numbers satisfying the conditions
\[
\left| x_1 + x_2 + \cdots + x_n \right| = 1
\]
and
\[
\left| x_i \right| \leq \frac{n+1}{2} \qquad i = 1, 2, \dots, n.
\]
Show that there exists a permutation $y_1, y_2, \dots, y_n$ of $x_1,
x_2, \dots, x_n$ such that
\[
\left| y_1 + 2y_2 + \cdots + ny_n \right| \leq \frac{n+1}{2}.
\]

\end{enumerate}

\pagebreak %% DAY 2
\begin{center}
${\bf 38}^{\mbox{\bf th}}$ {\bf International
Mathematical Olympiad} \\[.1in]
{\bf Mar del Plata, Argentina} \\ [.05in]
{\bf Day II}\\[.05in]
{\bf July 25, 1997}
\end{center}

\vspace*{.3in}

\begin{enumerate}
\setcounter{enumi}{3}
\item %% IMO4

An $n \times n$ matrix whose entries come from the set $S = \{1, 2,
\dots, 2n-1\}$ is called a \emph{silver} matrix if, for each $i=1, 2,
\dots, n$, the $i$th row and the $i$th column together contain all
elements of $S$. Show that
\begin{enumerate}
\item
there is no silver matrix for $n = 1997$; 
\item
silver matrices exist for infinitely many values of $n$. 
\end{enumerate}

\item %% IMO5
Find all pairs $(a,b)$ of integers $a, b \geq 1$ that satisfy the equation 
\[
a^{b^2} = b^a.
\]

\item %% IMO6

For each positive integer $n$ , let $f(n)$ denote the number of ways
of representing $n$ as a
sum of powers of 2 with nonnegative integer exponents.
Representations which differ only in the ordering of their summands are considered to
be the same. For instance, $f(4)=4$, because the number $4$ can be represented in the
following four ways:
\[
                       4; 2+2; 2+1+1; 1+1+1+1. 
\]
Prove that, for any integer $n \geq 3$, 
\[
2^{n^2/4} < f(2^n) < 2^{n^2/2}.
\]

\end{enumerate}
\end{document}
