\documentclass[12pt]{article}

\pagestyle{empty}
\setlength{\oddsidemargin}{.15in}
\setlength{\evensidemargin}{.15in}
\setlength{\textwidth}{6in}
\setlength{\textheight}{9.25in}
\setlength{\topmargin}{.2in}
\setlength{\headheight}{0in}
\setlength{\headsep}{0in}
\setlength{\parskip}{20pt}
\setlength{\labelsep}{10pt}
\setlength{\parindent}{0pt}
\setlength{\medskipamount}{3ex}
\setlength{\smallskipamount}{1ex}


\begin{document}

%% DAY 1
\begin{center}
${\bf 41}^{\mbox{\bf st}}$ {\bf International
Mathematical Olympiad} \\[.1in]
{\bf Taejon, Republic of Korea} \\ [.05in]
{\bf Day I \hspace{.25in} 9 a.m. - 1:30 p.m.}\\[.05in]
{\bf July 19, 2000}
\end{center}

\vspace*{.3in}

\begin{enumerate}
\item %% IMO1
Two circles $\omega_1$ and $\omega_2$ intersect at $M$ and $N$. Line $\ell$
is tangent to the circles at $A$ and $B$, respectively, so that $M$ lies
closer to $\ell$ than $N$. Line $CD$, with $C$ on $\omega_1$ and $D$ on 
$\omega_2$, is parallel to $\ell$ and passes through $M$. Let lines
$AC$ and $BD$ meet at $E$; let lines $AN$ and $CD$ meet at $P$; and let lines
$BN$ and $CD$ meet at $Q$. Prove that $EP = EQ$.

\item %% IMO2
Let $a,b,c$ be positive real numbers such that $abc=1$. Prove that
\[
(a-1+1/b)(b-1+1/c)(c-1+1/a) \leq 1.
\]

\item %% IMO3
Let $n \geq 2$ be a positive integer. Initially, there are $n$ fleas on a
horizontal line, not all at the same point. For a positive real number
$\lambda$, define a \emph{move} as follows:
\begin{verse}
\noindent
  choose any two fleas, at points $A$ and $B$, with $A$ to the left of $B$;
let the flea at $A$ jump to the point $C$ on the line to the right of $B$
with $BC/AB = \lambda$.
\end{verse}
Determine all values of $\lambda$ such that, for any point $M$ on the line
and any initial positions of the $n$ fleas, there is a finite sequence of moves
that will take all the fleas to positions to the right of $M$.
\end{enumerate}

\pagebreak %% DAY 2
\begin{center}
${\bf 41}^{\mbox{\bf st}}$ {\bf International
Mathematical Olympiad} \\[.1in]
{\bf Taejon, Republic of Korea} \\ [.05in]
{\bf Day II \hspace{.25in} 9 a.m. - 1:30 p.m.}\\[.05in]
{\bf July 20, 2000}
\end{center}

\vspace*{.3in}

\begin{enumerate}
\setcounter{enumi}{3}
\item %% IMO4
A magician has one hundred cards numbered 1 to 100. He puts them into three
boxes, a red one, a white one and a blue one, so that each box contains
at least one card. A member of the audience selects two of the three boxes,
chooses one card from each and announces the sum of the numbers of the chosen
cards. Given this sum, the magician identifies the box from which no card has
been chosen. How many ways are there to put all the cards into the boxes so
that this trick always works? (Two ways are considered different if at least
one card is put into a different box.)

\item %% IMO5
Determine if there exists a positive integer $n$ such that $n$ has exactly
2000 prime divisors and $2^n+1$ is divisible by $n$.

\item %% IMO6
Let $\overline{AH_1}$, $\overline{BH_2}$, and $\overline{CH_3}$ be the
altitudes of an acute triangle $ABC$. The incircle $\omega$ of triangle
$ABC$ touches the sides $BC, CA$ and $AB$ at $T_1$, $T_2$ and $T_3$,
respectively. Consider the symmetric images of the lines $H_1H_2$,
$H_2H_3$, and $H_3H_1$ with respect to the lines $T_1T_2$, $T_2T_3$,
and $T_3T_1$. Prove that these images form a triangle whose vertices
lie on $\omega$.
\end{enumerate}

\end{document}
