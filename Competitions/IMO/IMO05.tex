\documentclass[12pt]{article}
\usepackage{amsfonts}

\pagestyle{empty}
\setlength{\oddsidemargin}{.15in}
\setlength{\evensidemargin}{.15in}
\setlength{\textwidth}{6in}
\setlength{\textheight}{9.25in}
\setlength{\topmargin}{.2in}
\setlength{\headheight}{0in}
\setlength{\headsep}{0in}
\setlength{\parskip}{20pt}
\setlength{\labelsep}{10pt}
\setlength{\parindent}{0pt}
\setlength{\medskipamount}{3ex}
\setlength{\smallskipamount}{1ex}
\def\th{^{\scriptstyle\mbox{th}}}
\def\ang{\angle}
\def\dg{^\circ}
\def\be{\begin{enumerate}}
\def\ee{\end{enumerate}}
\def\ii{\item}

\begin{document}

\begin{center}
${\bf 46\th}$ {\bf International Mathematical Olympiad} \\[.1in]
{\bf Merida, Mexico} \\ [.05in]
{\bf Day I}\\[.05in]
{\bf July 13, 2005}
\end{center}

\vspace*{.3in}

\begin{enumerate}
\item
Six points are chosen on the sides of an equilateral triangle $ABC$:
$A_1$, $A_2$ on $BC$; $B_1$, $B_2$ on $CA$; $C_1$, $C_2$ on $AB$.
These points are the vertices of a convex hexagon $A_1 A_2 B_1 B_2 C_1 C_2$
with equal side lengths. Prove that the lines $A_1 B_2$, $B_1 C_2$ and 
$C_1 A_2$ are concurrent.

\item
Let $a_1$, $a_2$, \dots be a sequence of integers with infinitely many 
positive terms and infinitely many negative terms. Suppose that for each 
positive integer $n$, the numbers $a_1$, $a_2$, \dots, $a_n$ leave $n$ 
different remainders on division by $n$. Prove that each integer occurs 
exactly once in the sequence.

\item
Let $x$, $y$ and $z$ be positive real numbers such that $xyz\ge 1$.
Prove that
\[
\frac{x^5 - x^2}{x^5+y^2+z^2} + 
\frac{y^5 - y^2}{y^5+z^2+x^2} + 
\frac{z^5 - z^2}{z^5+x^2+y^2} \ge 0.
\]
\end{enumerate}

\pagebreak %% DAY 2
\begin{center}
${\bf 46\th}$ {\bf International Mathematical Olympiad} \\[.1in]
{\bf Merida, Mexico} \\ [.05in]
{\bf Day II}\\[.05in]
{\bf July 14, 2005}
\end{center}

\vspace*{.3in}

\begin{enumerate}
\setcounter{enumi}{3}
\item
Consider the sequence $a_1$, $a_2$, \dots defined by
\[
a_n = 2^n + 3^n + 6^n - 1\quad (n=1, 2, \dots).
\]
Determine all positive integers that are relatively prime to every term of 
the sequence.

\item
Let $ABCD$ be a given convex quadrilateral with sides $BC$ and $AD$ equal 
in length and not parallel. Let $E$ and $F$ be interior points of the 
sides $BC$ and $AD$ respectively such that $BE=DF$. The lines $AC$ and 
$BD$ meet at $P$, the lines $BD$ and $EF$ meet at $Q$, the lines $EF$ and 
$AC$ meet at $R$. Consider all the triangles $PQR$ as $E$ and $F$ vary. 
Show that the circumcircles of these triangles have a common point other 
than $P$.

\item
In a mathematical competition 6 problems were posed to the contestants. 
Each pair of problems was solved by more than $\frac{2}{5}$ of the 
contestants. Nobody solved all 6 problems. Show that there were at least 2 
contestants who each solved exactly 5 problems.
\end{enumerate}

\end{document}
