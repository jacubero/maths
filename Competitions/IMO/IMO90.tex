\documentclass[12pt]{article}
\usepackage{amsfonts}

\pagestyle{empty}
\setlength{\oddsidemargin}{.15in}
\setlength{\evensidemargin}{.15in}
\setlength{\textwidth}{6in}
\setlength{\textheight}{9.25in}
\setlength{\topmargin}{.2in}
\setlength{\headheight}{0in}
\setlength{\headsep}{0in}
\setlength{\parskip}{20pt}
\setlength{\labelsep}{10pt}
\setlength{\parindent}{0pt}
\setlength{\medskipamount}{3ex}
\setlength{\smallskipamount}{1ex}

\begin{document}
\begin{center}
${\bf 31}^{\mbox{\bf st}}$ {\bf International
Mathematical Olympiad} \\[.1in]
{\bf Beijing, China} \\ [.05in]
{\bf Day I}\\[.05in]
{\bf July 12, 1990}
\end{center}

\vspace*{.3in}

\begin{enumerate}
\item Chords $AB$ and $CD$ of a circle intersect at a point $E$ inside the
circle.  Let $M$ be an interior point of the segment $EB$.  The tangent line at
$E$ to the circle through $D$, $E$, and $M$ intersects the lines $BC$ and $AC$
at $F$ and $G$, respectively.  If
$$\frac{AM}{AB} = t,$$
find
$$\frac{EG}{EF}$$
in terms of $t$.

\item
Let $n \geq 3$ and consider a set $E$ of $2n - 1$ distinct points on a circle.
Suppose that exactly $k$ of these points are to be colored black.  Such a
coloring is ``good'' if there is at least one pair of black points such that
the interior of one of the arcs between them contains exactly $n$ points from
$E$.  Find the smallest value of $k$ so that every such coloring of $k$ points
of $E$ is good.

\item
Determine all integers $n > 1$ such that 
$$\frac{2^n + 1}{n^2}$$
is an integer.

\end{enumerate}

\pagebreak %% DAY 2
\begin{center}
${\bf 31}^{\mbox{\bf st}}$ {\bf International
Mathematical Olympiad} \\[.1in]
{\bf Beijing, China} \\ [.05in]
{\bf Day II}\\[.05in]
{\bf July 13, 1990}
\end{center}

\vspace*{.3in}

\begin{enumerate}
\setcounter{enumi}{3}
\item
Let ${\mathbb Q}^+$ be the set of positive rational numbers.  Construct a
function $f : {\mathbb Q}^+ \rightarrow {\mathbb Q}^+$ such that
$$f(xf(y)) = \frac{f(x)}{y}$$
for all $x$, $y$ in ${\mathbb Q}^+$.

\item
Given an initial integer $n_0 > 1$, two players, ${\mathcal A}$ and ${\mathcal
B}$, choose integers $n_1$, $n_2$, $n_3$, \ldots alternately according to the
following rules:

Knowing $n_{2k}$, ${\mathcal A}$ chooses any integer $n_{2k+1}$ such that
$$n_{2k} \leq n_{2k+1} \leq n_{2k}^2.$$

Knowing $n_{2k+1}$, ${\mathcal B}$ chooses any integer $n_{2k+2}$ such that
$$\frac{n_{2k+1}}{n_{2k+2}}$$
is a prime raised to a positive integer power.

Player ${\mathcal A}$ wins the game by choosing the number 1990; player
${\mathcal B}$ wins by choosing the number 1.  For which $n_0$ does:
  \begin{enumerate}
  \item[(a)] ${\mathcal A}$ have a winning strategy?
  \item[(b)] ${\mathcal B}$ have a winning strategy?
  \item[(c)] Neither player have a winning strategy?
  \end{enumerate}

\item
Prove that there exists a convex 1990-gon with the following two properties:
  \begin{enumerate}
  \item[(a)] All angles are equal.
  \item[(b)] The lengths of the 1990 sides are the numbers $1^2$, $2^2$, $3^2$,
  \ldots, $1990^2$ in some order.
  \end{enumerate}
\end{enumerate}
\end{document}
