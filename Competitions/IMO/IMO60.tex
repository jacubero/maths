%% This document created by Scientific Notebook (R) Version 3.0


\documentclass[12pt,thmsa]{article}
%%%%%%%%%%%%%%%%%%%%%%%%%%%%%%%%%%%%%%%%%%%%%%%%%%%%%%%%%%%%%%%%%%%%%%%%%%%%%%%%%%%%%%%%%%%%%%%%%%%%%%%%%%%%%%%%%%%%%%%%%%%%
\usepackage{sw20jart}

%TCIDATA{TCIstyle=article/art4.lat,jart,sw20jart}

%TCIDATA{<META NAME="GraphicsSave" CONTENT="32">}
%TCIDATA{Created=Mon Aug 19 14:52:24 1996}
%TCIDATA{LastRevised=Mon Feb 10 12:02:37 1997}
%TCIDATA{Language=American English}
%TCIDATA{CSTFile=Lab Report.cst}
%TCIDATA{PageSetup=72,72,72,72,0}
%TCIDATA{AllPages=
%F=36,\PARA{038<p type="texpara" tag="Body Text" >\hfill \thepage}
%}


\input{tcilatex}
\begin{document}


\section{Second International Olympiad, 1960}

\subsection{1960/1.}

Determine all three-digit numbers $N$ having the property that $N$ is
divisible by $11,$ and $N/11$ is equal to the sum of the squares of the
digits of $N.$

\subsection{1960/2.}

For what values of the variable $x$ does the following inequality hold:
\[
\frac{4x^{2}}{(1-\sqrt{1+2x})^{2}}<2x+9?
\]

\subsection{1960/3.}

In a given right triangle $ABC,$ the hypotenuse $BC,$ of length $a,$ is
divided into $n$ equal parts ($n$ an odd integer). Let $\alpha $ be the
acute angle subtending, from $A,$ that segment which contains the midpoint
of the hypotenuse. Let $h$ be the length of the altitude to the hypotenuse
of the triangle. Prove:
\[
\tan \alpha =\frac{4nh}{(n^{2}-1)a}.
\]

\subsection{1960/4.}

Construct triangle $ABC,$ given $h_{a},h_{b}$ (the altitudes from $A$ and $B$%
) and $m_{a}$, the median from vertex $A.$

\subsection{1960/5.}

Consider the cube $ABCDA^{\prime }B^{\prime }C^{\prime }D^{\prime }$ (with
face $ABCD$ directly above face $A^{\prime }B^{\prime }C^{\prime }D^{\prime }
$).

(a) Find the locus of the midpoints of segments $XY,$ where $X$ is any point
of $AC$ and $Y$ is any point of $B^{\prime }D^{\prime }.$

(b) Find the locus of points $Z$ which lie on the segments $XY$ of part (a)
with $ZY=2XZ.$

\subsection{1960/6. }

Consider a cone of revolution with an inscribed sphere tangent to the base
of the cone. A cylinder is circumscribed about this sphere so

that one of its bases lies in the base of the cone. Let $V_{1}$ be the
volume of the cone and $V_{2}$ the volume of the cylinder.

(a) Prove that $V_{1}\neq V_{2}$. 

(b) Find the smallest number $k$ for which $V_{1}=kV_{2}$, for this case,
construct the angle subtended by a diameter of the base of the cone at the
vertex of the cone.

\subsection{1960/7.}

An isosceles trapezoid with bases $a$ and $c$ and altitude $h$ is given.

(a) On the axis of symmetry of this trapezoid, find all points $P$ such that
both legs of the trapezoid subtend right angles at $P.$

(b) Calculate the distance of $P$ from either base. 

(c) Determine under what conditions such points $P$ actually exist. (Discuss
various cases that might arise.)

\end{document}
