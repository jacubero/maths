%% This document created by Scientific Notebook (R) Version 3.0


\documentclass[12pt,thmsa]{article}
%%%%%%%%%%%%%%%%%%%%%%%%%%%%%%%%%%%%%%%%%%%%%%%%%%%%%%%%%%%%%%%%%%%%%%%%%%%%%%%%%%%%%%%%%%%%%%%%%%%%%%%%%%%%%%%%%%%%%%%%%%%%
\usepackage{sw20jart}

%TCIDATA{TCIstyle=article/art4.lat,jart,sw20jart}

%TCIDATA{<META NAME="GraphicsSave" CONTENT="32">}
%TCIDATA{Created=Mon Aug 19 14:52:24 1996}
%TCIDATA{LastRevised=Mon Feb 10 15:14:46 1997}
%TCIDATA{Language=American English}
%TCIDATA{CSTFile=Lab Report.cst}
%TCIDATA{PageSetup=72,72,72,72,0}
%TCIDATA{AllPages=
%F=36,\PARA{038<p type="texpara" tag="Body Text" >\hfill \thepage}
%}


\input{tcilatex}
\begin{document}


\section{Sixth International Olympiad, 1964}

\subsection{1964/1.}

(a) Find all positive integers $n$ for which $2^{n}-1$ is divisible by $7.$

(b) Prove that there is no positive integer $n$ for which $2^{n}+1$  is
divisible by $7.$

\subsection{1964/2.}

Suppose $a,b,c$ are the sides of a triangle. Prove that
\[
a^{2}(b+c-a)+b^{2}(c+a-b)+c^{2}(a+b-c)\leq 3abc.
\]

\subsection{1964/3.}

A circle is inscribed in triangle $ABC$ with sides $a,b,c.$ Tangents to the
circle parallel to the sides of the triangle are constructed. Each of these
tangents cuts off a triangle from $\Delta ABC.$ In each of these triangles,
a circle is inscribed. Find the sum of the areas of all four inscribed
circles (in terms of $a,b,c$).

\subsection{1964/4.}

Seventeen people correspond by mail with one another - each one with all the
rest. In their letters only three different topics are discussed. Each pair
of correspondents deals with only one of these topics. Prove that there are
at least three people who write to each other about the same topic.

\subsection{1964/5.}

Suppose five points in a plane are situated so that no two of the straight
lines joining them are parallel, perpendicular, or coincident. From each
point perpendiculars are drawn to all the lines joining the other four
points. Determine the maximum number of intersections that these
perpendiculars can have.

\subsection{1964/6.}

In tetrahedron $ABCD,$ vertex $D$ is connected with $D_{0}$ the centroid of $%
\Delta ABC.$ Lines parallel to $DD_{0}$ are drawn through $A,B$ and $C.$
These lines intersect the planes $BCD,CAD$ and $ABD$ in points $A_{1},B_{1}$
and $C_{1}$, respectively. Prove that the volume of $ABCD$ is one third the
volume of $A_{1}B_{1}C_{1}D_{0}.$ Is the result true if point $D_{0}$ is
selected anywhere within $\Delta ABC?$

\end{document}
