%% This document created by Scientific Notebook (R) Version 3.0


\documentclass[12pt,thmsa]{article}
%%%%%%%%%%%%%%%%%%%%%%%%%%%%%%%%%%%%%%%%%%%%%%%%%%%%%%%%%%%%%%%%%%%%%%%%%%%%%%%%%%%%%%%%%%%%%%%%%%%%%%%%%%%%%%%%%%%%%%%%%%%%
\usepackage{sw20jart}

%TCIDATA{TCIstyle=article/art4.lat,jart,sw20jart}

%TCIDATA{<META NAME="GraphicsSave" CONTENT="32">}
%TCIDATA{Created=Mon Aug 19 14:52:24 1996}
%TCIDATA{LastRevised=Mon Feb 10 10:59:06 1997}
%TCIDATA{CSTFile=Lab Report.cst}
%TCIDATA{PageSetup=72,72,72,72,0}
%TCIDATA{AllPages=
%F=36,\PARA{038<p type="texpara" tag="Body Text" >\hfill \thepage}
%}


\input{tcilatex}
\begin{document}


\section{Twentieth International Olympiad, 1978}

1978/1. $m$ and $n$ are natural numbers with $1\le m<n.$ In their decimal
representations, the last three digits of $1978^{m}$ are equal,
respectively, to the last three digits of $1978^{n}$. Find $m$ and $n$ such
that $m+n$ has its least value.

1978/2. $P$ is a given point inside a given sphere. Three mutually
perpendicular rays from $P$ intersect the sphere at points $U,V,$ and $W;Q$
denotes the vertex diagonally opposite to $P$ in the parallelepiped
determined by $PU,PV,$ and $PW.$ Find the locus of $Q$ for all such triads
of rays from $P$

1978/3. The set of all positive integers is the union of two disjoint
subsets $\{f(1),f(2),...,f(n),...\},\{g(1),g(2),...,g(n),...\},$ where 
\[
f(1)<f(2)<\cdots <f(n)<\cdots , 
\]
\[
g(1)<g(2)<\cdots <g(n)<\cdots , 
\]
and 
\[
g(n)=f(f(n))+1\text{for all }n\ge 1. 
\]
Determine $f(240).$

1978/4. In triangle $ABC,AB=AC.$ A circle is tangent internally to the
circumcircle of triangle $ABC$ and also to sides $AB,AC$ at $P,Q,$
respectively. Prove that the midpoint of segment $PQ$ is the center of the
incircle of triangle $ABC.$

1978/5. Let $\{a_{k}\}(k=1,2,3,...,n,...)$ be a sequence of distinct
positive integers. Prove that for all natural numbers $n,$%
\[
\sum_{k=1}^{n}\frac{a_{k}}{k^{2}}\ge \sum_{k=1}^{n}\frac{1}{k}. 
\]

1978/6. An international society has its members from six different
countries. The list of members contains $1978$ names, numbered $1,2,...,1978.
$ Prove that there is at least one member whose number is the sum of the
numbers of two members from his own country, or twice as large as the number
of one member from his own country.

\end{document}
