%% This document created by Scientific Notebook (R) Version 3.0


\documentclass[12pt,thmsa]{article}
%%%%%%%%%%%%%%%%%%%%%%%%%%%%%%%%%%%%%%%%%%%%%%%%%%%%%%%%%%%%%%%%%%%%%%%%%%%%%%%%%%%%%%%%%%%%%%%%%%%%%%%%%%%%%%%%%%%%%%%%%%%%
\usepackage{sw20jart}

%TCIDATA{TCIstyle=article/art4.lat,jart,sw20jart}

%TCIDATA{<META NAME="GraphicsSave" CONTENT="32">}
%TCIDATA{Created=Mon Aug 19 14:52:24 1996}
%TCIDATA{LastRevised=Mon Feb 10 18:00:38 1997}
%TCIDATA{Language=American English}
%TCIDATA{CSTFile=Lab Report.cst}
%TCIDATA{PageSetup=72,72,72,72,0}
%TCIDATA{AllPages=
%F=36,\PARA{038<p type="texpara" tag="Body Text" >\hfill \thepage}
%}


\input{tcilatex}
\begin{document}


\section{Fourteenth International Olympiad, 1972}

\subsection{1972/1.}

Prove that from a set of ten distinct two-digit numbers (in the decimal
system), it is possible to select two disjoint subsets whose members have
the same sum.

\subsection{1972/2.}

Prove that if $n\geq 4,$ every quadrilateral that can be inscribed in a
circle can be dissected into $n$ quadrilaterals each of which is inscribable
in a circle.

\subsection{1972/3.}

Let $m$ and $n$ be arbitrary non-negative integers. Prove that
\[
\frac{(2m)!(2n)!}{m\prime n!(m+n)!}
\]
is an integer. $(0!=1.)$

\subsection{1972/4.}

Find all solutions $(x_{1},x_{2},x_{3},x_{4},x_{5})$ of the system of
inequalities

\begin{eqnarray*}
(x_{1}^{2}-x_{3}x_{5})(x_{2}^{2}-x_{3}x_{5}) &\leq &0 \\
(x_{2}^{2}-x_{4}x_{1})(x_{3}^{2}-x_{4}x_{1}) &\leq &0 \\
(x_{3}^{2}-x_{5}x_{2})(x_{4}^{2}-x_{5}x_{2}) &\leq &0 \\
(x_{4}^{2}-x_{1}x_{3})(x_{5}^{2}-x_{1}x_{3}) &\leq &0 \\
(x_{5}^{2}-x_{2}x_{4})(x_{1}^{2}-x_{2}x_{4}) &\leq &0
\end{eqnarray*}

where $x_{1},x_{2},x_{3},x_{4},x_{5}$ are positive real numbers.

\subsection{1972/5.}

Let $f$ and $g$ be real-valued functions defined for all real values of $x$
and $y,$ and satisfying the equation
\[
f(x+y)+f(x-y)=2f(x)g(y)
\]

for all $x,y.$ Prove that if $f(x)$ is not identically zero, and if  $\left|
f(x)\right| $ $\leq 1$ for all $x,$ then $\left| g(y)\right| \leq 1$ for all 
$y.$

\subsection{1972/6.}

Given four distinct parallel planes, prove that there exists a regular
tetrahedron with a vertex on each plane.

\end{document}
