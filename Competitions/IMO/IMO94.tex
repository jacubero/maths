% Process using LaTeX
%
\documentclass[12pt]{article}
\begin{document}

The 35th International Mathematical Olympiad (July 13-14, 1994, Hong Kong)

{\noindent \bf 1.}
Let $m$ and $n$ be positive integers. Let $a_1, a_2, \dots, a_m$ 
be distinct elements of $\{1, 2, \dots, n\}$ such that whenever $a_i + 
a_j \leq n$ for some $i, j$, $1 \leq i \leq j \leq m$, there exists $k$, 
$1 \leq k \leq m$, with $a_i + a_j = a_k$. Prove that
\[
\frac{a_1 + a_2 + \cdots + a_m}{m} \geq \frac{n+1}{2}.
\]

{\noindent \bf 2.}
$ABC$ is an isosceles triangle with $AB = AC$. Suppose that
\begin{enumerate}
	\item  $M$ is the midpoint of $BC$ and $O$ is the point on the line 
	$AM$ such that $OB$ is perpendicular to $AB$;

	\item  $Q$ is an arbitrary point on the segment $BC$ different from 
	$B$ and $C$;

	\item  $E$ lies on the line $AB$ and $F$ lies on the line $AC$ 
	such that $E$, $Q$, $F$ are distinct and collinear.
\end{enumerate}
Prove that $OQ$ is perpendicular to $EF$ if and only if $QE = QF$.

{\noindent \bf 3.}
For any positive integer $k$, let $f(k)$ be the number of 
elements in the set $\{k+1, k+2, \dots, 2k\}$ whose base 2 representation 
has precisely three 1s.
\begin{itemize}
	\item  (a) Prove that, for each positive integer $m$, there exists at 
	least one positive integer $k$ such that $f(k) = m$.

	\item  (b) Determine all positive integers $m$ for which there exists 
	exactly one $k$ with $f(k) = m$.
\end{itemize}

{\noindent \bf 4.}
Determine all ordered pairs $(m, n)$ of positive integers such that
\[
\frac{n^3 + 1}{mn - 1}
\]
is an integer.

{\noindent \bf 5.}
Let $S$ be the set of real numbers strictly greater than $-1$. 
Find all functions $f: S \to S$ satisfying the two conditions:
\begin{enumerate}
	\item  $f(x + f(y) + xf(y)) = y + f(x) + yf(x)$ for all $x$ and $y$ in $S$;

	\item  $\frac{f(x)}{x}$ is strictly increasing on each of the intervals 
	$-1 < x < 0$ and $0 < x$.
\end{enumerate}

{\noindent \bf 6.}
Show that there exists a set $A$ of positive integers with the 
following property: For any infinite set $S$ of primes there exist two 
positive integers $m \in A$ and $n \notin A$ each of which is a product 
of $k$ distinct elements of $S$ for some $k \geq 2$.

\end{document}
