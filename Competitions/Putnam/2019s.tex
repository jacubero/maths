\documentclass[amssymb,twocolumn,pra,10pt,aps]{revtex4-1}
\usepackage{mathptmx,amsmath,amsthm}

\newtheorem{lemma}{Lemma}
\newtheorem{cor}[lemma]{Corollary}
\newtheorem*{lemma*}{Lemma}
\newcommand{\FF}{\mathbb{F}}
\newcommand{\QQ}{\mathbb{Q}}
\newcommand{\RR}{\mathbb{R}}
\newcommand{\CC}{\mathbb{C}}
\newcommand{\ZZ}{\mathbb{Z}}
\DeclareMathOperator{\lcm}{lcm}
\DeclareMathOperator{\sgn}{sgn}
\DeclareMathOperator{\Trace}{Trace}
\newcommand{\ee}{\ell}

\begin{document}
\title{Solutions to the 80th William Lowell Putnam Mathematical Competition \\
    Saturday, December 7, 2019}
\author{Kiran Kedlaya and Lenny Ng}
\noaffiliation
\maketitle

\begin{itemize}
\item[A1]
The answer is all nonnegative integers not congruent to $3$ or $6 \pmod{9}$. Let $X$ denote the given expression;
we first show that we can make $X$ equal to each of the claimed values. Write $B=A+b$ and $C=A+c$, so that
\[
X = (b^2-bc+c^2)(3A+b+c).
\]
By taking $(b,c) = (0,1)$ or $(b,c) = (1,1)$, we obtain respectively $X = 3A+1$ and $X = 3A+2$; consequently, as $A$ varies, we achieve every nonnegative integer not divisible by 3. By taking $(b,c) = (1,2)$, we obtain $X = 9A+9$; consequently, as $A$ varies, we achieve every positive integer divisible by 9. We may also achieve $X=0$
by taking $(b,c) = (0,0)$.

In the other direction, $X$ is always nonnegative: either apply the arithmetic mean-geometric mean inequality, or write $b^2-bc+c^2 = (b - c/2)^2 + 3c^2/4$ to see that it is nonnegative.
It thus only remains to show that if $X$ is a multiple of $3$, then it is a multiple of $9$. Note that
$3A+b+c \equiv b+c \pmod{3}$ and $b^2-bc+c^2 \equiv (b+c)^2 \pmod{3}$; consequently, if $X$ is divisible by $3$,
then $b+c$ must be divisible by $3$, so each factor in $X = (b^2-bc+c^2)(3A+b+c)$ is divisible by $3$.
This proves the claim.

\noindent
\textbf{Remark.}
The factorization of $X$ used above can be written more symmetrically as
\[
X = (A+B+C)(A^2+B^2+C^2-AB-BC-CA).
\]
One interpretation of the factorization is that $X$ is the determinant of the circulant matrix
\[
\begin{pmatrix}
A & B & C \\
C & A & B \\
B & C & A
\end{pmatrix}
\]
which has the vector $(1,1,1)$ as an eigenvector (on either side) with eigenvalue $A+B+C$. The other eigenvalues are $A + \zeta B + \zeta^2 C$ where $\zeta$ is a primitive cube root of unity; in fact, $X$ is the norm form for the ring $\ZZ[T]/(T^3 - 1)$, from which it follows directly that the image of $X$ is closed under multiplication. (This is similar to the fact that the image of $A^2+B^2$, which is the norm form for the ring $\mathbb{Z}[i]$ of Gaussian integers, is closed under multiplication.)

One can also the unique factorization property of the ring $\ZZ[\zeta]$ of Eisenstein integers as follows.
The three factors of $X$ over $\ZZ[\zeta_3]$ are pairwise congruent modulo $1-\zeta_3$; consequently,
if $X$ is divisible by 3, then it is divisible by $(1-\zeta_3)^3 = -3\zeta_3(1-\zeta_3)$ and hence 
(because it is a rational integer) by $3^2$.

\item[A2]
\noindent \textbf{Solution 1.}
Let $M$ and $D$ denote the midpoint of $AB$ and the foot of the altitude from $C$ to $AB$, respectively, and let $r$ be the inradius of $\bigtriangleup ABC$. Since $C,G,M$ are collinear with $CM = 3GM$, the distance from $C$ to line $AB$ is $3$ times the distance from $G$ to $AB$, and the latter is $r$ since $IG \parallel AB$; hence the altitude $CD$ has length $3r$. By the double angle formula for tangent, $\frac{CD}{DB} = \tan\beta = \frac{3}{4}$, and so $DB = 4r$. Let $E$ be the point where the incircle meets $AB$; then $EB = r/\tan(\frac{\beta}{2}) = 3r$. It follows that $ED = r$, whence the incircle is tangent to the altitude $CD$. This implies that $D=A$, $ABC$ is a right triangle, and $\alpha = \frac{\pi}{2}$.

\noindent
\textbf{Remark.}
One can obtain a similar solution by fixing a coordinate system with $B$ at the origin and $A$ on the positive 
$x$-axis. Since $\tan \frac{\beta}{2} = \frac{1}{3}$, we may assume without loss of generality that
$I = (3,1)$. Then $C$ lies on the intersection of the line $y=3$ (because $CD = 3r$ as above) 
with the line $y = \frac{3}{4} x$ (because $\tan \beta = \frac{3}{4}$ as above), forcing $C = (4,3)$ and so forth.

\noindent \textbf{Solution 2.}
Let $a,b,c$ be the lengths of $BC,CA,AB$, respectively.
Let $r$, $s$, and $K$ denote the inradius, semiperimeter, and area of $\triangle ABC$. 
By Heron's Formula,
\[
r^2s^2 = K^2 = s(s-a)(s-b)(s-c).
\]

If $IG$ is parallel to $AB$, then 
\[
\frac{1}{2} rc = 
\mathrm{area}(\triangle ABI) = \mathrm{area}(\triangle ABG) = \frac{1}{3} K = \frac{1}{3} rs
\]
and so $c = \frac{a+b}{2}$. Since $s = \frac{3(a+b)}{4}$ and $s-c = \frac{a+b}{4}$, we have 
$3r^2 = (s-a)(s-b)$. Let $E$ be the point at which the incircle meets $AB$; then $s-b = EB = r/\tan(\frac{\beta}{2})$ and $s-a = EA = r/\tan(\frac{\alpha}{2})$. It follows that $\tan(\frac{\alpha}{2})\tan(\frac{\beta}{2}) = \frac{1}{3}$ and so $\tan(\frac{\alpha}{2}) = 1$. This implies that $\alpha = \frac{\pi}{2}$.

\noindent
\textbf{Remark.}
The equality $c = \frac{a+b}{2}$ can also be derived from the vector representations
\[
G = \frac{A+B+C}{3}, \qquad I = \frac{aA+bB+cC}{a+b+c}.
\]


\noindent
\textbf{Solution 3.}
(by Catalin Zara)
It is straightforward to check that a right triangle with $AC = 3, AB = 4, BC = 5$ works. For example,
in a coordinate system with $A = (0,0), B = (4,0), C = (0,3)$, we have
\[
G = \left(\frac{4}{3}, 1 \right), \qquad 
I = (1, 1)
\]
and for $D = (1,0)$, 
\[
\tan \frac{\beta}{2} = \frac{ID}{BD} = \frac{1}{3}.
\]
It thus suffices to suggest that this example is unique up to similarity.

Let $C'$ be the foot of the angle bisector at $C$. Then 
\[
\frac{CI}{IC'} = \frac{CA + CB}{AB}
\] 
and so $IG$ is parallel to $AB$ if and only if $CA + CB = 2AB$. We may assume without loss of generality that $A$ and $B$ are fixed, in which case this condition restricts $C$ to an ellipse with foci at $A$ and $B$.
Since the angle $\beta$ is also fixed, up to symmetry 
$C$ is further restricted to a half-line starting at $B$; this intersects the ellipse in a unique point.

\noindent 
\textbf{Remark.}
Given that $CA + CB = 2AB$, one can also recover the ratio of side lengths using the law of cosines.

\item[A3]
The answer is $M = 2019^{-1/2019}$. For any choices of $b_0,\ldots,b_{2019}$ as specified, AM-GM gives
\[
\mu \geq |z_1\cdots z_{2019}|^{1/2019} = |b_0/b_{2019}|^{1/2019} \geq 2019^{-1/2019}.
\]
To see that this is best possible, consider $b_0,\ldots,b_{2019}$ given by $b_k = 2019^{k/2019}$ for all $k$. Then 
\[
P(z/2019^{1/2019}) = \sum_{k=0}^{2019} z^k = \frac{z^{2020}-1}{z-1}
\]
has all of its roots on the unit circle. It follows that all of the roots of $P(z)$ have modulus $2019^{-1/2019}$, and so $\mu = 2019^{-1/2019}$ in this case.


\item[A4]
The answer is no. Let $g :\thinspace \mathbb{R} \to \mathbb{R}$ be any continuous function with $g(t+2) = g(t)$ for all $t$ and $\int_0^2 g(t)\,dt = 0$ (for instance, $g(t) = \sin(\pi t)$). Define $f(x,y,z) = g(z)$. We claim that for any sphere $S$ of radius $1$, $\iint_S f\,dS = 0$.


Indeed, let $S$ be the unit sphere centered at $(x_0,y_0,z_0)$. We can parametrize $S$ by $S(\phi,\theta) = (x_0,y_0,z_0)+(\sin\phi\cos\theta,
\sin\phi\sin\theta,\cos\phi)$ for $\phi \in [0,\pi]$ and $\theta \in [0,2\pi]$. Then we have

\begin{align*}
\iint_S f(x,y,z)\,dS &= \int_0^\pi \int_0^{2\pi} f(S(\phi,\theta))\left\|\frac{\partial S}{\partial \phi} \times \frac{\partial S}{\partial \theta}\right\|\,d\theta\,d\phi \\
&=  \int_0^\pi \int_0^{2\pi} g(z_0+\cos\phi) \sin\phi\,d\theta\,d\phi \\
&= 2\pi \int_{-1}^1 g(z_0+t)\,dt,
\end{align*}

where we have used the substitution $t = \cos\phi$; but this last integral is $0$ for any $z_0$ by construction.

\noindent
\textbf{Remark.}
The solution recovers the famous observation of Archimedes that the surface area of a spherical cap is linear in the height of the cap. In place of spherical coordinates, one may also compute $\iint_S f(x,y,z)\,dS$ by computing the integral over a ball of radius $r$, then computing the derivative with respect to $r$ and evaluating at $r=1$.

Noam Elkies points out that a similar result holds in $\mathbb{R}^n$ for any $n$. Also, there exist nonzero continuous functions on $\mathbb{R}^n$ whose integral over any unit ball vanishes; this implies certain negative results about image reconstruction.

\item[A5]
The answer is $\frac{p-1}{2}$. 
Define the operator $D = x \frac{d}{dx}$, where $\frac{d}{dx}$ indicates formal differentiation of polynomials.
For $n$ as in the problem statement, we have $q(x) = (x-1)^n r(x)$ for some polynomial $r(x)$ in $\mathbb{F}_p$ not divisible by $x-1$. For $m=0,\dots,n$, by the product rule we have
\[
(D^m q)(x) \equiv n^m x^m (x-1)^{n-m} r(x) \pmod{(x-1)^{n-m+1}}.
\]
Since $r(1) \neq 0$ and $n \not\equiv 0 \pmod{p}$ (because $n \leq \deg(q) = p-1$), we may identify $n$ as the smallest nonnegative integer for which $(D^n q)(1) \neq 0$.

Now note that $q = D^{(p-1)/2} s$ for
\[
s(x) = 1 + x + \cdots + x^{p-1} = \frac{x^p-1}{x-1} = (x-1)^{p-1}
\]
since $(x-1)^p = x^p-1$ in $\mathbb{F}_p[x]$.
By the same logic as above, $(D^n s)(1) = 0$ for $n=0,\dots,p-2$ but not for $n=p-1$.
This implies the claimed result.

\noindent
\textbf{Remark.}
One may also finish by checking directly that 
for any positive integer $m$,
\[
\sum_{k=1}^{p-1} k^m \equiv \begin{cases} -1 \pmod{p} & \mbox{if $(p-1)|m$} \\
0 \pmod{p} & \mbox{otherwise.}
\end{cases}
\]
If $(p-1) | m$, then $k^m \equiv 1 \pmod{p}$ by the little Fermat theorem, and so the sum is congruent
to $p-1 \equiv -1 \pmod{p}$. Otherwise, for any primitive root $\ell$ mod $p$, multiplying the sum by $\ell^m$ permutes the terms modulo $p$ and hence does not change the sum modulo $p$; since $\ell^n \not\equiv 1 \pmod{p}$, this is only possible if the sum is zero modulo $p$.


\item[A6]
\textbf{Solution 1.}
(by Harm Derksen)
We assume that $\limsup_{x \to 0^+} x^r |g''(x)| < \infty$
and deduce that $\lim_{x \to 0^+} g'(x) = 0$.
Note that
\[
\limsup_{x \to 0^+} x^r \sup\{| g''(\xi)|: \xi \in [x/2, x]\}
< \infty.
\]
Suppose for the moment that there exists a function $h$ on $(0,1)$ which is positive, nondecreasing, and satisfies
\[
\lim_{x \to 0^+} \frac{g(x)}{h(x)} = \lim_{x \to 0^+} \frac{h(x)}{x^r} = 0.
\]
For some $c>0$, $h(x) < x^r < x$ for $x \in (0,c)$. By Taylor's theorem with remainder, we can find a function $\xi$ on $(0,c)$ such that
$\xi(x) \in [x-h(x),x]$ and
\[
g(x-h(x)) = g(x) - g'(x) h(x) + \frac{1}{2} g''(\xi(x)) h(x)^2.
\]
We can thus express $g'(x)$ as
\[
\frac{g(x)}{h(x)} + \frac{1}{2} x^r g''(\xi(x)) \frac{h(x)}{x^r}
-  \frac{g(x-h(x))}{h(x-h(x))} \frac{h(x-h(x))}{h(x)}.
\]
As $x \to 0^+$, $g(x)/h(x)$, $g(x-h(x))/h(x-h(x))$, and
$h(x)/x^r$ tend to 0, while $x^r g''(\xi(x))$ remains bounded
(because $\xi(x) \geq x-h(x) \geq x - x^r \geq x/2$ for $x$ small)
and $h(x-h(x))/h(x)$ is bounded in $(0,1]$.
Hence $\lim_{x \to 0^+} g'(x) = 0$ as desired.

It thus only remains to produce a function $h$ with the desired properties; this amounts to ``inserting'' a function between $g(x)$ and $x^r$ while taking care to ensure the positive and nondecreasing properties.
One of many options is $h(x) = x^r \sqrt{f(x)}$ where
\[
f(x) = \sup\{|z^{-r} g(z)|: z \in (0,x)\},
\]
so that
\[
\frac{h(x)}{x^r} = \sqrt{f(x)}, \qquad \frac{g(x)}{h(x)} = \sqrt{f(x)} x^{-r} g(x).
\]

\noindent
\textbf{Solution 2.}
We argue by contradiction. Assume that $\limsup_{x\to 0^+} x^r|g''(x)|<\infty$, so that there is an $M$ such that $|g''(x)| < M x^{-r}$ for all $x$; and that $\lim_{x\to 0^+} g'(x) \neq 0$, so that there is an $\epsilon_0>0$ and a sequence $x_n\to 0$ with $|g'(x_n)| > \epsilon_0$ for all $n$.

Now let $\epsilon>0$ be arbitrary. Since $\lim_{x\to 0^+} g(x) x^{-r} = 0$, there is a $\delta>0$ for which $|g(x)|<\epsilon x^r$ for all $x<\delta$.
Choose $n$ sufficiently large that $\frac{\epsilon_0 x_n^r}{2M}<x_n$ and $x_n < \delta/2$; then $x_n+\frac{\epsilon_0 x_n^r}{2M} < 2 x_n < \delta$. In addition,
we have $|g'(x)| > \epsilon_0/2$ for all $x\in [x_n,x_n+\frac{\epsilon_0 x_n^r}{2M}]$ since $|g'(x_n)| > \epsilon_0$ and $|g''(x)| < Mx^{-r} \leq M x_n^{-r}$ in this range. It follows that
\begin{align*}
\frac{\epsilon_0^2}{2} \frac{x_n^r}{2M} &<
|g(x_n+\frac{\epsilon_0 x_n^r}{2M}) - g(x_n)| \\
&\leq |g(x_n+\frac{\epsilon_0 x_n^r}{2M})|+|g(x_n)| \\
&< \epsilon \left((x_n+\frac{\epsilon_0 x_n^r}{2M})^r+x_n^r\right) \\
&< \epsilon(1+2^r)x_n^r,
\end{align*}
whence $4M(1+2^r)\epsilon > \epsilon_0^2$. Since $\epsilon>0$ is arbitrary and $M,r,\epsilon_0$ are fixed, this gives the desired contradiction.


\noindent
\textbf{Remark.}
Harm Derksen points out that the ``or'' in the problem need not be exclusive. For example, take
\[
g(x) = \begin{cases} x^5\sin(x^{-3}) & x \in (0,1] \\
0 & x = 0.
\end{cases}
\]
Then for $x \in (0,1)$,
\begin{align*}
g'(x) &= 5x^4\sin(x^{-3})-3x\cos(x^{-3}) \\
g''(x) &=(20x^3-9x^{-3})\sin(x^{-3})-18\cos(x^{-3}).
\end{align*}
For $r=2$, $\lim_{x\to 0^+}x^{-r}g(x)=\lim_{x\to 0^+}x^3\sin(x^{-3})=0$, $\lim_{x\to 0^+}g'(x)=0$ and
$x^rg''(x)=(20x^5-9x^{-1})\sin(x^{-3})-18x^2\cos(x^{-3})$ is unbounded as $x\to 0^+$.
(Note that $g'(x)$ is not differentiable at $x=0$.)

\item[B1]
The answer is $5n+1$.

We first determine the set $P_n$. Let $Q_n$ be the set of points in $\mathbb{Z}^2$ of the form $(0, \pm 2^k)$ or $(\pm 2^k, 0)$ for some $k \leq n$. Let $R_n$ be the set of points in $\mathbb{Z}^2$ of the form $(\pm 2^k, \pm 2^k)$ for some $k \leq n$ (the two signs being chosen independently). 
We prove by induction on $n$ that
\[
P_n = \{(0,0)\} \cup Q_{\lfloor n/2 \rfloor} \cup R_{\lfloor (n-1)/2 \rfloor}.
\]
We take as base cases the straightforward computations
\begin{align*}
P_0 &= \{(0,0), (\pm 1, 0), (0, \pm 1)\} \\
P_1 &= P_0 \cup \{(\pm 1, \pm 1)\}.
\end{align*}
For $n \geq 2$, it is clear that $\{(0,0)\} \cup Q_{\lfloor n/2 \rfloor} \cup R_{\lfloor (n-1)/2 \rfloor} \subseteq P_n$, so it remains to prove the reverse inclusion. For $(x,y) \in P_n$, note that $x^2 + y^2 \equiv 0 \pmod{4}$;
since every perfect square is congruent to either 0 or 1 modulo 4, $x$ and $y$ must both be even. Consequently,
$(x/2, y/2) \in P_{n-2}$, so we may appeal to the induction hypothesis to conclude.

We next identify all of the squares with vertices in $P_n$. In the following discussion, let $(a,b)$
and $(c,d)$ be two opposite vertices of a square, so that the other two vertices are
\[
\left( \frac{a-b+c+d}{2}, \frac{a+b-c+d}{2} \right)
\]
and 
\[
\left( \frac{a+b+c-d}{2}, \frac{-a+b+c+d}{2} \right).
\]
\begin{itemize}
\item
Suppose that $(a,b) = (0,0)$. Then $(c,d)$ may be any element of $P_n$ not contained in $P_0$.
The number of such squares is $4n$.

\item
Suppose that $(a,b), (c,d) \in Q_k$ for some $k$. 
There is one such square with vertices 
\[
\{(0, 2^k), (0, 2^{-k}), (2^k, 0), (2^{-k}, 0)\}
\]
for $k = 0,\dots,\lfloor \frac{n}{2} \rfloor$, for a total of $\lfloor \frac{n}{2} \rfloor + 1$.
To show that there are no others, by symmetry it suffices to rule out the existence of a square with
opposite vertices $(a,0)$ and $(c,0)$ where $a > \left| c \right|$. 
The other two vertices of this square would be $((a+c)/2, (a-c)/2)$ and $((a+c)/2, (-a+c)/2)$.
These cannot belong to any $Q_k$, or be equal to $(0,0)$,
because $|a+c|, |a-c| \geq a - |c| > 0$ by the triangle inequality.
These also cannot belong to any $R_k$ because $(a + |c|)/2 > (a - |c|)/2$. 
(One can also phrase this argument in geometric terms.)

\item
Suppose that $(a,b), (c,d) \in R_k$ for some $k$.
There is one such square with vertices
\[
\{(2^k, 2^k), (2^k, -2^k), (-2^k, 2^k), (-2^k, -2^k)\}
\]
for $k=0,\dots, \lfloor \frac{n-1}{2} \rfloor$, for a total of $\lfloor \frac{n+1}{2} \rfloor$.
To show that there are no others, we may reduce to the previous case: rotating by an angle of $\frac{\pi}{4}$ and then rescaling by a factor of $\sqrt{2}$ would yield a square with two opposite vertices in some $Q_k$ not centered at $(0,0)$, which we have already ruled out.

\item
It remains to show that we cannot have $(a,b) \in Q_k$ and $(c,d) \in R_k$ for some $k$.
By symmetry, we may reduce to the case where $(a,b) = (0, 2^k)$ and $(c,d) = (2^\ell, \pm 2^\ell)$.
If $d>0$, then the third vertex $(2^{k-1}, 2^{k-1} + 2^\ell)$ is impossible.
If $d<0$, then the third vertex $(-2^{k-1}, 2^{k-1} - 2^\ell)$ is impossible.

\end{itemize}

Summing up, we obtain
\[
4n + \left\lfloor \frac{n}{2} \right\rfloor + 1 + \left\lfloor \frac{n+1}{2} \right\rfloor = 5n+1
\]
squares, proving the claim.

\noindent
\textbf{Remark.}
Given the computation of $P_n$, we can alternatively show that the number of squares with vertices in $P_n$ is $5n+1$ as follows. Since this is clearly true for $n=1$, it suffices to show that for $n \geq 2$, there are exactly $5$ squares with vertices in $P_n$, at least one of which is not in $P_{n-1}$. Note that the convex hull of $P_n$ is a square $S$ whose four vertices are the four points in $P_n \setminus P_{n-1}$. If $v$ is one of these points, then a square with a vertex at $v$ can only lie in $S$ if its two sides containing $v$ are in line with the two sides of $S$ containing $v$. It follows that there are exactly two squares with a vertex at $v$ and all vertices in $P_n$: the square corresponding to $S$ itself, and a square whose vertex diagonally opposite to $v$ is the origin. Taking the union over the four points in $P_n \setminus P_{n-1}$ gives a total of $5$ squares, as desired. 

\item[B2]
The answer is $\frac{8}{\pi^3}$.

\noindent
\textbf{Solution 1.}
By the double angle and sum-product identities for cosine, we have
\begin{align*}
2\cos^2\left(\frac{(k-1)\pi}{2n}\right) - 2\cos^2 \left(\frac{k\pi}{2n}\right) &= \cos\left(\frac{(k-1)\pi}{n}\right) - \cos\left(\frac{k\pi}{n}\right) \\
&= 2\sin\left(\frac{(2k-1)\pi}{2n}\right) \sin\left(\frac{\pi}{2n}\right),
\end{align*}
and it follows that the summand in $a_n$ can be written as
\[
\frac{1}{\sin\left(\frac{\pi}{2n}\right)} \left(-\frac{1}{\cos^2\left(\frac{(k-1)\pi}{2n}\right)}+\frac{1}{\cos^2\left(\frac{k\pi}{2n}\right)}\right).
\]
Thus the sum telescopes and we find that
\[
a_n = \frac{1}{\sin\left(\frac{\pi}{2n}\right)} \left(-1+\frac{1}{\cos^2\left(\frac{(n-1)\pi}{2n}\right)}\right) =
- \frac{1}{\sin\left(\frac{\pi}{2n}\right)}+ \frac{1}{\sin^3\left(\frac{\pi}{2n}\right)}.
\]
Finally, since $\lim_{x\to 0} \frac{\sin x}{x} = 1$, we have $\lim_{n\to\infty} \left( n\sin\frac{\pi}{2n} \right) = \frac{\pi}{2}$, and thus
$\lim_{n\to\infty} \frac{a_n}{n^3} = \frac{8}{\pi^3}$.

\noindent
\textbf{Solution 2.}
We first substitute $n-k$ for $k$ to obtain
\[
a_n=\sum_{k=1}^{n-1} \frac{\sin\left(\frac{(2k+1)\pi}{2n}\right)}{\sin^2\left(\frac{(k+1)\pi}{2n}\right)\sin^2\left(\frac{k\pi}{2n}\right)}.
\]
We then use the estimate
\[
\frac{\sin x}{x} = 1 + O(x^2) \qquad (x \in [0, \pi])
\]
to rewrite the summand as
\[
\frac{\left( \frac{(2k-1)\pi}{2n} \right)}{\left(\frac{(k+1)\pi}{2n}\right)^2 \left(\frac{k\pi}{2n}\right)^2} \left(1 + O\left( \frac{k^2}{n^2} \right) \right)
\]
which simplifies to
\[
\frac{8 (2k-1) n^3}{k^2 (k+1)^2 \pi^3} + O\left( \frac{n}{k} \right).
\]
Consequently,
\begin{align*}
\frac{a_n}{n^3} &= \sum_{k=1}^{n-1} \left( \frac{8 (2k-1)}{k^2 (k+1)^2 \pi^3} + O\left( \frac{1}{kn^2} \right) \right) \\
&= \frac{8}{\pi^3} \sum_{k=1}^{n-1} \frac{(2k-1)}{k^2 (k+1)^2} 
+ O \left( \frac{\log n}{n^2} \right). 
\end{align*}
Finally, note that
\[
\sum_{k=1}^{n-1} \frac{(2k-1)}{k^2 (k+1)^2} = 
\sum_{k=1}^{n-1} \left( \frac{1}{k^2} - \frac{1}{(k+1)^2}\right) = 1 - \frac{1}{n^2}
\]
converges to 1, and so $\lim_{n \to \infty} \frac{a_n}{n^3} = \frac{8}{\pi^3}$.

\item[B3]
\noindent
\textbf{Solution 1.}
We first note that $P$ corresponds to the linear transformation on $\mathbb{R}^n$ given by reflection in the hyperplane perpendicular to $u$: $P(u) = -u$, and for any $v$ with $\langle u,v\rangle = 0$, $P(v) = v$. In particular, $P$ is an orthogonal matrix of determinant $-1$.

We next claim that if $Q$ is an $n\times n$ orthogonal matrix that does not have $1$ as an eigenvalue, then $\det Q = (-1)^n$. To see this, recall that the roots of the characteristic polynomial $p(t) = \det(tI-Q)$ all lie on the unit circle in $\mathbb{C}$, and all non-real roots occur in conjugate pairs ($p(t)$ has real coefficients, and orthogonality implies that $p(t) = \pm t^n p(t^{-1})$). The product of each conjugate pair of roots is $1$; thus $\det Q = (-1)^k$ where $k$ is the multiplicity of $-1$ as a root of $p(t)$. Since $1$ is not a root and all other roots appear in conjugate pairs, $k$ and $n$ have the same parity, and so $\det Q = (-1)^n$.

Finally, if neither of the orthogonal matrices $Q$ nor $PQ$ has $1$ as an eigenvalue, then $\det Q = \det(PQ) = (-1)^n$, contradicting the fact that $\det P = -1$. The result follows.

\noindent
\textbf{Remark.}
It can be shown that any $n \times n$ orthogonal matrix $Q$ can be written as a product of at most $n$ hyperplane reflections (Householder matrices). If equality occurs, then $\det(Q) = (-1)^n$;
if equality does not occur, then $Q$ has $1$ as an eigenvalue.
Consequently, equality fails for one of $Q$ and $PQ$, and that matrix has $1$ as an eigenvalue.

Sucharit Sarkar suggests the following topological interpretation: an orthogonal matrix without 1 as an eigenvalue
induces a fixed-point-free map from the $(n-1)$-sphere to itself, and the degree of such a map must be $(-1)^n$.

\noindent
\textbf{Solution 2.}
This solution uses the (reverse) \emph{Cayley transform}: if $Q$ is an orthogonal matrix not having 1 as an eigenvalue, then
\[
A = (I-Q)(I+Q)^{-1}
\]
is a skew-symmetric matrix (that is, $A^T = -A$).

Suppose then that $Q$ does not have $1$ as an eigenvalue.
Let $V$ be the orthogonal complement of $u$ in $\mathbb{R}^n$. On one hand, for $v \in V$,
\[
(I-Q)^{-1} (I - QP) v = (I-Q)^{-1} (I-Q)v = v.
\]
On the other hand,
\[
(I-Q)^{-1} (I - QP) u = (I-Q)^{-1} (I+Q)u = Au
\]
and $\langle u, Au \rangle = \langle A^T u, u \rangle
= \langle -Au, u \rangle$, so $Au \in V$.
Put $w = (1-A)u$; then $(1-QP)w = 0$, so $QP$ has 1 as an eigenvalue, and the same for $PQ$ because $PQ$ and $QP$ have the same characteristic polynomial.

\noindent
\textbf{Remark.}
The \emph{Cayley transform} is the following construction: if $A$ is a skew-symmetric matrix,
then $I+A$ is invertible and
\[
Q = (I-A)(I+A)^{-1}
\]
is an orthogonal matrix.

\noindent
\textbf{Remark.}
(by Steven Klee)
A related argument is to compute $\det(PQ-I)$ using the \emph{matrix determinant lemma}:
if $A$ is an invertible $n \times n$ matrix and $v, w$ are $1 \times n$ column vectors, then
\[
\det(A + vw^T) = \det(A) (1 + w^T A^{-1} v).
\]
This reduces to the case $A = I$, in which case it again comes down to the fact that the product of two square matrices (in this case, obtained from $v$ and $w$ by padding with zeroes) retains the same characteristic polynomial when the factors are reversed.

\item[B4]
\noindent
\textbf{Solution 1.}
We compute that $m(f) = 2 \ln 2 - \frac{1}{2}$.
Label the given differential equations by (1) and (2). If we write, e.g., $x\frac{\partial}{\partial x}(1)$ for the result of differentiating (1) by $x$ and multiplying the resulting equation by $x$, then the combination
$x\frac{\partial}{\partial x}(1)+y\frac{\partial}{\partial y}(1)-(1)-(2)$ gives the equation
$2xyf_{xy} = xy\ln(xy)+xy$, whence $f_{xy} = \frac{1}{2} (\ln(x)+\ln(y)+1)$.

Now we observe that
\begin{align*}
\lefteqn{f(s+1,s+1)-f(s+1,s)-f(s,s+1)+f(s,s)} \\
 &= \int_s^{s+1} \int_s^{s+1} f_{xy}\,dy\,dx \\
&= \frac{1}{2} \int_s^{s+1} \int_s^{s+1} (\ln(x)+\ln(y)+1)\,dy\,dx \\
&= \frac{1}{2} + \int_s^{s+1} \ln(x)\,dx.
\end{align*}

Since $\ln(x)$ is increasing, $\int_s^{s+1} \ln(x)\,dx$ is an increasing function of $s$, and so it is minimized over $s \in [1,\infty)$ when $s=1$. We conclude that
\[
m(f) = \frac{1}{2} + \int_1^2 \ln(x)\,dx = 2 \ln 2-\frac{1}{2}
\]
independent of $f$.

\noindent
\textbf{Remark.}
The phrasing of the question suggests that solvers were not expected to prove that $\mathcal{F}$ is nonempty,
even though this is necessary to make the definition of $m(f)$ logically meaningful. Existence will be explicitly established in the next solution.

\noindent
\textbf{Solution 2.}
We first verify that 
\[
f(x,y) = \frac{1}{2}(xy\ln(xy)-xy)
\]
is an element of $\mathcal{F}$, by computing that
\begin{gather*}
xf_x = yf_y = \frac{1}{2} xy \ln(xy) \\
x^2 f_{xx} = y^2 f_{yy} = xy.
\end{gather*}
(See the following remark for motivation for this guess.)

We next show that the only elements of $\mathcal{F}$ are $f+a \ln(x/y) + b$ where $a,b$ are constants.
Suppose that $f+g$ is a second element of $\mathcal{F}$. As in the first solution, we deduce that $g_{xy} = 0$; this implies that
$g(x,y) = u(x) + v(y)$ for some twice continuously differentiable functions $u$ and $v$. We also have
$xg_x + yg_y = 0$, which now asserts that $xg_x = - yg_y$ is equal to some constant $a$. This yields that
$g = a \ln(x/y) + b$ as desired.

We next observe that 
\[
g(s+1,s+1)-g(s+1,s)-g(s,s+1)+g(s,s) = 0,
\]
so $m(f) = m(f+g)$. It thus remains to compute $m(f)$. To do this, we verify that
\[
f(s+1,s+1)-f(s+1,s)-f(s,s+1)+f(s,s)
\]
is nondecreasing in $s$ by computing its derivative to be $\ln (s+1) - \ln(s)$
(either directly or using the integral representation from the first solution).
We thus minimize by taking $s=1$ as in the first solution.

\noindent
\textbf{Remark.}
One way to make a correct guess for $f$ is to notice that the given equations are both symmetric in $x$ and $y$
and posit that $f$ should also be symmetric. Any symmetric function of $x$ and $y$ can be written in terms of the variables $u = x+y$ and $v = xy$, so in principle we could translate the equations into those variables and solve. However, before trying this, we observe that $xy$ appears explicitly in the equations, so it is reasonable to make a first guess of the form $f(x,y) = h(xy)$. 
For such a choice, we have
\[
x f_x + y f_y = 2xy h' = xy \ln(xy)
\]
which forces us to set $h(t)  = \frac{1}{2}(t \ln(t) - t)$.

\item[B5]
\noindent
\textbf{Solution 1.}
We prove that $(j,k) = (2019, 1010)$ is a valid solution.
More generally, let $p(x)$ be the polynomial of degree $N$ such that $p(2n+1) = F_{2n+1}$ for $0 \leq n \leq N$. We will show that $p(2N+3) = F_{2N+3}-F_{N+2}$. 

Define a sequence of polynomials $p_0(x),\ldots,p_N(x)$ by $p_0(x) = p(x)$ and $p_k(x) = p_{k-1}(x)-p_{k-1}(x+2)$ for $k \geq 1$. Then by induction on $k$, it is the case that $p_k(2n+1) = F_{2n+1+k}$ for $0 \leq n \leq N-k$, and also that $p_k$ has degree (at most) $N-k$ for $k \geq 1$. Thus $p_N(x) = F_{N+1}$ since $p_N(1) = F_{N+1}$ and $p_N$ is constant.


We now claim that for $0\leq k\leq N$, $p_{N-k}(2k+3) = \sum_{j=0}^k F_{N+1+j}$. We prove this again by induction on $k$: for the induction step, we have
\begin{align*}
p_{N-k}(2k+3) &= p_{N-k}(2k+1)+p_{N-k+1}(2k+1) \\
&= F_{N+1+k}+\sum_{j=0}^{k-1} F_{N+1+j}.
\end{align*}
Thus we have $p(2N+3) = p_0(2N+3) = \sum_{j=0}^N F_{N+1+j}$. 

Now one final induction shows that $\sum_{j=1}^m F_j = F_{m+2}-1$, and so $p(2N+3) = F_{2N+3}-F_{N+2}$, as claimed. In the case $N=1008$, we thus have $p(2019) = F_{2019} - F_{1010}$.

\noindent
\textbf{Solution 2.}
This solution uses the \emph{Lagrange interpolation formula}: given $x_0,\dots,x_n$ and $y_0,\dots,y_n$, the unique polynomial $P$ of degree at most $n$ satisfying $P(x_i) = y_i$ for $i=0,\dots,n$ is
\[
\sum_{i=0}^n P(x_i) \prod_{j \neq i} \frac{x-x_j}{x_i-x_j} =
\]
Write 
\[
F_n = \frac{1}{\sqrt{5}}(\alpha^n - \beta^{-n}), \qquad \alpha = \frac{1+\sqrt{5}}{2}, \beta = \frac{1-\sqrt{5}}{2}.
\]
For $\gamma \in \mathbb{R}$, let $p_\gamma(x)$ be the unique polynomial of degree at most 1008 satisfying
\[
p_1(2n+1) = \gamma^{2n+1}, p_2(2n+1) = \gamma^{2n+1} \, (n=0,\dots,1008);
\]
then $p(x) = \frac{1}{\sqrt{5}}(p_\alpha(x) - p_\beta(x))$.

By Lagrange interpolation,
\begin{align*}
p_\gamma(2019) &= \sum_{n=0}^{1008} \gamma^{2n+1} \prod_{0 \leq j \leq 1008, j \neq n} \frac{2019-(2j+1)}{(2n+1)-(2j+1)}\\
&= \sum_{n=0}^{1008} \gamma^{2n+1} \prod_{0 \leq j \leq 1008, j \neq n} \frac{1009-j}{n-j}\\
&= \sum_{n=0}^{1008} \gamma^{2n+1} (-1)^{1008-n} \binom{1009}{n} \\
&= -\gamma ((\gamma^2-1)^{1009} - (\gamma^2)^{1009}).
\end{align*}
For $\gamma \in \{\alpha, \beta\}$ we have $\gamma^2 = \gamma + 1$ and so
\[
p_\gamma(2019) = \gamma^{2019} - \gamma^{1010}.
\]
We thus deduce that $p(x) = F_{2019} - F_{1010}$ as claimed.

\noindent
\textbf{Remark.}
Karl Mahlburg suggests the following variant of this. As above, use Lagrange interpolation to write
\[
p(2019) = \sum_{j=0}^{1008} \binom{1009}{j} F_j;
\]
it will thus suffice to verify (by substiting $j \mapsto 1009-j$) that
\[
\sum_{j=0}^{1009} \binom{1009}{j} F_{j+1} = F_{2019}.
\]
This identity has the following combinatorial interpretation. Recall that $F_{n+1}$ counts the number of ways to tile a $1 \times n$ rectangle with $1 \times 1$ squares and $1 \times 2$ dominoes (see below). In any such tiling with $n = 2018$, let $j$ be the number of squares among the first 1009 tiles.
These can be ordered in $\binom{1009}{j}$ ways, and the remaining $2018 - j - 2(1009-j) = j$ squares can be
tiled in $F_{j+1}$ ways.

As an aside, this interpretation of $F_{n+1}$ is the oldest known interpretation of the Fibonacci sequence,
long predating Fibonacci himself. In ancient Sanskrit, syllables were classified as long or short, and a long syllable was considered to be twice as long as a short syllable; consequently, the number of syllable patterns of total length $n$ equals $F_{n+1}$.

\noindent
\textbf{Remark.}
It is not difficult to show that the solution $(j,k) = (2019, 2010)$ is unique (in positive integers). 
First, note that to have $F_j - F_k > 0$, we must have $k < j$.
If $j < 2019$, then
\[
F_{2019} - F_{1010} = F_{2018} + F_{2017} - F_{1010} > F_{j} > F_j - F_k.
\]
If $j > 2020$, then  
\[
F_j - F_k \geq F_j - F_{j-1} = F_{j-2} \geq F_{2019} > F_{2019} - F_{1010}.
\]
Since $j = 2019$ obviously forces $k = 1010$, the only other possible solution would be with $j = 2020$.
But then
\[
(F_j - F_k) - (F_{2019} - F_{1010})
= (F_{2018} - F_k) + F_{1010} 
\]
which is negative for $k=2019$ (it equals $F_{1010} - F_{2017}$)
and positive for $k \leq 2018$.

\item[B6]
Such a set exists for every $n$. To construct an example, define the function $f: \mathbb{Z}^n \to \mathbb{Z}/(2n+1) \mathbb{Z}$ by
\[
f(x_1,\dots,x_n) = x_1 + 2x_2 + \cdots + nx_n \pmod{2n+1},
\]
then let $S$ be the preimage of 0.

To check condition (1), note that if $p \in S$ and $q$ is a neighbor of $p$ differing only in coordinate $i$, then
\[
f(q) = f(p) \pm i \equiv \pm i \pmod{2n+1}
\]
and so $q \notin S$.

To check condition (2), note that if $p \in \mathbb{Z}^n$ is not in $S$, then there exists a unique choice of $i \in \{1,\dots,n\}$ such that $f(p)$ is congruent to one of $+i$ or $-i$ modulo $2n+1$. The unique neighbor $q$ of $p$ in $S$ is then obtained by either subtracting $1$ from, or adding $1$ to, the $i$-th coordinate of $p$.

\noindent
\textbf{Remark.}
According to Art of Problem Solving (thread c6h366290), this problem was a 1985 IMO submission from Czechoslovakia. For an application to steganography, see:
J. Fridrich and P. Lison\v{e}k, Grid colorings in steganography,
\textit{IEEE Transactions on Information Theory} \textbf{53} (2007), 1547--1549.

\end{itemize}
\end{document}



