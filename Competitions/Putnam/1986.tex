\documentclass[amssymb,twocolumn,pra,10pt,aps]{revtex4-1}
\usepackage{mathptmx,amsmath}

\begin{document}
\title{The Forty-Seventh Annual William Lowell Putnam Competition \\
    Saturday, December 6, 1986}
\maketitle

\begin{itemize}
\item[A--1]
Find, with explanation, the maximum value of $f(x)=x^3-3x$ on the
set of all real numbers $x$ satisfying $x^4+36\leq 13x^2$.

\item[A--2]
What is the units (i.e., rightmost) digit of
\[
\left\lfloor \frac{10^{20000}}{10^{100}+3}\right\rfloor ?
\]
%Here $\lfloor x \rfloor$ is the greatest integer less than or equal to
%$x$.

\item[A--3]
Evaluate $\sum_{n=0}^\infty \mathrm{Arccot}(n^2+n+1)$, where
$\mathrm{Arccot}\,t$ for $t \geq 0$ denotes the number $\theta$ in the
interval $0 < \theta \leq \pi/2$ with $\cot \theta = t$.

\item[A--4]
A \emph{transversal} of an $n\times n$ matrix $A$ consists of $n$
entries of $A$, no two in the same row or column. Let $f(n)$ be the
number of $n \times n$ matrices $A$ satisfying the following two
conditions:
\begin{enumerate}
\item[(a)] Each entry $\alpha_{i,j}$ of $A$ is in the set
$\{-1,0,1\}$.
\item[(b)] The sum of the $n$ entries of a transversal is the same for
all transversals of $A$.
\end{enumerate}
An example of such a matrix $A$ is
\[
A = \left( \begin{array}{ccc} -1 & 0 & -1 \\ 0 & 1 & 0 \\ 0 & 1 & 0
\end{array}
\right).
\]
Determine with proof a formula for $f(n)$ of the form
\[
f(n) = a_1 b_1^n + a_2 b_2^n + a_3 b_3^n + a_4,
\]
where the $a_i$'s and $b_i$'s are rational numbers.

\item[A--5]
Suppose $f_1(x), f_2(x), \dots, f_n(x)$ are functions of $n$ real
variables $x = (x_1, \dots, x_n)$ with continuous second-order partial
derivatives everywhere on $\mathbb{R}^n$. Suppose further that there are
constants $c_{ij}$ such that
\[
\frac{\partial f_i}{\partial x_j} - \frac{\partial f_j}{\partial x_i}
= c_{ij}
\]
for all $i$ and $j$, $1\leq i \leq n$, $1 \leq j \leq n$. Prove that
there is a function $g(x)$ on $\mathbb{R}^n$ such that $f_i + \partial
g/\partial x_i$ is linear for all $i$, $1 \leq i \leq n$. (A linear
function is one of the form
\[
a_0 + a_1 x_1 + a_2 x_2 + \cdots + a_n x_n.)
\]

\item[A--6]
Let $a_1, a_2, \dots, a_n$ be real numbers, and let $b_1, b_2, \dots,
b_n$ be distinct positive integers. Suppose that there is a polynomial
$f(x)$ satisfying the identity
\[
(1-x)^n f(x) = 1 + \sum_{i=1}^n a_i x^{b_i}.
\]
Find a simple expression (not involving any sums) for $f(1)$ in terms
of $b_1, b_2, \dots, b_n$ and $n$ (but independent of $a_1, a_2,
\dots, a_n$).

\item[B--1]
Inscribe a rectangle of base $b$ and height $h$ in a circle of radius
one, and inscribe an isosceles triangle in the region of the circle
cut off by one base of the rectangle (with that side as the base of
the triangle).
For what
value of $h$ do the rectangle and triangle have the same area?

\item[B--2]
Prove that there are only a finite number of possibilities for the
ordered triple $T=(x-y,y-z,z-x)$, where $x,y,z$ are complex numbers
satisfying the simultaneous equations
\[
x(x-1)+2yz = y(y-1)+2zx = z(z-1)+2xy,
\]
and list all such triples $T$.

\item[B--3]
Let $\Gamma$ consist of all polynomials in $x$ with integer
coefficients. For $f$ and $g$ in $\Gamma$ and $m$ a positive integer,
let $f \equiv g \pmod{m}$ mean that every coefficient of $f-g$ is an
integral multiple of $m$. Let $n$ and $p$ be positive integers with
$p$ prime. Given that $f,g,h,r$ and $s$ are in $\Gamma$ with
$rf+sg\equiv 1 \pmod{p}$ and $fg \equiv h \pmod{p}$, prove that there
exist $F$ and $G$ in $\Gamma$ with $F \equiv f \pmod{p}$, $G \equiv g
\pmod{p}$, and $FG \equiv h \pmod{p^n}$.

\item[B--4]
For a positive real number $r$, let $G(r)$ be the minimum value of $|r
- \sqrt{m^2+2n^2}|$ for all integers $m$ and $n$. Prove or disprove
the assertion that $\lim_{r\to \infty}G(r)$ exists and equals 0.

\item[B--5]
Let $f(x,y,z) = x^2+y^2+z^2+xyz$. Let $p(x,y,z), q(x,y,z)$, $r(x,y,z)$
be polynomials with real coefficients satisfying
\[
f(p(x,y,z), q(x,y,z), r(x,y,z)) = f(x,y,z).
\]
Prove or disprove the assertion that the sequence $p,q,r$ consists of
some permutation of $\pm x, \pm y, \pm z$, where the number of minus
signs is 0 or 2.

\item[B--6]
Suppose $A,B,C,D$ are $n \times n$ matrices with entries in a field
$F$, satisfying the conditions that $AB^T$ and $CD^T$ are symmetric and
$AD^T - BC^T = I$. Here $I$ is the $n \times n$ identity matrix, and
if $M$ is an $n \times n$ matrix, $M^T$ is its transpose. Prove that
$A^T D - C^T B = I$.

\end{itemize}

\end{document}
