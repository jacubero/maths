\documentclass[amssymb,twocolumn,pra,10pt,aps]{revtex4-1}
\usepackage{mathptmx,amsmath}

\begin{document}
\title{The Forty-Eighth Annual William Lowell Putnam Competition \\
    Saturday, December 5, 1987 \vspace{-2\baselineskip}}
\maketitle

\begin{itemize}
\item[A--1]
Curves $A,B,C$ and $D$ are defined in the plane as follows:
\begin{align*}
A &= \left\{ (x,y): x^2-y^2 = \frac{x}{x^2+y^2} \right\}, \\
B &= \left\{ (x,y): 2xy + \frac{y}{x^2+y^2} = 3 \right\}, \\
C &= \left\{ (x,y): x^3-3xy^2+3y=1 \right\}, \\
D &= \left\{ (x,y): 3x^2 y - 3x - y^3 = 0\right\}.
\end{align*}
Prove that $A \cap B = C \cap D$.

\item[A--2]
The sequence of digits
\[
1 2 3 4 5 6 7 8 9 1 0 1 1 1 2 1 3 1 4 1 5 1 6 1 7 1 8 1 9 2 0 2 1 \dots
\]
is obtained by writing the positive integers in order. If the
$10^n$-th digit in this sequence occurs in the part of the sequence in
which the $m$-digit numbers are placed, define $f(n)$ to be $m$. For
example, $f(2)=2$ because the 100th digit enters the sequence in the
placement of the two-digit integer 55. Find, with proof, $f(1987)$.

\item[A--3]
For all real $x$, the real-valued function $y=f(x)$ satisfies
\[
y''-2y'+y=2e^x.
\]
\begin{enumerate}
\item[(a)] If $f(x)>0$ for all real $x$, must $f'(x) > 0$ for all real
$x$? Explain.
\item[(b)] If $f'(x)>0$ for all real $x$, must $f(x) > 0$ for all real
$x$? Explain.
\end{enumerate}

\item[A--4]
Let $P$ be a polynomial, with real coefficients, in three variables
and $F$ be a function of two variables such that
\[
P(ux, uy, uz) = u^2 F(y-x,z-x) \quad \mbox{for all real $x,y,z,u$},
\]
and such that $P(1,0,0)=4$, $P(0,1,0)=5$, and $P(0,0,1)=6$. Also let
$A,B,C$ be complex numbers with $P(A,B,C)=0$ and $|B-A|=10$. Find $|C-A|$.

\item[A--5]
Let
\[
\vec{G}(x,y) = \left( \frac{-y}{x^2+4y^2}, \frac{x}{x^2+4y^2},0
\right).
\]
Prove or disprove that there is a vector-valued function
\[
\vec{F}(x,y,z) = (M(x,y,z), N(x,y,z), P(x,y,z))
\]
with the following properties:
\begin{enumerate}
\item[(i)] $M,N,P$ have continuous partial derivatives for all
$(x,y,z) \neq (0,0,0)$;
\item[(ii)] $\mathrm{Curl}\,\vec{F} = \vec{0}$ for all $(x,y,z) \neq (0,0,0)$;
\item[(iii)] $\vec{F}(x,y,0) = \vec{G}(x,y)$.
\end{enumerate}

\item[A--6]
For each positive integer $n$, let $a(n)$ be the number of zeroes in
the base 3 representation of $n$. For which positive real numbers $x$
does the series
\[
\sum_{n=1}^\infty \frac{x^{a(n)}}{n^3}
\]
converge?

\item[B--1]
Evaluate
\[
\int_2^4 \frac{\sqrt{\ln(9-x)}\,dx}{\sqrt{\ln(9-x)}+\sqrt{\ln(x+3)}}.
\]

\item[B--2]
Let $r,s$ and $t$ be integers with $0 \leq r$, $0 \leq s$ and $r+s
\leq t$. Prove that
\[
\frac{\binom s0}{\binom tr}
+ \frac{\binom s1}{\binom{t}{r+1}} + \cdots
+ \frac{\binom ss}{\binom{t}{r+s}}
= \frac{t+1}{(t+1-s)\binom{t-s}{r}}.
\]

\item[B--3]
Let $F$ be a field in which $1+1 \neq 0$. Show that the set of
solutions to the equation $x^2+y^2=1$ with $x$ and $y$ in $F$ is given
by $(x,y)=(1,0)$ and
\[
(x,y) = \left( \frac{r^2-1}{r^2+1}, \frac{2r}{r^2+1} \right)
\]
where $r$ runs through the elements of $F$ such that $r^2\neq -1$.

\item[B--4]
Let $(x_1,y_1) = (0.8, 0.6)$ and let $x_{n+1} = x_n \cos y_n - y_n
\sin y_n$ and $y_{n+1}= x_n \sin y_n + y_n \cos y_n$ for
$n=1,2,3,\dots$. For each of $\lim_{n\to \infty} x_n$ and $\lim_{n \to
\infty} y_n$, prove that the limit exists and find it or prove that
the limit does not exist.

\item[B--5]
Let $O_n$ be the $n$-dimensional vector $(0,0,\cdots, 0)$. Let $M$ be
a $2n \times n$ matrix of complex numbers such that whenever $(z_1,
z_2, \dots, z_{2n})M = O_n$, with complex $z_i$, not all zero, then at
least one of the $z_i$ is not real. Prove that for arbitrary real
numbers $r_1, r_2, \dots, r_{2n}$, there are complex numbers $w_1,
w_2, \dots, w_n$ such that
\[
\mathrm{re}\left[ M \left( \begin{array}{c} w_1 \\ \vdots \\ w_n \end{array}
\right) \right] = \left( \begin{array}{c} r_1 \\ \vdots \\ r_n
\end{array} \right).
\]
(Note: if $C$ is a matrix of complex numbers, $\mathrm{re}(C)$ is the matrix
whose entries are the real parts of the entries of $C$.)

\item[B--6]
Let $F$ be the field of $p^2$ elements, where $p$ is an odd
prime. Suppose $S$ is a set of $(p^2-1)/2$ distinct nonzero elements
of $F$ with the property that for each $a\neq 0$ in $F$, exactly one
of $a$ and $-a$ is in $S$. Let $N$ be the number of elements in the
intersection $S \cap \{2a: a \in S\}$. Prove that $N$ is even.

\end{itemize}

\end{document}
