\documentclass[amssymb,twocolumn,pra,10pt,aps]{revtex4-1}
\usepackage{mathptmx,amsmath}

\begin{document}
\title{The 63rd William Lowell Putnam Mathematical Competition \\
    Saturday, December 7, 2002}
\maketitle

\begin{itemize}
\item[A1]
Let $k$ be a fixed positive integer. The $n$-th derivative of
$\frac{1}{x^k - 1}$ has the form $\frac{P_n(x)}{(x^k - 1)^{n+1}}$
where $P_n(x)$ is a polynomial. Find $P_n(1)$.

\item[A2]
Given any five points on a sphere, show that some four of them
must lie on a closed hemisphere.

\item[A3]
Let $n \geq 2$ be an integer and $T_n$ be the number of non-empty
subsets $S$ of $\{1, 2, 3, \dots, n\}$ with the property that the
average of the elements of $S$ is an integer. Prove that
$T_n - n$ is always even.

\item[A4]
In Determinant Tic-Tac-Toe, Player 1 enters a 1 in an empty
$3 \times 3$ matrix. Player 0 counters with a 0 in a vacant position,
and play continues in turn until the $3 \times 3$ matrix is
completed with five 1's and four 0's. Player 0 wins if the
determinant is 0 and player 1 wins otherwise. Assuming both
players pursue optimal strategies, who will win and how?

\item[A5]
Define a sequence by $a_0=1$, together with the rules
$a_{2n+1} = a_n$ and $a_{2n+2} = a_n + a_{n+1}$ for each
integer $n \geq 0$. Prove that every positive rational number
appears in the set
\[
\left\{ \frac{a_{n-1}}{a_n}: n \geq 1 \right\} =
\left\{ \frac{1}{1}, \frac{1}{2}, \frac{2}{1}, \frac{1}{3},
\frac{3}{2}, \dots \right\}.
\]

\item[A6]
Fix an integer $b \geq 2$. Let $f(1) = 1$, $f(2) = 2$, and for each
$n \geq 3$, define $f(n) = n f(d)$, where $d$ is the number of
base-$b$ digits of $n$. For which values of $b$ does
\[
\sum_{n=1}^\infty \frac{1}{f(n)}
\]
converge?

\item[B1]
Shanille O'Keal shoots free throws on a basketball court. She hits
the first and misses the second, and thereafter the probability that
she hits the next shot is equal to the proportion of shots she
has hit so far. What is the probability she hits exactly 50 of
her first 100 shots?

\item[B2]
Consider a polyhedron with at least five faces such that exactly three
edges emerge from each of its vertices. Two players play the following
game:
\begin{verse}
\noindent
Each player, in turn, signs his or her name on a previously
unsigned face. The winner is the player who first succeeds in
signing three faces that share a common vertex.
\end{verse}
Show that the player who signs first will always win by playing
as well as possible.

\item[B3]
Show that, for all integers $n > 1$,
\[
\frac{1}{2ne} < \frac{1}{e} - \left( 1 - \frac{1}{n} \right)^n
< \frac{1}{ne}.
\]

\item[B4]
An integer $n$, unknown to you, has been randomly chosen in the
interval $[1, 2002]$ with uniform probability. Your objective is
to select $n$ in an \textbf{odd} number of guesses. After
each incorrect guess, you are informed whether $n$ is higher
or lower, and you \textbf{must} guess an integer on your next turn
among the numbers that are still feasibly correct. Show that you
have a strategy so that the chance of winning is greater than $2/3$.

\item[B5]
A palindrome in base $b$ is a positive integer whose base-$b$
digits read the same backwards and forwards; for example,
$2002$ is a 4-digit palindrome in base 10. Note that 200 is not
a palindrome in base 10, but it is the 3-digit palindrome
242 in base 9, and 404 in base 7. Prove that there is an integer
which is a 3-digit palindrome in base $b$ for at least 2002
different values of $b$.

\item[B6]
Let $p$ be a prime number. Prove that the determinant of the matrix
\[
\begin{pmatrix}
x & y & z \\
x^p & y^p & z^p \\
x^{p^2} & y^{p^2} & z^{p^2}
\end{pmatrix}
\]
is congruent modulo $p$ to a product of polynomials of the form
$ax+by+cz$, where $a,b,c$ are integers. (We say two integer
polynomials are congruent modulo $p$ if corresponding coefficients
are congruent modulo $p$.)

\end{itemize}
\end{document}
