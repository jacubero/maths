\documentclass[amssymb,twocolumn,pra,10pt,aps]{revtex4-1}
\usepackage{mathptmx,amsmath}

\begin{document}
\title{The 61st William Lowell Putnam Mathematical Competition \\
    Saturday, December 2, 2000}
\maketitle

\begin{itemize}
\item[A--1]
Let $A$ be a positive real number.  What are the possible values of
$\sum_{j=0}^\infty x_j^2$, given that $x_0,x_1,\ldots$ are positive
numbers
for which $\sum_{j=0}^\infty x_j=A$?

\item[A--2]
Prove that there exist infinitely many integers $n$ such that
$n,n+1,n+2$ are each the sum of the squares of two integers.
[Example: $0=0^2+0^2$, $1=0^2+1^2$, $2=1^2+1^2$.]


\item[A--3]

The octagon $P_1P_2P_3P_4P_5P_6P_7P_8$ is inscribed in a circle, with
the
vertices around the circumference in the given order.  Given that the
polygon
$P_1P_3P_5P_7$ is a square of area 5, and the polygon $P_2P_4P_6P_8$ is a
rectangle of area 4, find the maximum possible area of the octagon.

\item[A--4]
Show that the improper integral
\[ \lim_{B\to\infty}\int_{0}^B \sin(x) \sin(x^2)\,dx\]
converges.


\item[A--5]
Three distinct points with integer coordinates lie in the plane on a
circle of radius $r>0$.  Show that two of these points are separated by a
distance of at least $r^{1/3}$.

\item[A--6]
Let $f(x)$ be a polynomial with integer coefficients.  Define a
sequence $a_0,a_1,\ldots$ of integers such that $a_0=0$ and
$a_{n+1}=f(a_n)$
for all $n\geq 0$.  Prove that if there exists a positive integer $m$ for
which $a_m=0$ then either $a_1=0$ or $a_2=0$.

\item[B--1]
Let $a_j,b_j,c_j$ be integers for $1\leq j\leq N$.  Assume for each
$j$, at least one of $a_j,b_j,c_j$ is odd.  Show that there exist integers
$r$, $s$, $t$ such that $ra_j+sb_j+tc_j$ is odd for at least $4N/7$ values
of $j$, $1\leq j\leq N$.

\item[B--2]
Prove that the expression
\[ \frac{gcd(m,n)}{n}\binom{n}{m} \]
is an integer for all pairs of integers $n\geq m\geq 1$.

\item[B--3]
Let $f(t)=\sum_{j=1}^N a_j \sin(2\pi jt)$, where each $a_j$ is real
and
$a_N$ is not equal to 0.  Let $N_k$ denote the number of zeroes (including
multiplicities) of $\frac{d^k f}{dt^k}$.
Prove that
\[N_0\leq N_1\leq N_2\leq \cdots \mbox{ and } \lim_{k\to\infty} N_k =
2N.\]
[Editorial clarification: only zeroes in $[0, 1)$ should be counted.]

\item[B--4]
Let $f(x)$ be a continuous function such that $f(2x^2-1)=2xf(x)$ for
all $x$.  Show that $f(x)=0$ for $-1\leq x\leq 1$.

\item[B--5]
Let $S_0$ be a finite set of positive integers.  We define finite
sets
$S_1,S_2,\ldots$ of positive integers as follows:
the integer $a$ is in $S_{n+1}$ if and only if exactly one of $a-1$ or $a$ is
in
$S_n$.
Show that there exist infinitely many integers $N$ for which
$S_N=S_0\cup\{N+a: a\in S_0\}$.


\item[B--6]
Let $B$ be a set of more than $2^{n+1}/n$ distinct points with
coordinates
of the form $(\pm 1,\pm 1,\ldots,\pm 1)$ in $n$-dimensional space with
$n\geq 3$.
Show that there are three distinct points in $B$ which are the vertices of
an
equilateral triangle.

\end{itemize}
\end{document}
